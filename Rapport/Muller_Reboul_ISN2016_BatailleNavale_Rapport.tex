\documentclass[a4paper,12pt,fleqn]{book}

% ---------------------------------------------------------------------------
%				Packages
% ---------------------------------------------------------------------------

% texdoc <nom_package> pour avoir des infos
\usepackage{etex}

\usepackage[utf8]{inputenc}						% Encodage français
\usepackage[frenchb]{babel}						% Mise en forme française
\usepackage[T1]{fontenc}						% Encodage caractères français
\usepackage{pmboxdraw}
\usepackage{newunicodechar}
\newunicodechar{✖}{×}
%\newunicodechar{◯}{o}
\newunicodechar{┪}{\pmboxdrawuni{252A}}
\newunicodechar{┨}{\pmboxdrawuni{2528}}
\newunicodechar{┩}{\pmboxdrawuni{2529}}

\usepackage{fourier}							% Différents symboles et polices
\usepackage[scaled=0.875]{helvet}				% Font générale
\usepackage{courier}							% Font télétype
\renewcommand{\ttdefault}{lmtt}					% Font télétype
\usepackage{frcursive}							% Ecriture manuscrite type écolier
\usepackage{calligra}							% Ecriture manuscrite classieuse
\usepackage{verbatim}							% Commentaires
%\newcommand{\verbatimfont}[1]{\renewcommand{\verbatim@font}{\ttfamily#1}}
%\newcommand{\verbatimfont}[1]{\renewcommand{\verbatim@font}{\ttdefault#1}}

%\usepackage{fancyvrb}

\usepackage[toc,page]{appendix}

\usepackage{listings}
\usepackage{amsfonts,amsmath,amssymb}			% Symboles maths
\usepackage{bm}									% Symboles maths en gras \bm{}
\usepackage{amstext}							% Texte en mode math de taille adaptée
\usepackage{amsopn}								% \DeclareMathOperator
\usepackage{mathrsfs}							% Symboles maths
\usepackage{mathtools}							% Symboles maths
\usepackage{theorem}							% Mise en forme des théorèmes

\usepackage{textcomp}							% Symboles
\usepackage{pifont}								% Symboles "ding"
\usepackage{wasysym}							% Symboles (smiley et logos)
\usepackage{epsdice}							% Symboles (faces d'un dé)
\usepackage[normalem]{ulem}						% Fioritures de texte (barré, etc...)
\usepackage{cancel}								% Barrer du texte (simplifier termes)
\usepackage{fancybox}							% Boîtes

\usepackage{tabularx}							% Tableaux évolués
\usepackage{diagbox}							% Cases en diagonale
%~ \usepackage{tabls}							% Espaces dans les tableaux (conflit avec bclogo)
\usepackage{multirow}							% Fusionner les lignes d'un tableau
\usepackage{enumerate}							% Enumérations personnalisées
\usepackage{multicol}							% Environnement multicolonnes
\usepackage{fancyhdr}							% En-têtes et pieds de page
%~ \usepackage[np]{numprint}					% Mise en forme des nombres

\usepackage[usenames, dvipsnames]{xcolor}		% Couleurs
\usepackage{graphicx}							% Insérer des images
\usepackage{pgf, tikz, tkz-tab, tkz-fct}		% Graphiques avec Tikz
\usetikzlibrary{arrows}
\usetikzlibrary{snakes}
\usepackage{alterqcm}							% QCM
\usepackage{circuitikz}							% Circuit éléctriques
\usepackage[tikz]{bclogo}						% Boites à logo

\usepackage{titlesec}							% Mise en forme des titres de sections
\usepackage{lastpage}							% Dernière page : \pageref{LastPage}

\usepackage{ifthen}								% Programmation conditions
\usepackage{multido}							% Boucles
\usepackage{calc}								% Calculs

\usepackage{footnote}
\makesavenoteenv{tabular}
\usepackage{pdflscape}
\usepackage{adjustbox}
% ---------------------------------------------------------------------------
%				Macros simples (caractères)
% ---------------------------------------------------------------------------

\newcommand{\euro}{\eurologo{}}
\newcommand{\R}{\ensuremath{\mathbb{R}}}
\newcommand{\N}{\ensuremath{\mathbb{N}}}
\newcommand{\D}{\ensuremath{\mathbb{D}}}
\newcommand{\Z}{\ensuremath{\mathbb{Z}}}
\newcommand{\Q}{\ensuremath{\mathbb{Q}}}
\newcommand{\C}{\ensuremath{\mathbb{C}}}
\newcommand{\e}{\text{e}}
\renewcommand{\i}{\text{i}}
\newcommand{\s}{\ensuremath{\mathcal{S}}}
\newcommand{\sol}[1]{\mathcal{S}=\left\lbrace #1 \right\rbrace}
\newcommand{\ou}{\mbox{ ou }}
\newcommand{\et}{\mbox{ et }}
\newcommand{\si}{\mbox{ si }}
\newcommand{\Df}{\ensuremath{\mathcal{D}_f}}
\newcommand{\Cf}{\ensuremath{\mathcal{C}_f}}
\newcommand{\Dg}{\ensuremath{\mathcal{D}_g}}
\newcommand{\Cg}{\ensuremath{\mathcal{C}_g}}

\renewcommand{\P}{\ensuremath{\text{P}}}
\newcommand{\card}{\text{card}}
\newcommand{\E}{\text{E}}
\newcommand{\V}{\text{V}}

\newcommand{\FI}{\textbf{F.I.}}

\newcommand{\eq}{\ \Leftrightarrow\ } % ou \iff
\newcommand{\implique}{\Rightarrow}
\newcommand{\pheq}{\phantom{\eq}}
\newcommand{\egdef}{\stackrel{\textit{déf}}{=}}

\renewcommand{\ge}{\geqslant}
\renewcommand{\le}{\leqslant}
\newcommand{\supeg}{\geqslant}
\newcommand{\infeg}{\leqslant}

\newcommand{\lacco}{\left\lbrace}
\newcommand{\racco}{\right\rbrace}
\newcommand{\labs}{\left|}
\newcommand{\rabs}{\right|}

\newcommand{\inclus}{\subset}
\newcommand{\ninclus}{\not\subset}
\newcommand{\union}{\cup}
\newcommand{\inter}{\cap}

\newcommand{\non}[1]{\text{non(}#1\text{)}}

\newcommand{\dx}{~\text{d}x}
\newcommand{\dt}{~\text{d}t}

\renewcommand{\Re}{\text{Re}}
\renewcommand{\Im}{\text{Im}}
\newcommand{\conj}[1]{\overline{#1}}
\newcommand{\abs}[1]{|#1|}

\newcommand{\pinf}{+\infty}
\newcommand{\minf}{-\infty}
\newcommand{\pminf}{\pm\infty}

\newcommand{\para}{\ /\!\!/\ }

\newcommand{\comb}[2]{\text{C}_{#1}^{#2}}

\newcommand{\vect}[1]{\mathchoice
	{\overrightarrow{\displaystyle\mathstrut#1\,\,}}
	{\overrightarrow{\textstyle\mathstrut#1\,\,}}
	{\overrightarrow{\scriptstyle\mathstrut#1\,\,}}
	{\overrightarrow{\scriptscriptstyle\mathstrut#1\,\,}}}

\def\Oij{$\left(\text{O},~\vect{i},~\vect{j}\right)$}
\def\Oijk{$\left(\text{O},~\vect{i},~ \vect{j},~ \vect{k}\right)$}
\def\Ouv{$\left(\text{O},~\vect{u},~\vect{v}\right)$}

\newcommand{\vu}{\vect{u}}
\newcommand{\vv}{\vect{v}}
\newcommand{\vw}{\vect{w}}
\newcommand{\vn}{\vect{n}}

\newcommand{\veccol}[3]{\left(\begin{array}{c}
{#1}\\{#2}\\{#3}
\end{array}\right)}

\definecolor{gris}{gray}{0.85}
\newcommand{\surl}[1]{\colorbox{gris}{\textbf{#1}}}

\renewcommand{\emph}{\textbf}

\newcommand{\saut}{\ \\}
\newcommand{\lignesep}{\vspace*{5pt}\hrule\vspace*{5pt}}

\newcommand{\fct}[5]{
	\begin{array}[t]{r ccl}
	{#1}\ : \ &{#2}&\longrightarrow&{#3}\\
	&{#4}&\longmapsto&{#5}
	\end{array}}

\newcommand{\encadre}[1]{\fbox{\begin{minipage}{\textwidth}
#1
\end{minipage}}}

%\newcommand{\boite}[1]{\fbox{\Huge\phantom{A}\hspace*{#1}}}
\newcommand{\boite}[1]{\fbox{\rule[-0.2cm]{0pt}{0.6cm}\hspace*{#1}}}


\newcommand{\suit}{\begin{tikzpicture}
\draw[color=white](-0.4em,0em)--(-0.25em,0em);
\draw ((-0.25em,-0.15em)--(0.07em,0.18em);
\draw (0.07em,0.18em) arc (135:0:0.25em);
\draw (0.5em,0em) arc (-180:-45:0.25em);
\draw (0.93em,-0.18em)--(1.36em,0.25em);
\draw (1.01em,0.25em)--(1.36em,0.25em)--(1.36em,-0.1em);
\draw[color=white](1.40em,0em)--(1.55em,0em);
\end{tikzpicture}}


%\newcommand{\suit}{\hookrightarrow}

% Commande programmation Casio

\newcommand{\touche}[1]{\fbox{\texttt{#1}}}

\newcommand{\toucheF}[2]{
$\underset{\text{\scriptsize F#2}}{\touche{#1}}$
}

\newcommand{\sto}{\ensuremath{\rightarrow}}

%\newcommand{\toucheFF}[2]{\begin{tabular}[t]{c}
%\touche{#1}\\
%{\scriptsize F#2}
%\end{tabular}}

\newcommand{\suiv}{$\vartriangleright$} % Menu suivant

\newcommand{\rl}{\begin{tikzpicture}[scale=0.7] % Retour à la ligne de la Casio
\draw[color=white] (0,0)--(0,0);
\draw [<-] (0.2em,0.2em)--(1em,0.2em)--(1em,0.8em);
\end{tikzpicture}}

\newcommand{\disp}{\begin{tikzpicture}[scale=0.7] % Triangle de la Casio
\draw[color=white] (0,0)--(0,0);
\fill (0.2em,0em)--(0.8em,0em)--(0.8em,0.8em)--(0.2em,0em);
\draw[color=white] (1em,1em)--(1em,1em);
\end{tikzpicture}}

\newcommand{\vers}{$\rightarrow$}

\newcommand{\enonce}{\textbf{Énoncé :}}
\newcommand{\solution}{\textbf{Solution :}}
\newcommand{\tq}{~,~}

\newcommand{\xmin}{x_{\text{min}}}
\newcommand{\xmax}{x_{\text{max}}}

% -------------------- Symboles : -------------------------------------------

\newcommand{\happy}{\smiley}
\newcommand{\sad}{\frownie}

\newcommand{\attention}{\danger}
\newcommand{\piege}{\bomb}
\newcommand{\interdit}{\noway}

\newcommand{\facede}[1]{\epsdice{#1}}

\newcommand{\hand}{\text{\ding{43}}}
\newcommand{\victory}{\text{\ding{44}}}

\newcommand{\trefle}{\text{\ding{168}}}
\newcommand{\carreau}{\text{\ding{169}}}
\newcommand{\coeur}{\text{\ding{170}}}
\newcommand{\pique}{\text{\ding{171}}}

\newcommand{\checkbox}{\text{\ding{114}}}
\newcommand{\checkedbox}{\text{\mbox{\ding{114}\hspace{-.7em}\raisebox{.2ex}[1ex]{\ding{51}}}}}

\newcommand{\scisors}{\ding{34}}
\newcommand{\couperici}{\scisors\dotfill\textit{\small{couper ici}}\dotfill\scisors}

% ---------------------------------------------------------------------------
% 				Environnements 
% ---------------------------------------------------------------------------

% ------------------------ Théorèmes ----------------------------------------

\theorembodyfont{\normalfont} \theoremstyle{break}



%\newcommand{\Ex}{\noindent\textbf{Exemple : }}
\newcommand{\Rappel}{\noindent\textbf{Rappel : }}

% ------------------------ Python --------------------------------------------

\newcounter{cptspace}
\newcommand{\tab}[1]{
	\setcounter{cptspace}{#1}
	\whiledo{\value{cptspace}>0}{
		\hspace*{0em}\hspace*{0em}
		\addtocounter{cptspace}{-1}}}

\newcommand{\prompt}{{>}{>}{>}\ }

\newenvironment{python}
	{\par\ttfamily\small\vspace{0.2cm}
	\setbox0=\hbox\bgroup
	\begin{minipage}{\textwidth}
	\vspace{0.2cm}
%	\begin{tabbing}
	}
	{\vspace{0.2cm}
%	\end{tabbing}
	\end{minipage}
	\egroup
    \fbox{\box0}
	\par\rmfamily\normalsize\vspace{0.2cm}\noindent
	}

\newenvironment{algo1}
	{\begin{center}\begin{tabular}{|>{\texttt\bgroup}l<{\egroup}|}
	\hline}
	{\hline
	\end{tabular}\end{center}
	}

%\newenvironment{python}
%	{\par\ttfamily\small\vspace{0.2cm}
%	\begin{bclogo}[couleurBord=black, arrondi = 0.1, logo={}, barre = none]{Code python :}
%%	\begin{minipage}{\textwidth}
%%	\vspace{0.2cm}
%	}
%	{%\vspace{0.2cm}
%%	\end{minipage}
%	\end{bclogo}
%	\par\rmfamily\normalsize\vspace{0.2cm}\noindent
%	}

\newcommand{\pyth}[1]{\texttt{#1}}

% -------------------------- Boites bclogo -----------------------------------
\newenvironment{warning}
	{\begin{bclogo}[couleurBord=black, arrondi = 0.1, logo = \bcattention]}%{\Large\attention}]}
	{\end{bclogo}}
	
\newenvironment{forbidden}
	{\begin{bclogo}[couleurBord=black, arrondi = 0.1, logo = \bcinterdit]}%{\Large\interdit}]}
	{\end{bclogo}}

% ---------------------------------------------------------------------------
% 				Raccoucis clavier
% ---------------------------------------------------------------------------

\newcommand{\fctsurI}[3]{
	Soit $#1$ la fonction définie sur $#3$ par $$#1(x)=#2$$}
	
\newcommand{\hp}{Hors programme de TSTI2D}

% ---------------------------------------------------------------------------
%				Macros évoluées
% ---------------------------------------------------------------------------

% ----------------- \tournerpage --------------------------------------------
% Indique de tourner la page en bas de page
%
\def\tournerpage{\vfill%
	\begin{flushright}	
		\textbf{Tourner la page }$\mathbf{\rightarrow}$
	\end{flushright}
	\newpage}

% ------------------ Exercices : \exo et \pb --------------------------------
% Crée un exo (ou un pb) valant #1 points (si #1=0 alors correction)
% Les exos sont numérotés automatiquement
% 
\newcounter{numexo}
\setcounter{numexo}{0}

\newcommand\exo[1]{\addtocounter{numexo}{1}\par\vspace{1cm}\textbf{\textsc{Exercice \thenumexo}}%
	\ifthenelse{\equal{#1}{0}}{}{ \hfill \textbf{#1 points}}\medskip\par}
	
\newcommand\pb[1]{\par\vspace{1cm}\textbf{\textsc{Problème}}%
	\ifthenelse{\equal{#1}{0}}{}{\hfill \textbf{#1 points}}\medskip\par}

% ------------------ Cartouche : \makecartouche -------------------------------
% {Titre}{{Classe}{Durée}{Calculatrice autorisée}{Ligne pour le nom}

\newcommand{\makecartoucheDsNOM}[4]{
	\ifthenelse{\boolean{#4}}{\newcommand{\calculatrice}{est }}{\newcommand{\calculatrice}{\textbf{n'est pas }}}
	\begin{center}\begin{tabular}{|p{.82\linewidth}|p{.15\linewidth}|}
	\hline
	#3 & \multicolumn{1}{r|}{#2}\\
	%~ \multicolumn{2}{|l|}{#3}\\
	% ----- Nom Prénom ----
	\hline
%	\textbf{\textsc{Nom - Prénom :}} & \textbf{\textsc{Classe :}}\\
	\multicolumn{2}{|l|}{\textbf{\textsc{Nom - Prénom :}}}\\
	\multicolumn{2}{|l|}{}\\
	\multicolumn{2}{|l|}{}\\
	% ---------------------
	\hline
	\multicolumn{2}{|c|}{}\\
	\multicolumn{2}{|c|}{\textbf{\LARGE{#1}}}\\
	\multicolumn{2}{|c|}{}\\
	\hline
	\multicolumn{2}{|c|}{\textit{La calculatrice \calculatrice autorisée. Aucun autre document n'est autorisé.}} \\
	\hline
	\end{tabular}\end{center}
}

\newcommand{\makecartoucheDsPASNOM}[4]{
	\ifthenelse{\boolean{#4}}{\newcommand{\calculatrice}{est }}{\newcommand{\calculatrice}{\textbf{n'est pas }}}
	\begin{center}\begin{tabular}{|p{.82\linewidth}|p{.15\linewidth}|}
	\hline
	#3 & \multicolumn{1}{r|}{#2}\\
	\hline
	\multicolumn{2}{|c|}{}\\
	\multicolumn{2}{|c|}{\textbf{\LARGE{#1}}}\\
	\multicolumn{2}{|c|}{}\\
	\hline
	\multicolumn{2}{|c|}{\textit{La calculatrice \calculatrice autorisée. Aucun autre document n'est autorisé.}} \\
	\hline
	\end{tabular}\end{center}
}

\newcommand{\makecartoucheDS}[5]{\ifthenelse{\boolean{#5}}{\makecartoucheDsNOM{#1}{#2}{#3}{#4}}{\makecartoucheDsPASNOM{#1}{#2}{#3}{#4}}}

\newcommand{\makecartoucheCours}[2]{
	\begin{center}\begin{tabular}{|p{.82\linewidth}|p{.15\linewidth}|}
	\hline
	 & \multicolumn{1}{r|}{#2}\\
	\hline
	\multicolumn{2}{|c|}{}\\
	\multicolumn{2}{|c|}{\textbf{\LARGE{#1}}}\\
	\multicolumn{2}{|c|}{}\\
	\hline
	\end{tabular}\end{center}
}


% --------------------------------- Pages de cahier ------------------------------

% --------------------- \cahierseyes ---------------------------------------------
% Crée une page de cahier réglures seyes (8mm et interlignes) de #1 lignes
%

% Périmé
\newcounter{dimcahierX}
\setcounter{dimcahierX}{16}
\newcounter{dimcahiersY}
\newcommand\cahierseyes[1]{
	\vspace*{-1cm}
	\setcounter{dimcahiersY}{#1*\real{-0.8}}
	\begin{center}
	\begin{tikzpicture}
		\tkzInit[xmin=0,xmax=\thedimcahierX,ymin=\thedimcahiersY,ymax=0]
		\tkzGrid[xstep=0.8, ystep=0.8,sub, subxstep=0.8, subystep=0.2]
	\end{tikzpicture}
	\end{center}
}

% Bon
\newcounter{DimcahierX}
\setcounter{DimcahierX}{16}
\newcounter{DimcahierY}
\newcommand\Cahierseyes[1]{
%	\vspace*{-0.8cm}
	\setcounter{DimcahierY}{#1}
	\begin{center}
	\begin{tikzpicture}
	\draw[xstep=0.8cm, ystep=0.2cm, color=lightgray] (0,0) grid (\theDimcahierX, 0.8*\theDimcahierY);
	\draw[xstep=0.8cm, ystep=0.8cm, color=gray] (0,0) grid (\theDimcahierX, 0.8*\theDimcahierY);
	\end{tikzpicture}
	\end{center}
}


% --------------------- \cahierpetca ---------------------------------------------
% Crée une page de cahier petits carreaux (5mm) de #1 lignes
%

%Périmé
\newcounter{dimcahierpcY}
\newcommand\cahierpetca[1]{
	\setcounter{dimcahierpcY}{#1*\real{0.5}}
	\vspace*{-1cm}
	\begin{center}
	\begin{tikzpicture}
		\tkzInit[xmin=0,xmax=\thedimcahierX,ymin=0,ymax=\thedimcahierpcY]
		\tkzGrid[xstep=0.5, ystep=0.5]
	\end{tikzpicture}
	\end{center}
}

%Bon
\newcounter{DimcahierpcY}
\newcommand\Cahierpetca[1]{
	\setcounter{DimcahierpcY}{#1}
	%\vspace*{-1cm}
	\begin{center}
	\begin{tikzpicture}
	\draw[xstep=0.5cm, ystep=0.5cm, color=gray] (0,0) grid (\theDimcahierX, 0.5*\theDimcahierpcY);
	\end{tikzpicture}
	\end{center}
}



% --------------------- \cahierligne ---------------------------------------------
% Crée une page de cahier avec juste des lignes (10mm) de #1 lignes
%
\newcounter{dimcahierlY}
\newcounter{miX}
\setcounter{miX}{\thedimcahierX*\real{0.5}}
\newcommand\cahierligne[1]{
	\setcounter{dimcahierlY}{#1-1}
	\begin{center}
	\begin{tikzpicture}
		\tkzInit[xmin=0,xmax=\thedimcahierX,ymin=0,ymax=#1]
		\foreach \k in {0,1,...,\thedimcahierlY}
		{\draw[color=gray] (0,\k)--(16,\k);}
		\draw[color=lightgray] (\themiX,0)--(\themiX,#1);
	\end{tikzpicture}
	\end{center}
}

% --------------------- \cahier ---------------------------------------------
% Par défaut, réglures seyes
%
\newcommand\cahier[1]{\Cahierseyes{#1}}

% --------------------- \papiermilli ----------------------------------------
% Crée une feuille de papier millimétré de dimensions x=#1 par y=#2
%
\newcommand\papiermilli[2]{
	\begin{center}
	\begin{tikzpicture}
		\tkzInit[xmin=0,xmax=#1,ymin=0,ymax=#2]
		\tkzGrid[color=lightgray,xstep=1, ystep=1,sub, subxstep=0.1, subystep=0.1]
		\tkzGrid[color=gray,xstep=1, ystep=1,sub, subxstep=0.5, subystep=0.5]
		\tkzGrid[color=darkgray,xstep=5, ystep=5]
	\end{tikzpicture}
	\end{center}
}
% --------------------- \systlin ----------------------------------------
% Système linéaire 2*2
% Paramètres : a, ±b, c, a', ±b', c'
%
\newcommand{\systlin}[6]{
$\left\lbrace
\begin{array}{r @{x~} l @{y~=~} l l}
#1&#2&#3&(L_1)\\
#4&#5&#6&(L_2)
\end{array}
\right.$
}


% ---------------------------------------------------------------------------
%				Formatage
% ---------------------------------------------------------------------------

% --------------------- Chapitres -------------------------------------------
\addto\captionsfrench{\renewcommand{\chaptername}{~Chapitre}}

\titleformat{\chapter}[frame]
	{\vspace*{-2cm}\titleline[r]{}\normalfont}
	{\filright\texttt\LARGE{~\chaptername~\thechapter~}}
	{5pt}
	{\Huge\bfseries\scshape\filcenter}
	{}
	
%~ \titleformat{\chapter}[display]
%~ {\normalfont\Large\filcenter}
%~ {\titlerule[1pt]%
 %~ \vspace{1pt}%
 %~ \titlerule
 %~ \vspace{1pc}%
 %~ \LARGE{\chaptertitlename} \thechapter}
%~ {1pc}
%~ {\titlerule
 %~ \vspace{1pc}%
 %~ \Huge\bfseries}
\renewcommand{\appendixname}{Annexe}
% --------------------  Divisions logiques ----------------------------------

\setcounter{secnumdepth}{4} \setcounter{tocdepth}{3}
\renewcommand{\thepart}{Partie \arabic{part}}
\renewcommand{\thechapter}{\arabic{chapter}}
\renewcommand{\thesection}{\arabic{section}}
\renewcommand{\thesubsection}{\arabic{section}.\arabic{subsection}}
\renewcommand{\thesubsubsection}{\arabic{section}.\arabic{subsection}.\arabic{subsubsection}}

% -------------------- Numérotations des questions et puces -----------------

%\renewcommand{\theenumi}{\textbf{\arabic{enumi}}}
%\renewcommand{\labelenumi}{\textbf{\theenumi.}}
%\renewcommand{\theenumii}{\textbf{\alph{enumii}}}
%\renewcommand{\labelenumii}{\textbf{\theenumii.}}
%\renewcommand{\labelenumiii}{\textbullet}

\AtBeginDocument{\renewcommand{\labelitemi}{\textbullet}}
\AtBeginDocument{\renewcommand{\labelitemii}{\textbullet}}


% % -------------------- Mise en page --------------------------------------- 

\usepackage[francais]{layout}
%\usepackage[a4paper]{geometry}


% % -------------------- Réglages divers------------------------------------- 

\everymath{\displaystyle\everymath{}}		% Toutes les équations en mode \displaystyle
\DecimalMathComma							% Virgule comme séparateur décimal
\frenchspacing								% Espaces français
%\setlength{\parindent}{0pt}					% Pas d'indentation de paragraphes

\colorlet{gris1}{black!20}
\colorlet{gris2}{black!30}

% ----------------------------------------------------------------------------

\lstset{literate=
  {á}{{\'a}}1 {é}{{\'e}}1 {í}{{\'i}}1 {ó}{{\'o}}1 {ú}{{\'u}}1
  {Á}{{\'A}}1 {É}{{\'E}}1 {Í}{{\'I}}1 {Ó}{{\'O}}1 {Ú}{{\'U}}1
  {à}{{\`a}}1 {è}{{\`e}}1 {ì}{{\`i}}1 {ò}{{\`o}}1 {ù}{{\`u}}1
  {À}{{\`A}}1 {È}{{\'E}}1 {Ì}{{\`I}}1 {Ò}{{\`O}}1 {Ù}{{\`U}}1
  {ä}{{\"a}}1 {ë}{{\"e}}1 {ï}{{\"i}}1 {ö}{{\"o}}1 {ü}{{\"u}}1
  {Ä}{{\"A}}1 {Ë}{{\"E}}1 {Ï}{{\"I}}1 {Ö}{{\"O}}1 {Ü}{{\"U}}1
  {â}{{\^a}}1 {ê}{{\^e}}1 {î}{{\^i}}1 {ô}{{\^o}}1 {û}{{\^u}}1
  {Â}{{\^A}}1 {Ê}{{\^E}}1 {Î}{{\^I}}1 {Ô}{{\^O}}1 {Û}{{\^U}}1
  {œ}{{\oe}}1 {Œ}{{\OE}}1 {æ}{{\ae}}1 {Æ}{{\AE}}1 {ß}{{\ss}}1
  {ű}{{\H{u}}}1 {Ű}{{\H{U}}}1 {ő}{{\H{o}}}1 {Ő}{{\H{O}}}1
  {ç}{{\c c}}1 {Ç}{{\c C}}1 {ø}{{\o}}1 {å}{{\r a}}1 {Å}{{\r A}}1
  {€}{{\EUR}}1 {£}{{\pounds}}1
}
\definecolor{mygreen}{rgb}{0,0.6,0}
\definecolor{mygray}{rgb}{0.5,0.5,0.5}
\definecolor{mymauve}{rgb}{0.58,0,0.82}
\lstset{ %
  backgroundcolor=\color{white},   % choose the background color; you must add \usepackage{color} or \usepackage{xcolor}
  basicstyle=\footnotesize,        % the size of the fonts that are used for the code
  breakatwhitespace=false,         % sets if automatic breaks should only happen at whitespace
  breaklines=true,                 % sets automatic line breaking
  captionpos=b,                    % sets the caption-position to bottom
  commentstyle=\color{mygreen},    % comment style
  deletekeywords={...},            % if you want to delete keywords from the given language
  escapeinside={\%*}{*)},          % if you want to add LaTeX within your code
  extendedchars=true,              % lets you use non-ASCII characters; for 8-bits encodings only, does not work with UTF-8
  frame=single,	                   % adds a frame around the code
  keepspaces=true,                 % keeps spaces in text, useful for keeping indentation of code (possibly needs columns=flexible)
  keywordstyle=\color{blue},       % keyword style
  language=Octave,                 % the language of the code
  otherkeywords={*,...},           % if you want to add more keywords to the set
  numbers=left,                    % where to put the line-numbers; possible values are (none, left, right)
  numbersep=5pt,                   % how far the line-numbers are from the code
  numberstyle=\tiny\color{mygray}, % the style that is used for the line-numbers
  rulecolor=\color{black},         % if not set, the frame-color may be changed on line-breaks within not-black text (e.g. comments (green here))
  showspaces=false,                % show spaces everywhere adding particular underscores; it overrides 'showstringspaces'
  showstringspaces=false,          % underline spaces within strings only
  showtabs=false,                  % show tabs within strings adding particular underscores
  stepnumber=2,                    % the step between two line-numbers. If it's 1, each line will be numbered
  stringstyle=\color{mymauve},     % string literal style
  tabsize=2,	                   % sets default tabsize to 2 spaces
%  title=\lstname                   % show the filename of files included with \lstinputlisting; also try caption instead of title
}

% =============================================================================
% 								FIN PREAMBULE
% =============================================================================


% -------------------- Cours --------------------
\renewcommand{\theenumi}{\arabic{enumi}}
\renewcommand{\labelenumi}{\theenumi)}
\renewcommand{\theenumii}{\alph{enumii}}
\renewcommand{\labelenumii}{\theenumii)}
\renewcommand{\labelenumiii}{\textbullet}

\setlength{\textwidth}{156truemm}  %largeur de texte
\setlength{\textheight}{220truemm} %longueur de texte
\setlength{\topmargin}{-0.8cm} %début de text
\setlength{\oddsidemargin}{2truemm}
\setlength{\evensidemargin}{2truemm}
\setlength{\baselineskip}{.2cm} 

\pagestyle{fancy}
\fancyhf{}
\renewcommand{\chaptermark}[1]{\markboth{\bsc{\chaptername~\thechapter{} : #1}}{}}
\renewcommand{\headrulewidth}{0pt}
\renewcommand \footrulewidth{.2pt}
\lhead[\textsl{\leftmark}]{}%{\textsl{\rightmark}}
\rhead[]{\textsl{\leftmark}}
%\lfoot{\texttt{maths.muller@gmail.com}}
%\cfoot{\textsc{}}
%\rfoot{\thepage/\pageref{LastPage}}

\newtheorem{Th}{Théorème}[chapter]
\newtheorem{Dem}{Démonstration}[chapter]
\newtheorem{DemBac}{Démonstration (démo Bac)}[chapter]
\newtheorem{Exmp}{Exemple}[chapter]
\newtheorem{Rem}{Remarque}[chapter]
\newtheorem{Def}{Définition}[chapter]
\newtheorem{Nota}{Notations}[chapter]
\newtheorem{Prop}{Propriété}[chapter]
\newtheorem{Exo}{Exercice}[chapter]
\newtheorem{App}{Application}[chapter]
\newtheorem{Cons}{Conséquence}[chapter]
\newtheorem{Ex}{Exemple}[chapter]
\newtheorem{Algo}{Algorithme}[chapter]
\newtheorem{Lemme}{Lemme}[chapter]


\lfoot[\thepage]{}
\cfoot{\textsc{ISN 2016 - Bataille navale}}
\rfoot[]{\thepage}

\begin{document}


\title{\fbox{\Huge{\textsc{Bataille navale}}}\\ \medskip \medskip \large{Projet de validation ISN 2016\\de l'académie de Lyon}}
%\author{\textsc{Frédéric Muller\\Lionel Reboul}}
\author{\textsc{Frédéric Muller} - \texttt{maths.muller@gmail.com}\\ \textsc{Lionel Reboul} - \texttt{reboul.lionel@free.fr}\\ \\
   }
\date{13 mars 2016}
%\date{\vfill \flushleft \textit{Description}}

\maketitle


\clearpage{\pagestyle{empty}\cleardoublepage}
\setcounter{tocdepth}{1}
\tableofcontents
\thispagestyle{fancy}

%\chapter{Présentation du projet}
\chapter{Présentation du projet}

\section{Le jeu de la bataille navale}
Le jeu de la bataille navale est un jeu qui se joue à deux joueurs.\\
Chaque joueur dispose d'une grille sur laquelle il place des bateaux rectangulaires de différentes tailles et essaie, chacun son tour, de deviner l'emplacement des bateaux de l'adversaire par des tirs successifs, ce dernier annonçant à chaque coup \og manqué \fg{} ou \og touché \fg{}. Nous avons pris le parti de ne pas annoncer \og coulé \fg{} lorsque toutes les cases d'un bateau ont été touchées pour rendre l'algorithme de résolution un petit peu plus intéressant.\\
Les bateaux peuvent être placés horizontalement ou verticalement et deux bateaux ne peuvent pas se trouver sur des cases adjacentes.

Les règles retenues dans ce projet sont les règles du jeu original, mais elles peuvent être facilement modifiées, à savoir que la grille est un carré 10 cases de côté et la composition de la flotte est la suivante :
\begin{itemize}
\item Un bateau de 5 cases
\item Un bateau de 4 cases
\item Deux bateaux de 3 cases
\item Un bateau de 2 cases
\end{itemize}



Notons tout de suite quelques implications stratégiques de ces règles qui seront utilisées dans l'algorithme de résolution :
\begin{itemize}
\item Le plus petit bateau étant de taille 2, il suffit de ne tirer que sur une cases sur 2 (imaginez les cases noires d'un damier) lors de la recherche d'un bateau.
\item Une fois qu'un bateau a été coulé (soit parce que c'est le plus grand de la liste, soit parce que les cases adjacentes à ses extrémités ont été manquées), on peut éliminer de la recherche toutes ses cases adjacentes.
\end{itemize}

\section{Objectifs du projet}
Nos objectifs ont été les suivants :
\begin{itemize}
\item Définir une structure de données pour modéliser la grille de jeu, ainsi que les joueurs.
\item Implémenter un algorithme de résolution par l'ordinateur qui soit le plus performant possible (en nombre de coups ainsi qu'en temps de résolution d'un grille) et en faire une étude statistique complète.
\item Avoir une interface permettant de jouer contre l'ordinateur. Cette interface a été réalisée d'un part en mode console avec un affichage grâce à des caractères graphiques (en unicode) et, d'autre part, avec le module tkinter. 
\end{itemize}

\section{Liste des modules du projet}
Afin de faciliter les développement et la maintenance du projet, celui-ci a été décomposé en un certain nombre de modules :
\begin{itemize}
\item \texttt{main.py} : le programme principal. Il permet, via un argument \texttt{--interface} en ligne de commande de choisir l'interface de jeu (\texttt{console} ou \texttt{tkinter}).
\item \texttt{bn\_utiles.py} : contient quelques fonctions utiles ainsi que les constantes du projet.
\item \texttt{bn\_grille.py} : gère la grille et les bateaux.
\item \texttt{bn\_joueur.py} : gère les joueurs et implémente l'algorithme de résolution.
\item \texttt{bn\_console.py} : toute l'interface en mode console, et l'étude statistique de l'algorithme de résolution.
\end{itemize}

\section{Constantes de direction}
Les constantes suivantes, définies dans le module \texttt{bn\_utiles.py} indiquent les différentes directions, et sont utilisées dans tous le projet :
\begin{itemize}
\item \texttt{BN\_DROITE = (1, 0)}
\item \texttt{BN\_GAUCHE = (-1, 0)}
\item \texttt{BN\_BAS = (1, 0)}
\item \texttt{BN\_HAUT = (-1, 0)}
\end{itemize}
Ainsi que :
\begin{itemize}
\item \texttt{BN\_ALLDIR = (1, 1)} (toutes les direction)
\item \texttt{BN\_HORIZONTAL = (1, 0)} (à gauche et à droite)
\item \texttt{BN\_VERTICAL = (0, 1)} (en haut et en bas)
\end{itemize}
Elles permettent de rendre le code plus clair et plus compact.

%\chapter{Fonctions utiles}
\chapter{Quelques fonctions utiles}

\section{Coordonnées des cases}
Les cases de la grille sont codées par des tuples $(x,y)$, où $x$ est la colonne et $y$ la ligne. Aussi nous utilisons la fonction \texttt{alpha(case)} qui, à partir des coordonnées, retourne sa représentation naturelle (par exemple \texttt{'B4'}) et \texttt{coord(case\_alpha)} qui réalise l'opération réciproque.

Ces deux fonctions utilisent le code ASCII.

\section{Constantes de direction}
Les constantes suivantes, définies dans le module \texttt{bn\_utiles.py}, indiquent les différentes directions, et sont utilisées dans tout le projet :
\begin{itemize}
\item \texttt{BN\_DROITE = (1, 0)}
\item \texttt{BN\_GAUCHE = (-1, 0)}
\item \texttt{BN\_BAS = (0, 1)}
\item \texttt{BN\_HAUT = (0, -1)}
\end{itemize}
Ainsi que :
\begin{itemize}
\item \texttt{BN\_ALLDIR = (1, 1)} (toutes les direction)
\item \texttt{BN\_HORIZONTAL = (1, 0)} (à gauche et à droite)
\item \texttt{BN\_VERTICAL = (0, 1)} (en haut et en bas)
\end{itemize}
Elles permettent de rendre le code plus clair et plus compact.

\section{Paramètres en ligne de commande}
Le lancement du programme principal \texttt{main.py} admet un paramètre en ligne de commande. Ce paramètre est géré par le module \texttt{argparse}. La prototype est le suivant :

\begin{verbatim}
$ main.py -h
usage: main.py [-h] [--interface INTERFACE]

Jeu de bataille navale

optional arguments:
  -h, --help            show this help message and exit
  --interface INTERFACE, -i INTERFACE
                        Choix de l'interface : 'console' ou 'tkinter'
\end{verbatim}

%\chapter{Gestion de la grille}
\chapter{Gestion de la grille}\label{chap_grille}

La gestion de la grille et des bateaux est effectuée dans le module \texttt{bn\_grille.py}.

\section{La classe Bateau}
Cette classe, très minimaliste, définit un bateau par sa case de départ \texttt{Bateau.start}, sa taille \texttt{Bateau.taille} et sa direction \texttt{Bateau.sens}. Elle permet de récupérer :
\begin{itemize}
\item sa case de fin \texttt{Bateau.end},
\item la liste de ses cases occupées \texttt{Bateau.cases},
\item la liste de ses cases adjacentes \texttt{Bateau.cases\_adj}.
\end{itemize}

\section{La classe Grille}
\subsection{Présentation}
Cette classe est au c\oe ur du projet. Elle permet de mémoriser l'état de chaque case de la grille et d'effectuer des opérations comme :
\begin{itemize}
\item Gérer la liste des bateaux de la flotte : placer un bateau à une position donnée ou aléatoirement, placer une flotte aléatoire, supprimer un bateau coulé, ou encore garder la trace du plus grand bateau restant à couler.
\item Déterminer le nombre de cases vides autour d'une case donnée, dans chacune des directions.
\item Déterminer la liste, et le nombre, de bateaux possibles sur chaque case.
\item Déterminer lorsque la grille est terminée.
\end{itemize}
Beaucoup de ces fonctionnalités seront utilisées par l'algorithme de résolution.

\medskip

Afin de pouvoir faire évoluer les règles, elle prend les paramètres suivants lors de son initialisation :
\begin{itemize}
\item \texttt{xmax} et \texttt{ymax} : les dimensions de la grille
\item \texttt{taille\_bateaux} : la liste des bateaux
\end{itemize}
\medskip

Dans la mesure où la grille a deux utilisations différentes (la grille du joueur et la grille de suivi des coups), j'ai d'abord décidé de créer deux classes héritées de \texttt{Grille} lors de la conception du projet, \texttt{GrilleJoueur(Grille)} et \texttt{GrilleSuivi(Grille)}, afin de distinguer leurs méthodes spécifiques. Après coup je me suis rendu compte que cela n'apportait pas d'avantage significatif en terme de qualité de code donc je ne les utiliserai pas, mais elles sont encore présentes dans mon code pour une évolution future du projet.

\subsection{État de la grille}

L'attribut \texttt{Grille.etat} fournit l'état de la grille. C'est un dictionnaire indexé par les tuples \texttt{(0,0)}, \texttt{(0,1)},... , \texttt{(9,9)}, dans lesquels la première coordonnée correspond à la colonne de la case et la deuxième à sa ligne.\\
L'état d'une case peut être :
\begin{itemize}
\item $0$ : case vide
\item $1$ : case touchée (ou contenant un bateau)
\item $-1$ : case manquée ou impossible
\end{itemize}
%L'intérêt d'utiliser un dictionnaire plutôt qu'une liste double tient au fait que les appels sont plus simples et plus naturels et, surtout, que l'utilisation d'une table de hachage permet la recherche d'un élément en $O(1)$.  

\medskip

La méthode \texttt{Grille.test\_case(self, case)} permet de déterminer si une case est valide et vide, et \texttt{Grille.is\_touche(self, case)} indique si une case donnée contient ou non un bateau.

\medskip

Notons également l'utilisation de l'attribut \texttt{Grille.vides} qui est la liste des cases vides. Bien entendu, cette classe contient une fonction \texttt{Grille.update(self)} de mise à jour jour de l'état de la grille (liste des cases vides, tailles du plus petit et du plus grand bateau restant).

\medskip

Enfin, la méthode \texttt{Grille.adjacent(self, case)} renvoie la liste de cases adjacentes à une case donnée.

\subsection{Gestion des espaces vides}
La méthodes \texttt{Grille.get\_max\_space(self, case, direction, face=True)} renvoie le nombre de cases vides dans une direction donnée. Grâce aux constantes de direction, un seul calcul est nécessaire pour englober tous les cas (horizontal et vertical). L'algorithme est donné en annexe \ref{algo_liste}, page \pageref{get_max_space}.
%\begin{algo1}
%direction[0]\sto dh\\
%direction[1]\sto dv\\
%0\sto m\\
%case[0]\sto x\\
%case[1]\sto y\\
%Tant que la case (x+dh, y+dv) est vide et est dans la grille :\\
%\tab{1} m+1\sto m\\
%\tab{1} x+dh\sto x\\
%\tab{1} y+dv\sto y\\
%Retourner m\\
%\end{algo1}

Si le paramètre \texttt{face==True}, la détermination se fait dans les deux sens (espace libre total horizontal ou vertical).

\medskip

La méthode \texttt{Grille.elimine\_cases(self)} parcourt toutes les cases vides et élimine celles dans lesquelles le plus petit bateau ne peut pas rentrer en mettant leur état à \texttt{-1}.

\subsection{Liste des bateaux possibles sur chaque case}
La méthode \texttt{Grille.get\_possibles(self)} renvoie d'une part la liste des bateaux possibles démarrant sur chaque case (ainsi que leurs directions) et, d'autre part, la liste des positions (et directions) de départ possibles pour chaque bateau. Pour ce faire on procède en deux temps :
\begin{itemize}
\item Dans un premier temps, on parcourt la liste des cases vides et pour chacune de ces cases on détermine, pour chaque bateau et chaque direction (droite et bas), s'il peut démarrer sur cette case. Cela fournit le dictionnaire \texttt{Grille.possibles\_cases} indexé par les cases et dont les éléments sont une liste de tuples de la forme \texttt{(taille, direction)}.\\
Par exemple : \texttt{\{(0,0):[(5,(1,0)), (5,(0,1)),...], (0,1):...\}}
\item Dans un deuxième temps, on "retourne" ce dictionnaire pour obtenir le dictionnaire \texttt{Grille.possibles} indexé par les tailles des bateaux et dont les éléments sont une liste de tuples de la forme \texttt{(case, direction)}.\\
Par exemple : \texttt{\{5:[((0,0), (1,0)), ((0,0), (0,1)), ((1,0), (1,0)),...], 4:...\}}
\end{itemize}

\medskip

Cette méthode va servir à faire plusieurs choses :
\begin{itemize}
\item Créer une flotte aléatoire grâce au dictionnaire \texttt{Grille.possibles}, dans la méthode \texttt{Grille.init\_bateaux\_alea(self)}.
\item Déterminer la case optimale dans l'algorithme de résolution au niveau 5.
\item Optimiser la file d'attente dans les algorithmes de résolution.
\item Déterminer tous les arrangements possibles de bateaux sur la grille.
\end{itemize}

L'algorithme complet de cette méthode est donné en annexe \ref{algo_liste}, page \pageref{get_possibles}.

% -------------------------------------------------------------
\begin{comment}
\subsection{Nombre de possibilités sur chaque case}
L'une des parties importantes de l'algorithme de résolution consiste en la détermination de la case dans laquelle rentrent le plus de bateaux. Cette question intervient lors de la phase de tirs en aveugle et, lorsqu'on a touché une première case, qu'on doit tester ses cases adjacentes (phase de tir ciblé).
\subsubsection{Optimisation de la phase de tir en aveugle}\label{opti_aveugle}
La méthode \texttt{Grille.case\_max(self)} renvoie la case vide contenant le plus de bateaux, ainsi que le nombre de bateaux qu'elle contient. L'algorithme est très simple : d'abord on crée un dictionnaire \texttt{Grille.probas} indexé par les cases et contenant le nombre de bateaux possibles grâce à \texttt{Grille.possibles}. Ensuite il ne reste plus qu'à renvoyer celle qui en contient le plus.

\subsubsection{Optimisation de la phase de tir ciblé}\label{opti_touche}
Cette optimisation est un petit peu plus délicate. Une fois qu'une case a été touchée, l'algorithme va tester ses 4 (au maximum) cases adjacentes et les ranger en ordre décroissant du nombre de bateaux possibles. C'est le rôle de la méthode \texttt{Grille.case\_max\_touchee(self, case\_touchee)}. 

Notons \texttt{(x, y)} les coordonnées de la case touchée et intéressons nous au nombre de bateaux possibles sur les cases adjacentes horizontales (pour les verticales, c'est exactement la même chose). Pour chaque taille de bateau à couler possible contenant \texttt{case\_touchee} il faudra distinguer trois cas :
\begin{enumerate}
\item Le bateau est à gauche de \texttt{case\_touchee} et se termine sur cette case. Dans ce cas on augmente de 1 le nombre de possibilités de la case à gauche \texttt{(x-1, y)}
\item Le bateau est à cheval sur \texttt{case\_touchee}. Dans ce cas on augmente de 1 le nombre de possibilités de la case à gauche \texttt{(x-1, y)} et de celle à droite \texttt{(x+1, y)}
\item Le bateau est à droite de \texttt{case\_touchee} et commence sur cette case. Dans ce cas on augmente de 1 le nombre de possibilités de la case à droite \texttt{(x+1, y)}
\end{enumerate}

\textbf{Exemple :}

Imaginons que, sur une grille vierge, on vienne de toucher la case de coordonnées $(5,0)$ et regardons le nombre de façons de placer le bateau de taille 4 à gauche et à droite :
\begin{enumerate}
\item Le bateau rentre à gauche de la case $(5,0)$ :

\begin{center}
\begin{tikzpicture}
\draw (0,1)--(10,1);
\draw (-1,0)--(10,0);
\draw (-1,-1)--(10,-1);
\foreach \x in {0,1,...,10}{
\draw (\x,1)--(\x,-1);
}
\draw (-1,0)--(-1,-1);
\foreach \x in {0,1,...,9}{
\draw (\x+0.5,0.5) node{\x};
}
\draw (-0.5, -0.5) node{$0$};
\draw (5.5, -0.5) node{\textbf{\textsf{X}}};

\draw (2.5, -0.5) node{\textsf{X}};
\draw (3.5, -0.5) node{\textsf{X}};
\draw (4.5, -0.5) node{\textsf{X}};
\end{tikzpicture}\\
\textit{La case $(4,0)$ est augmentée de 1
}\end{center}

\item Le bateau est à cheval sur la case $(5,0)$ (2 possibilités) :
\begin{center}

\begin{tikzpicture}
\draw (0,1)--(10,1);
\draw (-1,0)--(10,0);
\draw (-1,-1)--(10,-1);
\foreach \x in {0,1,...,10}{
\draw (\x,1)--(\x,-1);
}
\draw (-1,0)--(-1,-1);
\foreach \x in {0,1,...,9}{
\draw (\x+0.5,0.5) node{\x};
}
\draw (-0.5, -0.5) node{$0$};
\draw (5.5, -0.5) node{\textbf{\textsf{X}}};

\draw (6.5, -0.5) node{\textsf{X}};
\draw (3.5, -0.5) node{\textsf{X}};
\draw (4.5, -0.5) node{\textsf{X}};
\end{tikzpicture}
%\textit{Les cases $(4,0)$ et $(6,0)$ sont augmentées de 1
%}\end{center}
%
%\begin{center}

\medskip

\begin{tikzpicture}
\draw (0,1)--(10,1);
\draw (-1,0)--(10,0);
\draw (-1,-1)--(10,-1);
\foreach \x in {0,1,...,10}{
\draw (\x,1)--(\x,-1);
}
\draw (-1,0)--(-1,-1);
\foreach \x in {0,1,...,9}{
\draw (\x+0.5,0.5) node{\x};
}
\draw (-0.5, -0.5) node{$0$};
\draw (5.5, -0.5) node{\textbf{\textsf{X}}};

\draw (6.5, -0.5) node{\textsf{X}};
\draw (7.5, -0.5) node{\textsf{X}};
\draw (4.5, -0.5) node{\textsf{X}};
\end{tikzpicture}\\
\textit{Les cases $(4,0)$ et $(6,0)$ sont augmentées de 2
}\end{center}

\item Le bateau rentre à droite de la case $(5,0)$ :

\begin{center}
\begin{tikzpicture}
\draw (0,1)--(10,1);
\draw (-1,0)--(10,0);
\draw (-1,-1)--(10,-1);
\foreach \x in {0,1,...,10}{
\draw (\x,1)--(\x,-1);
}
\draw (-1,0)--(-1,-1);
\foreach \x in {0,1,...,9}{
\draw (\x+0.5,0.5) node{\x};
}
\draw (-0.5, -0.5) node{$0$};
\draw (5.5, -0.5) node{\textbf{\textsf{X}}};

\draw (6.5, -0.5) node{\textsf{X}};
\draw (7.5, -0.5) node{\textsf{X}};
\draw (8.5, -0.5) node{\textsf{X}};
\end{tikzpicture}\\
\textit{La case $(6,0)$ est augmentée de 1
}\end{center}
\end{enumerate} 
Au final, la case $(4,0)$ admet 3 bateaux horizontaux de taille 4 et idem pour la case $(6,0)$.

Si la case $(3,0)$ avait été jouée et manquée nous aurions obtenu 1 bateau horizontal de taille 4 possible sur la case $(4,0)$ et 2 sur la case $(6,0)$ :

\begin{center}
\begin{tikzpicture}
\draw (0,1)--(10,1);
\draw (-1,0)--(10,0);
\draw (-1,-1)--(10,-1);
\foreach \x in {0,1,...,10}{
\draw (\x,1)--(\x,-1);
}
\draw (-1,0)--(-1,-1);
\foreach \x in {0,1,...,9}{
\draw (\x+0.5,0.5) node{\x};
}
\draw (-0.5, -0.5) node{$0$};
\draw (5.5, -0.5) node{\textbf{\textsf{X}}};

\draw (3.5, -0.5) node{\textsf{O}};
\end{tikzpicture}\\
\end{center}

Une fois que le compte des bateaux possibles a été effectué sur chacune des cases adjacentes, on crée une liste \texttt{probas\_liste} contenant des tuples de la forme \texttt{(case, probas[case])} que l'on ordonne en ordre décroissant de possibilités grâce à l'instruction \texttt{sorted(probas\_liste, key=lambda proba: proba[1], reverse = True)} et que l'on retourne.

\end{comment}
% -------------------------------------------------------------


\subsection{Gestion de la flotte}
Le classe \texttt{Grille} contient toutes les méthodes nécessaires pour gérer la flotte de bateaux. Les méthodes \texttt{Grille.get\_taille\_max(self)} et \texttt{Grille.get\_taille\_min(self)} mettent à jour respectivement la taille maximum et la taille minimum des bateaux restant à trouver. La méthode \texttt{Grille.rem\_bateau(self, taille)} permet de supprimer de la liste \texttt{Grille.taille\_bateaux} un bateau coulé.

\subsubsection{Ajout d'un bateau}
La méthode \texttt{Grille.add\_bateau(self, bateau)} permet d'ajouter un bateau (instance de la classe \texttt{Bateau}) après avoir testé sa validité via la méthode \texttt{Grille.test\_bateau(self, bateau)}, et marque ses cases adjacentes comme impossibles.

\subsubsection{Création d'un flotte aléatoire}
La méthode \texttt{Grille.add\_bateau\_alea(self, taille)} permet d'ajouter un bateau aléatoire de taille donnée sur la grille.

Pour créer une flotte aléatoire, on utilise la méthode \texttt{Grille.init\_bateaux\_alea(self)} dont l'algorithme est donné en annexe \ref{algo_liste}, page \pageref{init_alea}.

%\begin{algo1}
%0\sto nb\_bateaux\\
%Tant que nb\_bateaux < nombre de bateaux à placer :\\
%\tab{1} 0\sto nb\_bateaux\\
%\tab{1} On crée une copie temporaire de la grille dans gtmp\\
%\tab{1} Pour chaque taille de bateau à placer :\\
%\tab{2} On récupère les positions possibles pour ce bateau dans gtmp\\
%\tab{2} Si aucune possibilité :\\
%\tab{3} On casse la boucle et on recommence tout\\
%\tab{4} (pour éviter une situation de blocage)\\
%\tab{2} Sinon :\\
%\tab{3} On choisit une position et une direction au hasard\\
%\tab{4} (parmi celles possibles)\\
%\tab{3} On ajoute le bateau à gtmp\\
%\tab{3} nb\_bateaux+1\sto nb\_bateaux\\
%Enfin on copie l'état de gtmp dans notre grille \\
%\end{algo1}

\subsection{Fin de la partie}
L'attribut \texttt{Grille.somme\_taille}, initialisé dès le départ avec la classe \texttt{Grille}, contient le nombre total de cases à toucher. La méthode \texttt{Grille.fini(self)} compare donc ce nombre avec le nombre de cases touchée dans \texttt{Grille.etat} pour déterminer si la grille a été résolue.

%\chapter{Gestion des joueurs et de la partie}
\chapter{Gestion des joueurs et de la partie}

La gestion des joueurs et du déroulement de la partie se font dans le module \texttt{bn\_joueur.py} mais les classes \texttt{Joueur} et \texttt{Partie} sont très minimales et seront largement héritées dans la suite (que ce soit par la classe \texttt{Ordi} qui implémente l'algorithme de résolution, que pour les différentes interfaces (console et graphique)).

\section{La classe Joueur}
Lors de son initialisation, on peut donner un nom au joueur et on initialise sa grille de jeu (\texttt{Joueur.grille\_joueur}), la grille de l'adversaire (\texttt{Joueur.grille\_adverse}) ainsi que sa grille de suivi des coups (\texttt{Joueur.grille\_suivi}).

On en profite aussi pour initialiser quelques variables d'état comme la liste des coups déjà joués et le nombre de coups joués.

La méthode principale de cette classe est \texttt{Joueur.tire(self, case)} qui permet de tirer sur une case et d'avoir en retour le résultat du coup (y compris si le coup n'est pas valide).

Notons l'attribut \texttt{Joueur.messages} qui est une liste contenant différents messages d'information (comme par exemple "A2 : Touché", ou encore les messages indiquant comment l'algorithme résout la grille). Lors de l'affichage des messages, il suffit de vider cette liste grâce à des \texttt{pop(0)} successifs en affichant chaque élément pour avoir un suivi.

\section{La classe Partie}
Ici encore, un squelette et des méthodes très générales pour une classe qui sera héritée dans les interfaces.

Elle se contente de définir l'adversaire (notons l'instruction \begin{center}
\texttt{isinstance(self.adversaire, Ordi)}
\end{center} qui permet de savoir que ce dernier est l'ordinateur), de placer les bateaux du joueur et de récupérer les paramètres de l'adversaire (sa grille et le coup qu'il vient de jouer).

À la base nous voulions faire un mode de jeu en réseau et c'est ici que ce serait trouvées les instructions de communication.


%\chapter{Algorithme de résolution}
\chapter{Algorithme de résolution}

\section{Description de l'algorithme}
L'algorithme de résolution est implémenté dans la classe \texttt{Ordi(Joueur)} du module \texttt{bn\_joueur.py} (qui hérite donc de la classe \texttt{Joueur}. Il fonctionne en deux temps : dans un premier temps une phase de tir en aveugle et, une fois qu'une case a été touchée, une phase de tir ciblé jusqu'à ce que le bateau soit coulé.

\subsection{Phase de tir en aveugle}
Lors de cette phase, l'algorithme va tirer sur la case qui peut contenir le plus de bateau comme vu au chapitre \ref{chap_grille}, section \ref{opti_aveugle}.

C'est la méthode la plus efficace que nous ayons trouvé. Néanmoins nous avons fait d'autres essais avec d'autres méthodes mais celles-ci étaient beaucoup moins performantes, que ce soit aussi bien en nombre de coup pour la résolution qu'en temps :
\begin{itemize}
\item La première méthode consiste à tout simplement tirer au hasard sur une case vide.
\item On peut raffiner la méthode précédente en ne tirant que sur une case sur deux (le plus petit bateau étant de taille 2, chaque bateau tombe obligatoirement sur une case noire du damier).
\item Nous avons aussi essayé de déterminer la case la plus probable en créant un échantillon d'un certain nombre $n$ de répartitions aléatoires des bateaux restant sur le grille et en comptant, pour chaque case, le nombre de bateaux la contenant. Les performances en nombre d'essais étaient satisfaisantes, mais le temps de calcul beaucoup trop élevé. Voici un petit tableau récapitulatif de quelques essais avec différents paramètres :

\begin{center}
\begin{tabular}{|l|c|c|c|c|}
\hline
Taille des échantillons & 100 & 1\,000 & 10\,000 & 100\,000\\
\hline
Nombre de parties & 10\,000 & 10\,000 & 1\,000 & 100\\
\hline
Nombre de coups moyens & 43,68 & 43,30 & 42,72 & 42,63\\
\hline
Temps moyen par partie (en secondes) & 0,38 & 3,6 & 36,2 & 380\\
\hline 
\end{tabular}\\
Temps mesurés sur un processeur Intel Core i7 4800-MQ à 2,7 GHz
\end{center}

Au final, le temps de résolution étant linéaire en $n$ pour des performances négligeables, cette approche a été abandonnée.

\item Enfin, une dernière approche consisterait à déterminer tous les arrangements de bateaux possibles sur la grille à chaque coup, de manière récursive. Cette approche semble optimale mais malheureusement, vu le nombre astronomique de configurations, cette approche est irréalisable que ce soit en temps de calcul qu'en utilisation mémoire. 

\end{itemize}  

\subsection{Phase de tir ciblé}

\subsection{Algorithme complet}

\section{Étude statistique}
 

%\chapter{Affichage console}
\chapter{Affichage console}
Le module \texttt{bn\_console.py} implémente l'interface en mode console.

L'idée de cette interface est de rendre hommage au style de jeu des années 80 en essayant d'en garder au maximum l'esprit.

\section{Préliminaires}
\subsection{Constantes graphiques}
Pour afficher les grilles nous utilisons des caractères graphiques en unicode (famille Box Drawing de codes U2500 à U257F). Ceux-ci donnent tous les outils afin de fabriquer des grilles, y compris avec des caractères gras. Pour des raisons de commodité, le code de chacun des caractères utilisés est stocké dans une constante (par exemple \texttt{CAR\_CX=u'\textbackslash u253C'} correspond à la croix centrale). La lise des codes caractères utilisés est donnée en annexe \ref{annexe_codescar} page \pageref{annexe_codescar}.

\subsection{Effacer le terminal}
Le module \texttt{os} permet d'une part d'accéder à la version du système d'exploitation avec \texttt{os.name} et, d'autre part, de lancer des commandes système avec \texttt{os.system(commande)}. La combinaison de ces deux commandes permet facilement de pouvoir effacer l'écran en utilisant la commande \texttt{cls} sous Windows et \texttt{clear} sous Linux.

\subsection{Fusion des deux grilles}
Lors d'une partie contre un adversaire, il faut pouvoir afficher côte à côte la grille de suivi du joueur ainsi que sa propre grille avec, au fur et à mesure, les coups joués par l'adversaire. Afin de réaliser cette opération nous utilisons la fonction \texttt{fusion(chaine1, chaine2)}. Celle-ci prend en entrée deux chaînes de caractères et retourne la chaîne fusionnée de la façon suivante : chaque chaîne est convertie en liste en prenant comme séparateur le caractère de retour de ligne \texttt{'\textbackslash n'} grâce à la méthode \texttt{String.join('\textbackslash n')}. Ensuite, en prenant les éléments à tour de rôle les éléments de chacune des listes et en insérant un caractère de trait vertical entre les deux on crée la chaîne fusionnée. 

\subsection{Autres fonctions d'affichage}
La fonction \texttt{centre(chaine, longueur)} centre la chaîne sur un espace de longueur donnée en insérant le nombre d'espaces nécessaires. Cette fonction sera utilisée pour l'affichage des noms des joueurs.

La fonction \texttt{boite(texte, prefixe, longueur)} permet d'encadrer le texte dans une boîte de longueur donnée, chaque ligne étant précédée d'un préfixe. Cette fonction sera utilisée pour afficher la liste des messages pour chaque joueur à chaque tour, les préfixe servant à identifier l'auteur du message.

Notons enfin qu'afin de pouvoir réutiliser le code de ce module dans d'autres contextes (comme une interface en \texttt{tkinter}), la fonction \texttt{print()} a été encapsulée dans une fonction \texttt{info(*args)} de sorte qu'il suffit de surcharger cette dernière pour envoyer l'affichage ailleurs (par exemple dans une boîte de texte dans une fenêtre graphique)

\section{Affichage des grilles}
La classe \texttt{GrilleC(Grille)} hérite de la classe \texttt{Grille} en ajoutant uniquement les fonctions d'affichage.

\subsection{Affichage simple de la grille}
La méthode \texttt{GrilleC.make\_chaine(self)} crée la chaîne de caractères de la grille simple. En plus des coins, chaque case utilise 3 caractère horizontaux (ce qui permet de centrer un symbole) et 1 caractère vertical.

Pour cet affichage, on crée les lignes les unes après les autres (dans deux boucles impbriquées) en marquant les cases suivant les valeurs de \texttt{Grille.etat}. La seule subtilité provient des deux premières et de la dernière ligne (à cause des coins).


\subsection{Affichage de la grille avec ses propres bateaux en gras}
Cette partie est beaucoup plus délicate. L'idée est d'afficher une grille en entourant ses propres bateaux et en marquant les coups joués par l'adversaire (sa grille de suivi). C'est le rôle de la fonction \texttt{GrilleC.make\_chaine\_adverse(self, grille)}, où \texttt{grille} est soit sa propre grille de jeu, soit celle de l'adversaire si on veut tricher (pour les tests bien sûr...) ou en fin de partie, si on a perdu, pour avoir la solution. Par convention, comme \texttt{grille} est une grille de jeu, nous allons noter dans les explications les seuls états possibles par $1$ si la case est occupée par un bateau et $0$ sinon (ou si la case est hors grille).

Les contraintes que nous nous fixons sont les suivantes :
\begin{itemize}
\item Les bords des bateaux doivent être en gras
\item Les séparations à l'intérieur d'un bateau doivent être en clair
\item Lorsque deux bateaux se touchent par un coin, il faut bien sûr que ces coins soit en gras (soit une croix en gras)
\end{itemize}


Le bord des cases de la grille se fera sur la ligne du bas (hormis la première ligne sous les lettres) et sur la séparation verticale de gauche (hormis pour la dernière colonne). Le cas de la dernière ligne horizontale et de la dernière ligne verticale se fera à part à cause du coin.

\begin{enumerate}
\item Première ligne, sous les lettres des colonnes : pour chaque case de la ligne $0$ on va tester son état, ainsi que l'état de la case de gauche (pour savoir si on est en début ou en fin de bateau, ou au milieu d'un bateau). On obtient les configurations suivantes (la case testée est celle de droite) :

\begin{verbatim}
───┼───  ━━━┿━━━   ━━━╅─── ───╆━━━
 0   0    1   1     1   0   0   1
\end{verbatim}


\item Lignes suivantes, jusqu'à l'avant dernière : ici c'est plus délicat car il faut tester, en plus de celle de gauche, la case en-dessous et celle en-dessous à gauche pour obtenir les configuration suivantes (la case testée est celle en haut à droite) :

\begin{verbatim}
 1 │ 1    0 ┃ 1    0 ┃ 1    0 ┃ 1
   ┿━━━     ╄━━━     ╋━━━     ╂───
 0   0    0   0    1   0    0   1
  
 
 1 ┃ 0    1 ┃ 0    1 ┃ 0    0 │ 0    0 │ 0    0 │ 0    0 │ 0
   ╃───     ╂───     ╋━━━     ┿━━━     ┼───     ╅───     ╆━━━
 0   0    1   0    0   1    1   1    0   0    1   0    0   1
\end{verbatim}

\item Enfin, pour la dernière colonne va juste tester la case en-dessous, et pour la dernière ligne, on va juste tester celle de gauche. Pour la case tout en bas à droite il faudra juste finir en mettant un coin.

\begin{verbatim}
 0 │      0 │      1 ┃      1 ┃  
   ┤        ┪        ┨        ┩
 0        1        1        0
 
 1   1    1 ┃ 0    0   1    0 │ 0
   ┷━━━     ┹───     ┺━━━     ┴───

 1 ┃      0 │
   ┛        ┘
\end{verbatim}
\end{enumerate}

Le résultat est visible dans l'annexe \ref{annexe_algo_action}, page \pageref{annexe_algo_action}.





\section{Modes de jeu}
\subsection{Jeu solo}

\subsection{Résolution d'un grille par l'ordinateur}

\subsection{Jeu contre l'ordinateur}


%\chapter{Interface graphique}
\chapter{Interface graphique}

Une interface réalisée grâce à la bibliothèque \texttt{tkinter} est codée dans le fichier \texttt{bn\_tkinter}. Dans cette interface, on a la possibilité de joueur seul sur une grille aléatoire, de voir l'algorithme de résolution (avec choix du niveau) et de jouer contre l'algorithme (avec choix du niveau de ce dernier et placement de nos bateaux).

Cette interface devait être réalisée par mon binôme, mais ce dernier ayant abandonné la formation quelques semaines avant las date de rendu de projet, j'ai du la coder très rapidement et, malgré l'urgence, je suis assez satisfait du résultat.

\section{Principes de l'interface}

Le point principal se situe dans la classe \texttt{GrilleTK(Grille)} qui implémente les fonctionnalités graphiques (avec un \texttt{canvas}) des grilles.

La fenêtre principale est basée sur une \texttt{Frame},\texttt{main\_frame}, dans laquelle on va placer les grilles, ainsi que des fenêtres de texte. Entre chaque partie on l'efface en supprimant tous ses widgets enfants, obtenus dans la liste \texttt{main\_frame.pack\_slaves()}, grâce à leurs méthodes \texttt{destroy()}. 

J'ai également créé deux fenêtres annexes pour placer nos bateaux (\texttt{PlaceWin}) et choisir pour le niveau de l'algorithme (\texttt{LevelWin}).

Enfin les classes \texttt{JoueurTK(Joueur)} et \texttt{OrdiTK(Ordi, JoueurTK)} implémentent les fonctions graphiques, et notamment la gestion des événements souris, aux joueurs. Notons que pour pouvoir accéder à la boîte de texte \texttt{info} de l'interface principale qui affiche les informations de partie, on met l'attribut \texttt{name="info"} à ce widget et on y accède depuis la classe \texttt{JoueurTK(Joueur)} via \texttt{self.parent.children["info"]}, où \texttt{self.parent} est la frame principale (\texttt{main\_frame}).

\section{Affichage et gestion des grilles}

%\chapter{Guide des modules utilisés}
%\chapter{Guide des modules utilisés}

%\chapter{Point de vue pédagogique}
\chapter{Point de vue pédagogique}

Bien évidemment, ce projet dépasse largement ce qui est exigible d'un élève (même très bon) de lycée. Certains points peuvent néanmoins être abordés en simplifiant certaines parties et en l'abordant soit comme une série de TP guidés (les élèves doivent coder le contenu des fonctions dont on leur donne le prototype), soit comme projet de fin d'année. On peut aborder les points suivants :
\begin{itemize}
\item La structure de la grille : un bon exemple de codage d'une structure complexe (définir les états des cases, utilisation d'un dictionnaire ou d'une liste double, tests des cases valides,...).
\item Le placement de bateaux : sûrement la partie la moins évidente, mais oblige à réfléchir sur la façon de définir un bateau.
\item L'affichage (simple) de la grille en console : utilisation de boucles imbriquées et de tests pour afficher les bons symboles, et gestion de la mises en page.
\item Éventuellement une interface graphique en utilisant des boutons ou un canevas pour les cases.
\item La possibilité pour un joueur de tirer sur une case et retour du résultat.
\item Une résolution de la grille par l'ordinateur avec uniquement des tirs aléatoires sur les cases vides (les plus en avance pourront réfléchir à des méthodes plus évoluées).
\end{itemize}



%\chapter{Conclusion}
\chapter{Conclusion}
\section{Conclusion}
Ce travail a été très stimulant et m'a pris un temps conséquent (plus de 200 heures et 3500 lignes de code), mais j'y ai pris beaucoup de plaisir. Il a d'abord fallu mettre en place des structures de données pour avoir un projet minimal. Ensuite est venu le temps de coder l'algorithme de résolution et son optimisation, les outils statistiques, la construction de l'interface console, puis de l'interface web et enfin l'interface en \texttt{tkinter} réalisée dans l'urgence. Lors de ces différentes phases, de nombreux problèmes sont survenus et leurs résolutions m'ont permis de progresser.

\medskip

La rédaction du rapport en \LaTeX\ a également été très plaisante, avec quelques figures réalisées avec \texttt{tikz}, l'affichage des caractères unicodes, ou encore la rapatriement automatique des docstrings. 

\section{Évolutions possibles}
Il reste de nombreux points à développer pour une évolution futur du projet :
\begin{itemize}
\item Mise en place d'une architecture réseau
\item Gestion et sauvegarde des paramètres (nom du joueur, paramètres de la grille,...)
\item Sauvegarde des scores du joueur et statistiques de jeu
\item Un hébergement de l'interface web et une gestion des utilisateurs (inscription, connexion, choix de l'adversaire, statistiques des joueurs, tournois, points,...)
\item On pourrait également facilement faire une interface pour tablette grâce à la bibliothèque \texttt{kivy} (\texttt{https://kivy.org/})
\item Ou encore une interface avec \texttt{pygame} pour avoir quelque chose de plus ludique (sprites pour les bateaux, effets d'explosion quand on touche, bruitages...)
\item On pourrait aussi envisager une bataille spatiale, sur une grille à trois dimensions  
\end{itemize}


%Certes il reste des points à développer, comme la mise en place d'une architecture de jeu en réseau, ou encore une gestion des paramètres ou une sauvegarde des scores dans l'interface \texttt{tkinter}, mais je pense que ce projet est déjà bien abouti. 

%\medskip


%
%Cela m'a rappelé mes années d'étude, et notamment mon DESS IM dans lequel je faisais beaucoup de projets informatiques de ce type.
%
%Enfin j'ai trouvé un grand intérêt dans l'obligation de rendre un travail propre et fini, ce qui n'est pas le cas dans un projet personnel, dans lequel on est moins exigeant. Cela faisait longtemps que je n'avais pas fait ça et, rien que pour ça, je suis heureux d'avoir suivi cette formation.


%\DeclareUnicodeCharacter{253C}{\CX}
%\unichar{253C}

%\lstinputlisting[language=Python]{../bnutiles.py}

\part*{Annexes}


%\begin{appendices}
\appendix
%\addto\captionsfrench{\renewcommand{\chaptername}{~Annexe}}

\chapter{L'algorithme en action} \label{annexe_algo_action}
{\scriptsize
\begin{verbatim}
    ┌───┬───┬───┬───┬───┬───┬───┬───┬───┬───┐
    │ A │ B │ C │ D │ E │ F │ G │ H │ I │ J │
┌───┼───┼───┼───┼───┼───┼───┼───┼───┼───┼───┤
│ 0 │   │   │   │   │   │   │   │   │   │   │
├───┼───┼───┼───┼───┼───┼───┼───┼───┼───┼───┤
│ 1 │   │   │   │   │   │   │   │   │   │   │
├───┼───┼───┼───┼───┼───┼───┼───┼───┼───╆━━━┪
│ 2 │   │   │   │   │   │   │   │   │   ┃   ┃
├───┼───┼───┼───┼───┼───┼───┼───┼───┼───╂───┨
│ 3 │   │   │   │   │   │   │   │   │   ┃   ┃
├───┼───╆━━━┿━━━┿━━━╅───┼───┼───┼───┼───╄━━━┩
│ 4 │   ┃   │   │   ┃   │   │   │   │   │   │
├───┼───╄━━━┿━━━┿━━━╃───┼───┼───┼───┼───┼───┤
│ 5 │   │   │   │   │   │   │   │   │   │   │
├───┼───┼───┼───┼───╆━━━┿━━━┿━━━┿━━━┿━━━╅───┤
│ 6 │   │   │   │   ┃   │   │   │   │   ┃   │
├───┼───┼───┼───┼───╄━━━┿━━━┿━━━┿━━━┿━━━╃───┤
│ 7 │   │   │   │   │   │   │   │   │   │   │
├───┼───┼───┼───┼───╆━━━┿━━━┿━━━┿━━━╅───┼───┤
│ 8 │   │   │   │   ┃   │   │   │   ┃   │   │
├───┼───╆━━━┿━━━┿━━━╋━━━┿━━━┿━━━┿━━━╃───┼───┤
│ 9 │   ┃   │   │   ┃   │   │   │   │   │   │
└───┴───┺━━━┷━━━┷━━━┹───┴───┴───┴───┴───┴───┘

╔══════════════════════════════════════════════════════════════════════════════════════════════════╗
║ <HAL> C'est parti !!!                                                                            ║
╚══════════════════════════════════════════════════════════════════════════════════════════════════╝
\end{verbatim}}
\newpage

{\scriptsize
\begin{verbatim}
    ┌───┬───┬───┬───┬───┬───┬───┬───┬───┬───┐
    │ A │ B │ C │ D │ E │ F │ G │ H │ I │ J │
┌───┼───┼───┼───┼───┼───┼───┼───┼───┼───┼───┤
│ 0 │   │   │   │   │   │   │   │   │   │   │
├───┼───┼───┼───┼───┼───┼───┼───┼───┼───┼───┤
│ 1 │   │   │   │   │   │   │   │   │   │   │
├───┼───┼───┼───┼───┼───┼───┼───┼───┼───╆━━━┪
│ 2 │   │   │   │   │   │   │   │   │   ┃   ┃
├───┼───┼───┼───┼───┼───┼───┼───┼───┼───╂───┨
│ 3 │   │   │   │   │   │   │   │   │   ┃   ┃
├───┼───╆━━━┿━━━┿━━━╅───┼───┼───┼───┼───╄━━━┩
│ 4 │   ┃   │   │   ┃   │   │   │   │   │   │
├───┼───╄━━━┿━━━┿━━━╃───┼───┼───┼───┼───┼───┤
│ 5 │   │   │   │   │   │ ◯ │   │   │   │   │
├───┼───┼───┼───┼───╆━━━┿━━━┿━━━┿━━━┿━━━╅───┤
│ 6 │   │   │   │   ┃   │   │   │   │   ┃   │
├───┼───┼───┼───┼───╄━━━┿━━━┿━━━┿━━━┿━━━╃───┤
│ 7 │   │   │   │   │   │   │   │   │   │   │
├───┼───┼───┼───┼───╆━━━┿━━━┿━━━┿━━━╅───┼───┤
│ 8 │   │   │   │   ┃   │   │   │   ┃   │   │
├───┼───╆━━━┿━━━┿━━━╋━━━┿━━━┿━━━┿━━━╃───┼───┤
│ 9 │   ┃   │   │   ┃   │   │   │   │   │   │
└───┴───┺━━━┷━━━┷━━━┹───┴───┴───┴───┴───┴───┘

╔══════════════════════════════════════════════════════════════════════════════════════════════════╗
║ <HAL> Je tire sur la case F5 qui est la plus probable (34 bateaux possibles)                     ║
║ <HAL> F5 : Manqué                                                                                ║
╚══════════════════════════════════════════════════════════════════════════════════════════════════╝
\end{verbatim}}
\newpage

{\scriptsize
\begin{verbatim}
    ┌───┬───┬───┬───┬───┬───┬───┬───┬───┬───┐
    │ A │ B │ C │ D │ E │ F │ G │ H │ I │ J │
┌───┼───┼───┼───┼───┼───┼───┼───┼───┼───┼───┤
│ 0 │   │   │   │   │   │   │   │   │   │   │
├───┼───┼───┼───┼───┼───┼───┼───┼───┼───┼───┤
│ 1 │   │   │   │   │   │   │   │   │   │   │
├───┼───┼───┼───┼───┼───┼───┼───┼───┼───╆━━━┪
│ 2 │   │   │   │   │   │   │   │   │   ┃   ┃
├───┼───┼───┼───┼───┼───┼───┼───┼───┼───╂───┨
│ 3 │   │   │   │   │   │   │   │   │   ┃   ┃
├───┼───╆━━━┿━━━┿━━━╅───┼───┼───┼───┼───╄━━━┩
│ 4 │   ┃   │   │   ┃ ◯ │   │   │   │   │   │
├───┼───╄━━━┿━━━┿━━━╃───┼───┼───┼───┼───┼───┤
│ 5 │   │   │   │   │   │ ◯ │   │   │   │   │
├───┼───┼───┼───┼───╆━━━┿━━━┿━━━┿━━━┿━━━╅───┤
│ 6 │   │   │   │   ┃   │   │   │   │   ┃   │
├───┼───┼───┼───┼───╄━━━┿━━━┿━━━┿━━━┿━━━╃───┤
│ 7 │   │   │   │   │   │   │   │   │   │   │
├───┼───┼───┼───┼───╆━━━┿━━━┿━━━┿━━━╅───┼───┤
│ 8 │   │   │   │   ┃   │   │   │   ┃   │   │
├───┼───╆━━━┿━━━┿━━━╋━━━┿━━━┿━━━┿━━━╃───┼───┤
│ 9 │   ┃   │   │   ┃   │   │   │   │   │   │
└───┴───┺━━━┷━━━┷━━━┹───┴───┴───┴───┴───┴───┘

╔══════════════════════════════════════════════════════════════════════════════════════════════════╗
║ <HAL> Je tire sur la case E4 qui est la plus probable (34 bateaux possibles)                     ║
║ <HAL> E4 : Manqué                                                                                ║
╚══════════════════════════════════════════════════════════════════════════════════════════════════╝
\end{verbatim}}
\newpage

{\scriptsize
\begin{verbatim}
    ┌───┬───┬───┬───┬───┬───┬───┬───┬───┬───┐
    │ A │ B │ C │ D │ E │ F │ G │ H │ I │ J │
┌───┼───┼───┼───┼───┼───┼───┼───┼───┼───┼───┤
│ 0 │   │   │   │   │   │   │   │   │   │   │
├───┼───┼───┼───┼───┼───┼───┼───┼───┼───┼───┤
│ 1 │   │   │   │   │   │   │   │   │   │   │
├───┼───┼───┼───┼───┼───┼───┼───┼───┼───╆━━━┪
│ 2 │   │   │   │   │   │   │   │   │   ┃   ┃
├───┼───┼───┼───┼───┼───┼───┼───┼───┼───╂───┨
│ 3 │   │   │   │   │   │   │   │   │   ┃   ┃
├───┼───╆━━━┿━━━┿━━━╅───┼───┼───┼───┼───╄━━━┩
│ 4 │   ┃   │   │   ┃ ◯ │   │   │   │   │   │
├───┼───╄━━━┿━━━┿━━━╃───┼───┼───┼───┼───┼───┤
│ 5 │   │   │   │   │   │ ◯ │   │   │   │   │
├───┼───┼───┼───┼───╆━━━┿━━━┿━━━┿━━━┿━━━╅───┤
│ 6 │   │   │   │   ┃   │   │ ✖ │   │   ┃   │
├───┼───┼───┼───┼───╄━━━┿━━━┿━━━┿━━━┿━━━╃───┤
│ 7 │   │   │   │   │   │   │   │   │   │   │
├───┼───┼───┼───┼───╆━━━┿━━━┿━━━┿━━━╅───┼───┤
│ 8 │   │   │   │   ┃   │   │   │   ┃   │   │
├───┼───╆━━━┿━━━┿━━━╋━━━┿━━━┿━━━┿━━━╃───┼───┤
│ 9 │   ┃   │   │   ┃   │   │   │   │   │   │
└───┴───┺━━━┷━━━┷━━━┹───┴───┴───┴───┴───┴───┘

╔══════════════════════════════════════════════════════════════════════════════════════════════════╗
║ <HAL> Je tire sur la case G6 qui est la plus probable (32 bateaux possibles)                     ║
║ <HAL> G6 : Touché                                                                                ║
║ <HAL> J'ajoute la case H6 à la file d'attente                                                    ║
║ <HAL> J'ajoute la case F6 à la file d'attente                                                    ║
║ <HAL> J'ajoute la case G5 à la file d'attente                                                    ║
║ <HAL> J'ajoute la case G7 à la file d'attente                                                    ║
║ <HAL> J'ordonne ma file d'attente en fonction des possibilités :                                 ║
║ <HAL> F6 : 12 bateaux possibles                                                                  ║
║ <HAL> G5 : 12 bateaux possibles                                                                  ║
║ <HAL> H6 : 11 bateaux possibles                                                                  ║
║ <HAL> G7 : 11 bateaux possibles                                                                  ║
║ <HAL> File d'attente : F6 G5 H6 G7                                                               ║
╚══════════════════════════════════════════════════════════════════════════════════════════════════╝
\end{verbatim}}
\newpage

{\scriptsize
\begin{verbatim}
    ┌───┬───┬───┬───┬───┬───┬───┬───┬───┬───┐
    │ A │ B │ C │ D │ E │ F │ G │ H │ I │ J │
┌───┼───┼───┼───┼───┼───┼───┼───┼───┼───┼───┤
│ 0 │   │   │   │   │   │   │   │   │   │   │
├───┼───┼───┼───┼───┼───┼───┼───┼───┼───┼───┤
│ 1 │   │   │   │   │   │   │   │   │   │   │
├───┼───┼───┼───┼───┼───┼───┼───┼───┼───╆━━━┪
│ 2 │   │   │   │   │   │   │   │   │   ┃   ┃
├───┼───┼───┼───┼───┼───┼───┼───┼───┼───╂───┨
│ 3 │   │   │   │   │   │   │   │   │   ┃   ┃
├───┼───╆━━━┿━━━┿━━━╅───┼───┼───┼───┼───╄━━━┩
│ 4 │   ┃   │   │   ┃ ◯ │   │   │   │   │   │
├───┼───╄━━━┿━━━┿━━━╃───┼───┼───┼───┼───┼───┤
│ 5 │   │   │   │   │   │ ◯ │   │   │   │   │
├───┼───┼───┼───┼───╆━━━┿━━━┿━━━┿━━━┿━━━╅───┤
│ 6 │   │   │   │   ┃   │ ✖ │ ✖ │   │   ┃   │
├───┼───┼───┼───┼───╄━━━┿━━━┿━━━┿━━━┿━━━╃───┤
│ 7 │   │   │   │   │   │   │   │   │   │   │
├───┼───┼───┼───┼───╆━━━┿━━━┿━━━┿━━━╅───┼───┤
│ 8 │   │   │   │   ┃   │   │   │   ┃   │   │
├───┼───╆━━━┿━━━┿━━━╋━━━┿━━━┿━━━┿━━━╃───┼───┤
│ 9 │   ┃   │   │   ┃   │   │   │   │   │   │
└───┴───┺━━━┷━━━┷━━━┹───┴───┴───┴───┴───┴───┘

╔══════════════════════════════════════════════════════════════════════════════════════════════════╗
║ <HAL> Je tire sur la case F6 de la file d'attente                                                ║
║ <HAL> F6 : Touché                                                                                ║
║ <HAL> Le bateau touché est horizontal                                                            ║
║ <HAL> J'enlève la case G5 de la file d'attente                                                   ║
║ <HAL> J'enlève la case G7 de la file d'attente                                                   ║
║ <HAL> J'ajoute la case E6 à la file d'attente                                                    ║
║ <HAL> File d'attente : H6 E6                                                                     ║
╚══════════════════════════════════════════════════════════════════════════════════════════════════╝
\end{verbatim}}
\newpage

{\scriptsize
\begin{verbatim}
    ┌───┬───┬───┬───┬───┬───┬───┬───┬───┬───┐
    │ A │ B │ C │ D │ E │ F │ G │ H │ I │ J │
┌───┼───┼───┼───┼───┼───┼───┼───┼───┼───┼───┤
│ 0 │   │   │   │   │   │   │   │   │   │   │
├───┼───┼───┼───┼───┼───┼───┼───┼───┼───┼───┤
│ 1 │   │   │   │   │   │   │   │   │   │   │
├───┼───┼───┼───┼───┼───┼───┼───┼───┼───╆━━━┪
│ 2 │   │   │   │   │   │   │   │   │   ┃   ┃
├───┼───┼───┼───┼───┼───┼───┼───┼───┼───╂───┨
│ 3 │   │   │   │   │   │   │   │   │   ┃   ┃
├───┼───╆━━━┿━━━┿━━━╅───┼───┼───┼───┼───╄━━━┩
│ 4 │   ┃   │   │   ┃ ◯ │   │   │   │   │   │
├───┼───╄━━━┿━━━┿━━━╃───┼───┼───┼───┼───┼───┤
│ 5 │   │   │   │   │   │ ◯ │   │   │   │   │
├───┼───┼───┼───┼───╆━━━┿━━━┿━━━┿━━━┿━━━╅───┤
│ 6 │   │   │   │   ┃   │ ✖ │ ✖ │ ✖ │   ┃   │
├───┼───┼───┼───┼───╄━━━┿━━━┿━━━┿━━━┿━━━╃───┤
│ 7 │   │   │   │   │   │   │   │   │   │   │
├───┼───┼───┼───┼───╆━━━┿━━━┿━━━┿━━━╅───┼───┤
│ 8 │   │   │   │   ┃   │   │   │   ┃   │   │
├───┼───╆━━━┿━━━┿━━━╋━━━┿━━━┿━━━┿━━━╃───┼───┤
│ 9 │   ┃   │   │   ┃   │   │   │   │   │   │
└───┴───┺━━━┷━━━┷━━━┹───┴───┴───┴───┴───┴───┘

╔══════════════════════════════════════════════════════════════════════════════════════════════════╗
║ <HAL> Je tire sur la case H6 de la file d'attente                                                ║
║ <HAL> H6 : Touché                                                                                ║
║ <HAL> J'ajoute la case I6 à la file d'attente                                                    ║
║ <HAL> File d'attente : E6 I6                                                                     ║
╚══════════════════════════════════════════════════════════════════════════════════════════════════╝
\end{verbatim}}
\newpage

{\scriptsize
\begin{verbatim}
    ┌───┬───┬───┬───┬───┬───┬───┬───┬───┬───┐
    │ A │ B │ C │ D │ E │ F │ G │ H │ I │ J │
┌───┼───┼───┼───┼───┼───┼───┼───┼───┼───┼───┤
│ 0 │   │   │   │   │   │   │   │   │   │   │
├───┼───┼───┼───┼───┼───┼───┼───┼───┼───┼───┤
│ 1 │   │   │   │   │   │   │   │   │   │   │
├───┼───┼───┼───┼───┼───┼───┼───┼───┼───╆━━━┪
│ 2 │   │   │   │   │   │   │   │   │   ┃   ┃
├───┼───┼───┼───┼───┼───┼───┼───┼───┼───╂───┨
│ 3 │   │   │   │   │   │   │   │   │   ┃   ┃
├───┼───╆━━━┿━━━┿━━━╅───┼───┼───┼───┼───╄━━━┩
│ 4 │   ┃   │   │   ┃ ◯ │   │   │   │   │   │
├───┼───╄━━━┿━━━┿━━━╃───┼───┼───┼───┼───┼───┤
│ 5 │   │   │   │   │   │ ◯ │   │   │   │   │
├───┼───┼───┼───┼───╆━━━┿━━━┿━━━┿━━━┿━━━╅───┤
│ 6 │   │   │   │   ┃ ✖ │ ✖ │ ✖ │ ✖ │   ┃   │
├───┼───┼───┼───┼───╄━━━┿━━━┿━━━┿━━━┿━━━╃───┤
│ 7 │   │   │   │   │   │   │   │   │   │   │
├───┼───┼───┼───┼───╆━━━┿━━━┿━━━┿━━━╅───┼───┤
│ 8 │   │   │   │   ┃   │   │   │   ┃   │   │
├───┼───╆━━━┿━━━┿━━━╋━━━┿━━━┿━━━┿━━━╃───┼───┤
│ 9 │   ┃   │   │   ┃   │   │   │   │   │   │
└───┴───┺━━━┷━━━┷━━━┹───┴───┴───┴───┴───┴───┘

╔══════════════════════════════════════════════════════════════════════════════════════════════════╗
║ <HAL> Je tire sur la case E6 de la file d'attente                                                ║
║ <HAL> E6 : Touché                                                                                ║
║ <HAL> J'ajoute la case D6 à la file d'attente                                                    ║
║ <HAL> File d'attente : I6 D6                                                                     ║
╚══════════════════════════════════════════════════════════════════════════════════════════════════╝
\end{verbatim}}
\newpage

{\scriptsize
\begin{verbatim}
    ┌───┬───┬───┬───┬───┬───┬───┬───┬───┬───┐
    │ A │ B │ C │ D │ E │ F │ G │ H │ I │ J │
┌───┼───┼───┼───┼───┼───┼───┼───┼───┼───┼───┤
│ 0 │   │   │   │   │   │   │   │   │   │   │
├───┼───┼───┼───┼───┼───┼───┼───┼───┼───┼───┤
│ 1 │   │   │   │   │   │   │   │   │   │   │
├───┼───┼───┼───┼───┼───┼───┼───┼───┼───╆━━━┪
│ 2 │   │   │   │   │   │   │   │   │   ┃   ┃
├───┼───┼───┼───┼───┼───┼───┼───┼───┼───╂───┨
│ 3 │   │   │   │   │   │   │   │   │   ┃   ┃
├───┼───╆━━━┿━━━┿━━━╅───┼───┼───┼───┼───╄━━━┩
│ 4 │   ┃   │   │   ┃ ◯ │   │   │   │   │   │
├───┼───╄━━━┿━━━┿━━━╃───┼───┼───┼───┼───┼───┤
│ 5 │   │   │   │   │   │ ◯ │   │   │   │   │
├───┼───┼───┼───┼───╆━━━┿━━━┿━━━┿━━━┿━━━╅───┤
│ 6 │   │   │   │   ┃ ✖ │ ✖ │ ✖ │ ✖ │ ✖ ┃   │
├───┼───┼───┼───┼───╄━━━┿━━━┿━━━┿━━━┿━━━╃───┤
│ 7 │   │   │   │   │   │   │   │   │   │   │
├───┼───┼───┼───┼───╆━━━┿━━━┿━━━┿━━━╅───┼───┤
│ 8 │   │   │   │   ┃   │   │   │   ┃   │   │
├───┼───╆━━━┿━━━┿━━━╋━━━┿━━━┿━━━┿━━━╃───┼───┤
│ 9 │   ┃   │   │   ┃   │   │   │   │   │   │
└───┴───┺━━━┷━━━┷━━━┹───┴───┴───┴───┴───┴───┘

╔══════════════════════════════════════════════════════════════════════════════════════════════════╗
║ <HAL> Je tire sur la case I6 de la file d'attente                                                ║
║ <HAL> I6 : Touché                                                                                ║
║ <HAL> J'ajoute la case J6 à la file d'attente                                                    ║
║ <HAL> File d'attente : D6 J6                                                                     ║
║ <HAL> Bateau de taille 5 coulé car c'est le plus grand restant                                   ║
║ <HAL> Je vide ma file d'attente                                                                  ║
╚══════════════════════════════════════════════════════════════════════════════════════════════════╝
\end{verbatim}}
\newpage

{\scriptsize
\begin{verbatim}
    ┌───┬───┬───┬───┬───┬───┬───┬───┬───┬───┐
    │ A │ B │ C │ D │ E │ F │ G │ H │ I │ J │
┌───┼───┼───┼───┼───┼───┼───┼───┼───┼───┼───┤
│ 0 │   │   │   │   │   │   │   │   │   │   │
├───┼───┼───┼───┼───┼───┼───┼───┼───┼───┼───┤
│ 1 │   │   │   │   │   │   │   │   │   │   │
├───┼───┼───┼───┼───┼───┼───┼───┼───┼───╆━━━┪
│ 2 │   │   │   │ ◯ │   │   │   │   │   ┃   ┃
├───┼───┼───┼───┼───┼───┼───┼───┼───┼───╂───┨
│ 3 │   │   │   │   │   │   │   │   │   ┃   ┃
├───┼───╆━━━┿━━━┿━━━╅───┼───┼───┼───┼───╄━━━┩
│ 4 │   ┃   │   │   ┃ ◯ │   │   │   │   │   │
├───┼───╄━━━┿━━━┿━━━╃───┼───┼───┼───┼───┼───┤
│ 5 │   │   │   │   │ ◯ │ ◯ │ ◯ │ ◯ │ ◯ │   │
├───┼───┼───┼───┼───╆━━━┿━━━┿━━━┿━━━┿━━━╅───┤
│ 6 │   │   │   │ ◯ ┃ ✖ │ ✖ │ ✖ │ ✖ │ ✖ ┃ ◯ │
├───┼───┼───┼───┼───╄━━━┿━━━┿━━━┿━━━┿━━━╃───┤
│ 7 │   │   │   │   │ ◯ │ ◯ │ ◯ │ ◯ │ ◯ │   │
├───┼───┼───┼───┼───╆━━━┿━━━┿━━━┿━━━╅───┼───┤
│ 8 │   │   │   │   ┃   │   │   │   ┃   │   │
├───┼───╆━━━┿━━━┿━━━╋━━━┿━━━┿━━━┿━━━╃───┼───┤
│ 9 │   ┃   │   │   ┃   │   │   │   │   │   │
└───┴───┺━━━┷━━━┷━━━┹───┴───┴───┴───┴───┴───┘

╔══════════════════════════════════════════════════════════════════════════════════════════════════╗
║ <HAL> Bateau de taille 5 coulé ! Je l'enlève de la liste des bateaux à chercher                  ║
║ <HAL> Bateaux restant à couler : 4 3 3 2                                                         ║
║ <HAL> J'élimine la case adjacente G5                                                             ║
║ <HAL> J'élimine la case adjacente G7                                                             ║
║ <HAL> J'élimine la case adjacente F7                                                             ║
║ <HAL> J'élimine la case adjacente H5                                                             ║
║ <HAL> J'élimine la case adjacente H7                                                             ║
║ <HAL> J'élimine la case adjacente D6                                                             ║
║ <HAL> J'élimine la case adjacente E5                                                             ║
║ <HAL> J'élimine la case adjacente E7                                                             ║
║ <HAL> J'élimine la case adjacente J6                                                             ║
║ <HAL> J'élimine la case adjacente I5                                                             ║
║ <HAL> J'élimine la case adjacente I7                                                             ║
║ <HAL> Je tire sur la case D2 qui est la plus probable (23 bateaux possibles)                     ║
║ <HAL> D2 : Manqué                                                                                ║
╚══════════════════════════════════════════════════════════════════════════════════════════════════╝
\end{verbatim}}
\newpage

{\scriptsize
\begin{verbatim}
    ┌───┬───┬───┬───┬───┬───┬───┬───┬───┬───┐
    │ A │ B │ C │ D │ E │ F │ G │ H │ I │ J │
┌───┼───┼───┼───┼───┼───┼───┼───┼───┼───┼───┤
│ 0 │   │   │   │   │   │   │   │   │   │   │
├───┼───┼───┼───┼───┼───┼───┼───┼───┼───┼───┤
│ 1 │   │   │   │   │   │   │   │   │   │   │
├───┼───┼───┼───┼───┼───┼───┼───┼───┼───╆━━━┪
│ 2 │   │   │   │ ◯ │   │   │   │   │   ┃   ┃
├───┼───┼───┼───┼───┼───┼───┼───┼───┼───╂───┨
│ 3 │   │   │ ◯ │   │   │   │   │   │   ┃   ┃
├───┼───╆━━━┿━━━┿━━━╅───┼───┼───┼───┼───╄━━━┩
│ 4 │   ┃   │   │   ┃ ◯ │   │   │   │   │   │
├───┼───╄━━━┿━━━┿━━━╃───┼───┼───┼───┼───┼───┤
│ 5 │   │   │   │   │ ◯ │ ◯ │ ◯ │ ◯ │ ◯ │   │
├───┼───┼───┼───┼───╆━━━┿━━━┿━━━┿━━━┿━━━╅───┤
│ 6 │   │   │   │ ◯ ┃ ✖ │ ✖ │ ✖ │ ✖ │ ✖ ┃ ◯ │
├───┼───┼───┼───┼───╄━━━┿━━━┿━━━┿━━━┿━━━╃───┤
│ 7 │   │   │   │   │ ◯ │ ◯ │ ◯ │ ◯ │ ◯ │   │
├───┼───┼───┼───┼───╆━━━┿━━━┿━━━┿━━━╅───┼───┤
│ 8 │   │   │   │   ┃   │   │   │   ┃   │   │
├───┼───╆━━━┿━━━┿━━━╋━━━┿━━━┿━━━┿━━━╃───┼───┤
│ 9 │   ┃   │   │   ┃   │   │   │   │   │   │
└───┴───┺━━━┷━━━┷━━━┹───┴───┴───┴───┴───┴───┘

╔══════════════════════════════════════════════════════════════════════════════════════════════════╗
║ <HAL> Je tire sur la case C3 qui est la plus probable (23 bateaux possibles)                     ║
║ <HAL> C3 : Manqué                                                                                ║
╚══════════════════════════════════════════════════════════════════════════════════════════════════╝
\end{verbatim}}
\newpage

{\scriptsize
\begin{verbatim}
    ┌───┬───┬───┬───┬───┬───┬───┬───┬───┬───┐
    │ A │ B │ C │ D │ E │ F │ G │ H │ I │ J │
┌───┼───┼───┼───┼───┼───┼───┼───┼───┼───┼───┤
│ 0 │   │   │   │   │   │   │   │   │   │   │
├───┼───┼───┼───┼───┼───┼───┼───┼───┼───┼───┤
│ 1 │   │   │   │   │   │   │   │   │   │   │
├───┼───┼───┼───┼───┼───┼───┼───┼───┼───╆━━━┪
│ 2 │   │   │   │ ◯ │   │   │ ◯ │   │   ┃   ┃
├───┼───┼───┼───┼───┼───┼───┼───┼───┼───╂───┨
│ 3 │   │   │ ◯ │   │   │   │   │   │   ┃   ┃
├───┼───╆━━━┿━━━┿━━━╅───┼───┼───┼───┼───╄━━━┩
│ 4 │   ┃   │   │   ┃ ◯ │   │   │   │   │   │
├───┼───╄━━━┿━━━┿━━━╃───┼───┼───┼───┼───┼───┤
│ 5 │   │   │   │   │ ◯ │ ◯ │ ◯ │ ◯ │ ◯ │   │
├───┼───┼───┼───┼───╆━━━┿━━━┿━━━┿━━━┿━━━╅───┤
│ 6 │   │   │   │ ◯ ┃ ✖ │ ✖ │ ✖ │ ✖ │ ✖ ┃ ◯ │
├───┼───┼───┼───┼───╄━━━┿━━━┿━━━┿━━━┿━━━╃───┤
│ 7 │   │   │   │   │ ◯ │ ◯ │ ◯ │ ◯ │ ◯ │   │
├───┼───┼───┼───┼───╆━━━┿━━━┿━━━┿━━━╅───┼───┤
│ 8 │   │   │   │   ┃   │   │   │   ┃   │   │
├───┼───╆━━━┿━━━┿━━━╋━━━┿━━━┿━━━┿━━━╃───┼───┤
│ 9 │   ┃   │   │   ┃   │   │   │   │   │   │
└───┴───┺━━━┷━━━┷━━━┹───┴───┴───┴───┴───┴───┘

╔══════════════════════════════════════════════════════════════════════════════════════════════════╗
║ <HAL> Je tire sur la case G2 qui est la plus probable (21 bateaux possibles)                     ║
║ <HAL> G2 : Manqué                                                                                ║
╚══════════════════════════════════════════════════════════════════════════════════════════════════╝
\end{verbatim}}
\newpage

{\scriptsize
\begin{verbatim}
    ┌───┬───┬───┬───┬───┬───┬───┬───┬───┬───┐
    │ A │ B │ C │ D │ E │ F │ G │ H │ I │ J │
┌───┼───┼───┼───┼───┼───┼───┼───┼───┼───┼───┤
│ 0 │   │   │   │   │   │   │   │   │   │   │
├───┼───┼───┼───┼───┼───┼───┼───┼───┼───┼───┤
│ 1 │   │   │   │   │   │ ◯ │   │   │   │   │
├───┼───┼───┼───┼───┼───┼───┼───┼───┼───╆━━━┪
│ 2 │   │   │   │ ◯ │   │   │ ◯ │   │   ┃   ┃
├───┼───┼───┼───┼───┼───┼───┼───┼───┼───╂───┨
│ 3 │   │   │ ◯ │   │   │   │   │   │   ┃   ┃
├───┼───╆━━━┿━━━┿━━━╅───┼───┼───┼───┼───╄━━━┩
│ 4 │   ┃   │   │   ┃ ◯ │   │   │   │   │   │
├───┼───╄━━━┿━━━┿━━━╃───┼───┼───┼───┼───┼───┤
│ 5 │   │   │   │   │ ◯ │ ◯ │ ◯ │ ◯ │ ◯ │   │
├───┼───┼───┼───┼───╆━━━┿━━━┿━━━┿━━━┿━━━╅───┤
│ 6 │   │   │   │ ◯ ┃ ✖ │ ✖ │ ✖ │ ✖ │ ✖ ┃ ◯ │
├───┼───┼───┼───┼───╄━━━┿━━━┿━━━┿━━━┿━━━╃───┤
│ 7 │   │   │   │   │ ◯ │ ◯ │ ◯ │ ◯ │ ◯ │   │
├───┼───┼───┼───┼───╆━━━┿━━━┿━━━┿━━━╅───┼───┤
│ 8 │   │   │   │   ┃   │   │   │   ┃   │   │
├───┼───╆━━━┿━━━┿━━━╋━━━┿━━━┿━━━┿━━━╃───┼───┤
│ 9 │   ┃   │   │   ┃   │   │   │   │   │   │
└───┴───┺━━━┷━━━┷━━━┹───┴───┴───┴───┴───┴───┘

╔══════════════════════════════════════════════════════════════════════════════════════════════════╗
║ <HAL> Je tire sur la case F1 qui est la plus probable (20 bateaux possibles)                     ║
║ <HAL> F1 : Manqué                                                                                ║
╚══════════════════════════════════════════════════════════════════════════════════════════════════╝
\end{verbatim}}
\newpage

{\scriptsize
\begin{verbatim}
    ┌───┬───┬───┬───┬───┬───┬───┬───┬───┬───┐
    │ A │ B │ C │ D │ E │ F │ G │ H │ I │ J │
┌───┼───┼───┼───┼───┼───┼───┼───┼───┼───┼───┤
│ 0 │   │   │   │   │   │   │   │   │   │   │
├───┼───┼───┼───┼───┼───┼───┼───┼───┼───┼───┤
│ 1 │   │   │   │   │   │ ◯ │   │   │   │   │
├───┼───┼───┼───┼───┼───┼───┼───┼───┼───╆━━━┪
│ 2 │   │   │   │ ◯ │   │   │ ◯ │   │   ┃   ┃
├───┼───┼───┼───┼───┼───┼───┼───┼───┼───╂───┨
│ 3 │   │   │ ◯ │   │   │   │   │ ◯ │   ┃   ┃
├───┼───╆━━━┿━━━┿━━━╅───┼───┼───┼───┼───╄━━━┩
│ 4 │   ┃   │   │   ┃ ◯ │   │   │   │   │   │
├───┼───╄━━━┿━━━┿━━━╃───┼───┼───┼───┼───┼───┤
│ 5 │   │   │   │   │ ◯ │ ◯ │ ◯ │ ◯ │ ◯ │   │
├───┼───┼───┼───┼───╆━━━┿━━━┿━━━┿━━━┿━━━╅───┤
│ 6 │   │   │   │ ◯ ┃ ✖ │ ✖ │ ✖ │ ✖ │ ✖ ┃ ◯ │
├───┼───┼───┼───┼───╄━━━┿━━━┿━━━┿━━━┿━━━╃───┤
│ 7 │   │   │   │   │ ◯ │ ◯ │ ◯ │ ◯ │ ◯ │   │
├───┼───┼───┼───┼───╆━━━┿━━━┿━━━┿━━━╅───┼───┤
│ 8 │   │   │   │   ┃   │   │   │   ┃   │   │
├───┼───╆━━━┿━━━┿━━━╋━━━┿━━━┿━━━┿━━━╃───┼───┤
│ 9 │   ┃   │   │   ┃   │   │   │   │   │   │
└───┴───┺━━━┷━━━┷━━━┹───┴───┴───┴───┴───┴───┘

╔══════════════════════════════════════════════════════════════════════════════════════════════════╗
║ <HAL> Je tire sur la case H3 qui est la plus probable (19 bateaux possibles)                     ║
║ <HAL> H3 : Manqué                                                                                ║
╚══════════════════════════════════════════════════════════════════════════════════════════════════╝
\end{verbatim}}
\newpage

{\scriptsize
\begin{verbatim}
    ┌───┬───┬───┬───┬───┬───┬───┬───┬───┬───┐
    │ A │ B │ C │ D │ E │ F │ G │ H │ I │ J │
┌───┼───┼───┼───┼───┼───┼───┼───┼───┼───┼───┤
│ 0 │   │   │   │   │   │   │   │   │   │   │
├───┼───┼───┼───┼───┼───┼───┼───┼───┼───┼───┤
│ 1 │   │   │   │   │   │ ◯ │   │   │   │   │
├───┼───┼───┼───┼───┼───┼───┼───┼───┼───╆━━━┪
│ 2 │   │   │   │ ◯ │   │   │ ◯ │   │   ┃   ┃
├───┼───┼───┼───┼───┼───┼───┼───┼───┼───╂───┨
│ 3 │   │   │ ◯ │   │   │   │   │ ◯ │   ┃   ┃
├───┼───╆━━━┿━━━┿━━━╅───┼───┼───┼───┼───╄━━━┩
│ 4 │   ┃   │   │   ┃ ◯ │   │   │   │   │   │
├───┼───╄━━━┿━━━┿━━━╃───┼───┼───┼───┼───┼───┤
│ 5 │   │   │   │   │ ◯ │ ◯ │ ◯ │ ◯ │ ◯ │   │
├───┼───┼───┼───┼───╆━━━┿━━━┿━━━┿━━━┿━━━╅───┤
│ 6 │   │   │   │ ◯ ┃ ✖ │ ✖ │ ✖ │ ✖ │ ✖ ┃ ◯ │
├───┼───┼───┼───┼───╄━━━┿━━━┿━━━┿━━━┿━━━╃───┤
│ 7 │   │   │   │   │ ◯ │ ◯ │ ◯ │ ◯ │ ◯ │   │
├───┼───┼───┼───┼───╆━━━┿━━━┿━━━┿━━━╅───┼───┤
│ 8 │   │   │ ◯ │   ┃   │   │   │   ┃   │   │
├───┼───╆━━━┿━━━┿━━━╋━━━┿━━━┿━━━┿━━━╃───┼───┤
│ 9 │   ┃   │   │   ┃   │   │   │   │   │   │
└───┴───┺━━━┷━━━┷━━━┹───┴───┴───┴───┴───┴───┘

╔══════════════════════════════════════════════════════════════════════════════════════════════════╗
║ <HAL> Je tire sur la case C8 qui est la plus probable (19 bateaux possibles)                     ║
║ <HAL> C8 : Manqué                                                                                ║
╚══════════════════════════════════════════════════════════════════════════════════════════════════╝
\end{verbatim}}
\newpage

{\scriptsize
\begin{verbatim}
    ┌───┬───┬───┬───┬───┬───┬───┬───┬───┬───┐
    │ A │ B │ C │ D │ E │ F │ G │ H │ I │ J │
┌───┼───┼───┼───┼───┼───┼───┼───┼───┼───┼───┤
│ 0 │   │   │   │   │   │   │   │   │   │   │
├───┼───┼───┼───┼───┼───┼───┼───┼───┼───┼───┤
│ 1 │   │   │   │   │   │ ◯ │   │   │   │   │
├───┼───┼───┼───┼───┼───┼───┼───┼───┼───╆━━━┪
│ 2 │   │   │   │ ◯ │   │   │ ◯ │   │   ┃   ┃
├───┼───┼───┼───┼───┼───┼───┼───┼───┼───╂───┨
│ 3 │   │   │ ◯ │   │   │   │   │ ◯ │   ┃   ┃
├───┼───╆━━━┿━━━┿━━━╅───┼───┼───┼───┼───╄━━━┩
│ 4 │   ┃ ✖ │   │   ┃ ◯ │   │   │   │   │   │
├───┼───╄━━━┿━━━┿━━━╃───┼───┼───┼───┼───┼───┤
│ 5 │   │   │   │   │ ◯ │ ◯ │ ◯ │ ◯ │ ◯ │   │
├───┼───┼───┼───┼───╆━━━┿━━━┿━━━┿━━━┿━━━╅───┤
│ 6 │   │   │   │ ◯ ┃ ✖ │ ✖ │ ✖ │ ✖ │ ✖ ┃ ◯ │
├───┼───┼───┼───┼───╄━━━┿━━━┿━━━┿━━━┿━━━╃───┤
│ 7 │   │   │   │   │ ◯ │ ◯ │ ◯ │ ◯ │ ◯ │   │
├───┼───┼───┼───┼───╆━━━┿━━━┿━━━┿━━━╅───┼───┤
│ 8 │   │   │ ◯ │   ┃   │   │   │   ┃   │   │
├───┼───╆━━━┿━━━┿━━━╋━━━┿━━━┿━━━┿━━━╃───┼───┤
│ 9 │   ┃   │   │   ┃   │   │   │   │   │   │
└───┴───┺━━━┷━━━┷━━━┹───┴───┴───┴───┴───┴───┘

╔══════════════════════════════════════════════════════════════════════════════════════════════════╗
║ <HAL> Je tire sur la case B4 qui est la plus probable (19 bateaux possibles)                     ║
║ <HAL> B4 : Touché                                                                                ║
║ <HAL> J'ajoute la case C4 à la file d'attente                                                    ║
║ <HAL> J'ajoute la case A4 à la file d'attente                                                    ║
║ <HAL> J'ajoute la case B3 à la file d'attente                                                    ║
║ <HAL> J'ajoute la case B5 à la file d'attente                                                    ║
║ <HAL> J'ordonne ma file d'attente en fonction des possibilités :                                 ║
║ <HAL> B5 : 8 bateaux possibles                                                                   ║
║ <HAL> B3 : 8 bateaux possibles                                                                   ║
║ <HAL> C4 : 6 bateaux possibles                                                                   ║
║ <HAL> A4 : 4 bateaux possibles                                                                   ║
║ <HAL> File d'attente : B5 B3 C4 A4                                                               ║
╚══════════════════════════════════════════════════════════════════════════════════════════════════╝
\end{verbatim}}
\newpage

{\scriptsize
\begin{verbatim}
    ┌───┬───┬───┬───┬───┬───┬───┬───┬───┬───┐
    │ A │ B │ C │ D │ E │ F │ G │ H │ I │ J │
┌───┼───┼───┼───┼───┼───┼───┼───┼───┼───┼───┤
│ 0 │   │   │   │   │   │   │   │   │   │   │
├───┼───┼───┼───┼───┼───┼───┼───┼───┼───┼───┤
│ 1 │   │   │   │   │   │ ◯ │   │   │   │   │
├───┼───┼───┼───┼───┼───┼───┼───┼───┼───╆━━━┪
│ 2 │   │   │   │ ◯ │   │   │ ◯ │   │   ┃   ┃
├───┼───┼───┼───┼───┼───┼───┼───┼───┼───╂───┨
│ 3 │   │   │ ◯ │   │   │   │   │ ◯ │   ┃   ┃
├───┼───╆━━━┿━━━┿━━━╅───┼───┼───┼───┼───╄━━━┩
│ 4 │   ┃ ✖ │   │   ┃ ◯ │   │   │   │   │   │
├───┼───╄━━━┿━━━┿━━━╃───┼───┼───┼───┼───┼───┤
│ 5 │   │ ◯ │   │   │ ◯ │ ◯ │ ◯ │ ◯ │ ◯ │   │
├───┼───┼───┼───┼───╆━━━┿━━━┿━━━┿━━━┿━━━╅───┤
│ 6 │   │   │   │ ◯ ┃ ✖ │ ✖ │ ✖ │ ✖ │ ✖ ┃ ◯ │
├───┼───┼───┼───┼───╄━━━┿━━━┿━━━┿━━━┿━━━╃───┤
│ 7 │   │   │   │   │ ◯ │ ◯ │ ◯ │ ◯ │ ◯ │   │
├───┼───┼───┼───┼───╆━━━┿━━━┿━━━┿━━━╅───┼───┤
│ 8 │   │   │ ◯ │   ┃   │   │   │   ┃   │   │
├───┼───╆━━━┿━━━┿━━━╋━━━┿━━━┿━━━┿━━━╃───┼───┤
│ 9 │   ┃   │   │   ┃   │   │   │   │   │   │
└───┴───┺━━━┷━━━┷━━━┹───┴───┴───┴───┴───┴───┘

╔══════════════════════════════════════════════════════════════════════════════════════════════════╗
║ <HAL> Je tire sur la case B5 de la file d'attente                                                ║
║ <HAL> B5 : Manqué                                                                                ║
╚══════════════════════════════════════════════════════════════════════════════════════════════════╝
\end{verbatim}}
\newpage

{\scriptsize
\begin{verbatim}
    ┌───┬───┬───┬───┬───┬───┬───┬───┬───┬───┐
    │ A │ B │ C │ D │ E │ F │ G │ H │ I │ J │
┌───┼───┼───┼───┼───┼───┼───┼───┼───┼───┼───┤
│ 0 │   │   │   │   │   │   │   │   │   │   │
├───┼───┼───┼───┼───┼───┼───┼───┼───┼───┼───┤
│ 1 │   │   │   │   │   │ ◯ │   │   │   │   │
├───┼───┼───┼───┼───┼───┼───┼───┼───┼───╆━━━┪
│ 2 │   │   │   │ ◯ │   │   │ ◯ │   │   ┃   ┃
├───┼───┼───┼───┼───┼───┼───┼───┼───┼───╂───┨
│ 3 │   │ ◯ │ ◯ │   │   │   │   │ ◯ │   ┃   ┃
├───┼───╆━━━┿━━━┿━━━╅───┼───┼───┼───┼───╄━━━┩
│ 4 │   ┃ ✖ │   │   ┃ ◯ │   │   │   │   │   │
├───┼───╄━━━┿━━━┿━━━╃───┼───┼───┼───┼───┼───┤
│ 5 │   │ ◯ │   │   │ ◯ │ ◯ │ ◯ │ ◯ │ ◯ │   │
├───┼───┼───┼───┼───╆━━━┿━━━┿━━━┿━━━┿━━━╅───┤
│ 6 │   │   │   │ ◯ ┃ ✖ │ ✖ │ ✖ │ ✖ │ ✖ ┃ ◯ │
├───┼───┼───┼───┼───╄━━━┿━━━┿━━━┿━━━┿━━━╃───┤
│ 7 │   │   │   │   │ ◯ │ ◯ │ ◯ │ ◯ │ ◯ │   │
├───┼───┼───┼───┼───╆━━━┿━━━┿━━━┿━━━╅───┼───┤
│ 8 │   │   │ ◯ │   ┃   │   │   │   ┃   │   │
├───┼───╆━━━┿━━━┿━━━╋━━━┿━━━┿━━━┿━━━╃───┼───┤
│ 9 │   ┃   │   │   ┃   │   │   │   │   │   │
└───┴───┺━━━┷━━━┷━━━┹───┴───┴───┴───┴───┴───┘

╔══════════════════════════════════════════════════════════════════════════════════════════════════╗
║ <HAL> Je tire sur la case B3 de la file d'attente                                                ║
║ <HAL> B3 : Manqué                                                                                ║
╚══════════════════════════════════════════════════════════════════════════════════════════════════╝
\end{verbatim}}
\newpage

{\scriptsize
\begin{verbatim}
    ┌───┬───┬───┬───┬───┬───┬───┬───┬───┬───┐
    │ A │ B │ C │ D │ E │ F │ G │ H │ I │ J │
┌───┼───┼───┼───┼───┼───┼───┼───┼───┼───┼───┤
│ 0 │   │   │   │   │   │   │   │   │   │   │
├───┼───┼───┼───┼───┼───┼───┼───┼───┼───┼───┤
│ 1 │   │   │   │   │   │ ◯ │   │   │   │   │
├───┼───┼───┼───┼───┼───┼───┼───┼───┼───╆━━━┪
│ 2 │   │   │   │ ◯ │   │   │ ◯ │   │   ┃   ┃
├───┼───┼───┼───┼───┼───┼───┼───┼───┼───╂───┨
│ 3 │   │ ◯ │ ◯ │   │   │   │   │ ◯ │   ┃   ┃
├───┼───╆━━━┿━━━┿━━━╅───┼───┼───┼───┼───╄━━━┩
│ 4 │   ┃ ✖ │ ✖ │   ┃ ◯ │   │   │   │   │   │
├───┼───╄━━━┿━━━┿━━━╃───┼───┼───┼───┼───┼───┤
│ 5 │   │ ◯ │   │   │ ◯ │ ◯ │ ◯ │ ◯ │ ◯ │   │
├───┼───┼───┼───┼───╆━━━┿━━━┿━━━┿━━━┿━━━╅───┤
│ 6 │   │   │   │ ◯ ┃ ✖ │ ✖ │ ✖ │ ✖ │ ✖ ┃ ◯ │
├───┼───┼───┼───┼───╄━━━┿━━━┿━━━┿━━━┿━━━╃───┤
│ 7 │   │   │   │   │ ◯ │ ◯ │ ◯ │ ◯ │ ◯ │   │
├───┼───┼───┼───┼───╆━━━┿━━━┿━━━┿━━━╅───┼───┤
│ 8 │   │   │ ◯ │   ┃   │   │   │   ┃   │   │
├───┼───╆━━━┿━━━┿━━━╋━━━┿━━━┿━━━┿━━━╃───┼───┤
│ 9 │   ┃   │   │   ┃   │   │   │   │   │   │
└───┴───┺━━━┷━━━┷━━━┹───┴───┴───┴───┴───┴───┘

╔══════════════════════════════════════════════════════════════════════════════════════════════════╗
║ <HAL> Je tire sur la case C4 de la file d'attente                                                ║
║ <HAL> C4 : Touché                                                                                ║
║ <HAL> Le bateau touché est horizontal                                                            ║
║ <HAL> J'ajoute la case D4 à la file d'attente                                                    ║
║ <HAL> File d'attente : A4 D4                                                                     ║
╚══════════════════════════════════════════════════════════════════════════════════════════════════╝
\end{verbatim}}
\newpage

{\scriptsize
\begin{verbatim}
    ┌───┬───┬───┬───┬───┬───┬───┬───┬───┬───┐
    │ A │ B │ C │ D │ E │ F │ G │ H │ I │ J │
┌───┼───┼───┼───┼───┼───┼───┼───┼───┼───┼───┤
│ 0 │   │   │   │   │   │   │   │   │   │   │
├───┼───┼───┼───┼───┼───┼───┼───┼───┼───┼───┤
│ 1 │   │   │   │   │   │ ◯ │   │   │   │   │
├───┼───┼───┼───┼───┼───┼───┼───┼───┼───╆━━━┪
│ 2 │   │   │   │ ◯ │   │   │ ◯ │   │   ┃   ┃
├───┼───┼───┼───┼───┼───┼───┼───┼───┼───╂───┨
│ 3 │   │ ◯ │ ◯ │   │   │   │   │ ◯ │   ┃   ┃
├───┼───╆━━━┿━━━┿━━━╅───┼───┼───┼───┼───╄━━━┩
│ 4 │ ◯ ┃ ✖ │ ✖ │   ┃ ◯ │   │   │   │   │   │
├───┼───╄━━━┿━━━┿━━━╃───┼───┼───┼───┼───┼───┤
│ 5 │   │ ◯ │   │   │ ◯ │ ◯ │ ◯ │ ◯ │ ◯ │   │
├───┼───┼───┼───┼───╆━━━┿━━━┿━━━┿━━━┿━━━╅───┤
│ 6 │   │   │   │ ◯ ┃ ✖ │ ✖ │ ✖ │ ✖ │ ✖ ┃ ◯ │
├───┼───┼───┼───┼───╄━━━┿━━━┿━━━┿━━━┿━━━╃───┤
│ 7 │   │   │   │   │ ◯ │ ◯ │ ◯ │ ◯ │ ◯ │   │
├───┼───┼───┼───┼───╆━━━┿━━━┿━━━┿━━━╅───┼───┤
│ 8 │   │   │ ◯ │   ┃   │   │   │   ┃   │   │
├───┼───╆━━━┿━━━┿━━━╋━━━┿━━━┿━━━┿━━━╃───┼───┤
│ 9 │   ┃   │   │   ┃   │   │   │   │   │   │
└───┴───┺━━━┷━━━┷━━━┹───┴───┴───┴───┴───┴───┘

╔══════════════════════════════════════════════════════════════════════════════════════════════════╗
║ <HAL> Je tire sur la case A4 de la file d'attente                                                ║
║ <HAL> A4 : Manqué                                                                                ║
╚══════════════════════════════════════════════════════════════════════════════════════════════════╝
\end{verbatim}}
\newpage

{\scriptsize
\begin{verbatim}
    ┌───┬───┬───┬───┬───┬───┬───┬───┬───┬───┐
    │ A │ B │ C │ D │ E │ F │ G │ H │ I │ J │
┌───┼───┼───┼───┼───┼───┼───┼───┼───┼───┼───┤
│ 0 │   │   │   │   │   │   │   │   │   │   │
├───┼───┼───┼───┼───┼───┼───┼───┼───┼───┼───┤
│ 1 │   │   │   │   │   │ ◯ │   │   │   │   │
├───┼───┼───┼───┼───┼───┼───┼───┼───┼───╆━━━┪
│ 2 │   │   │   │ ◯ │   │   │ ◯ │   │   ┃   ┃
├───┼───┼───┼───┼───┼───┼───┼───┼───┼───╂───┨
│ 3 │   │ ◯ │ ◯ │   │   │   │   │ ◯ │   ┃   ┃
├───┼───╆━━━┿━━━┿━━━╅───┼───┼───┼───┼───╄━━━┩
│ 4 │ ◯ ┃ ✖ │ ✖ │ ✖ ┃ ◯ │   │   │   │   │   │
├───┼───╄━━━┿━━━┿━━━╃───┼───┼───┼───┼───┼───┤
│ 5 │   │ ◯ │   │   │ ◯ │ ◯ │ ◯ │ ◯ │ ◯ │   │
├───┼───┼───┼───┼───╆━━━┿━━━┿━━━┿━━━┿━━━╅───┤
│ 6 │   │   │   │ ◯ ┃ ✖ │ ✖ │ ✖ │ ✖ │ ✖ ┃ ◯ │
├───┼───┼───┼───┼───╄━━━┿━━━┿━━━┿━━━┿━━━╃───┤
│ 7 │   │   │   │   │ ◯ │ ◯ │ ◯ │ ◯ │ ◯ │   │
├───┼───┼───┼───┼───╆━━━┿━━━┿━━━┿━━━╅───┼───┤
│ 8 │   │   │ ◯ │   ┃   │   │   │   ┃   │   │
├───┼───╆━━━┿━━━┿━━━╋━━━┿━━━┿━━━┿━━━╃───┼───┤
│ 9 │   ┃   │   │   ┃   │   │   │   │   │   │
└───┴───┺━━━┷━━━┷━━━┹───┴───┴───┴───┴───┴───┘

╔══════════════════════════════════════════════════════════════════════════════════════════════════╗
║ <HAL> Je tire sur la case D4 de la file d'attente                                                ║
║ <HAL> D4 : Touché                                                                                ║
╚══════════════════════════════════════════════════════════════════════════════════════════════════╝
\end{verbatim}}
\newpage

{\scriptsize
\begin{verbatim}
    ┌───┬───┬───┬───┬───┬───┬───┬───┬───┬───┐
    │ A │ B │ C │ D │ E │ F │ G │ H │ I │ J │
┌───┼───┼───┼───┼───┼───┼───┼───┼───┼───┼───┤
│ 0 │   │   │   │   │ ◯ │   │   │   │   │   │
├───┼───┼───┼───┼───┼───┼───┼───┼───┼───┼───┤
│ 1 │   │   │   │   │   │ ◯ │   │   │   │   │
├───┼───┼───┼───┼───┼───┼───┼───┼───┼───╆━━━┪
│ 2 │   │   │   │ ◯ │   │   │ ◯ │   │   ┃   ┃
├───┼───┼───┼───┼───┼───┼───┼───┼───┼───╂───┨
│ 3 │   │ ◯ │ ◯ │ ◯ │   │   │   │ ◯ │   ┃   ┃
├───┼───╆━━━┿━━━┿━━━╅───┼───┼───┼───┼───╄━━━┩
│ 4 │ ◯ ┃ ✖ │ ✖ │ ✖ ┃ ◯ │   │   │   │   │   │
├───┼───╄━━━┿━━━┿━━━╃───┼───┼───┼───┼───┼───┤
│ 5 │   │ ◯ │ ◯ │ ◯ │ ◯ │ ◯ │ ◯ │ ◯ │ ◯ │   │
├───┼───┼───┼───┼───╆━━━┿━━━┿━━━┿━━━┿━━━╅───┤
│ 6 │   │   │   │ ◯ ┃ ✖ │ ✖ │ ✖ │ ✖ │ ✖ ┃ ◯ │
├───┼───┼───┼───┼───╄━━━┿━━━┿━━━┿━━━┿━━━╃───┤
│ 7 │   │   │   │   │ ◯ │ ◯ │ ◯ │ ◯ │ ◯ │   │
├───┼───┼───┼───┼───╆━━━┿━━━┿━━━┿━━━╅───┼───┤
│ 8 │   │   │ ◯ │   ┃   │   │   │   ┃   │   │
├───┼───╆━━━┿━━━┿━━━╋━━━┿━━━┿━━━┿━━━╃───┼───┤
│ 9 │   ┃   │   │   ┃   │   │   │   │   │   │
└───┴───┺━━━┷━━━┷━━━┹───┴───┴───┴───┴───┴───┘

╔══════════════════════════════════════════════════════════════════════════════════════════════════╗
║ <HAL> Bateau de taille 3 coulé ! Je l'enlève de la liste des bateaux à chercher                  ║
║ <HAL> Bateaux restant à couler : 4 3 2                                                           ║
║ <HAL> J'élimine la case adjacente C5                                                             ║
║ <HAL> J'élimine la case adjacente D3                                                             ║
║ <HAL> J'élimine la case adjacente D5                                                             ║
║ <HAL> Je tire sur la case E0 qui est la plus probable (12 bateaux possibles)                     ║
║ <HAL> E0 : Manqué                                                                                ║
╚══════════════════════════════════════════════════════════════════════════════════════════════════╝
\end{verbatim}}
\newpage

{\scriptsize
\begin{verbatim}
    ┌───┬───┬───┬───┬───┬───┬───┬───┬───┬───┐
    │ A │ B │ C │ D │ E │ F │ G │ H │ I │ J │
┌───┼───┼───┼───┼───┼───┼───┼───┼───┼───┼───┤
│ 0 │   │   │   │   │ ◯ │   │   │   │   │   │
├───┼───┼───┼───┼───┼───┼───┼───┼───┼───┼───┤
│ 1 │   │   │   │   │   │ ◯ │   │   │ ◯ │   │
├───┼───┼───┼───┼───┼───┼───┼───┼───┼───╆━━━┪
│ 2 │   │   │   │ ◯ │   │   │ ◯ │   │   ┃   ┃
├───┼───┼───┼───┼───┼───┼───┼───┼───┼───╂───┨
│ 3 │   │ ◯ │ ◯ │ ◯ │   │   │   │ ◯ │   ┃   ┃
├───┼───╆━━━┿━━━┿━━━╅───┼───┼───┼───┼───╄━━━┩
│ 4 │ ◯ ┃ ✖ │ ✖ │ ✖ ┃ ◯ │   │   │   │   │   │
├───┼───╄━━━┿━━━┿━━━╃───┼───┼───┼───┼───┼───┤
│ 5 │   │ ◯ │ ◯ │ ◯ │ ◯ │ ◯ │ ◯ │ ◯ │ ◯ │   │
├───┼───┼───┼───┼───╆━━━┿━━━┿━━━┿━━━┿━━━╅───┤
│ 6 │   │   │   │ ◯ ┃ ✖ │ ✖ │ ✖ │ ✖ │ ✖ ┃ ◯ │
├───┼───┼───┼───┼───╄━━━┿━━━┿━━━┿━━━┿━━━╃───┤
│ 7 │   │   │   │   │ ◯ │ ◯ │ ◯ │ ◯ │ ◯ │   │
├───┼───┼───┼───┼───╆━━━┿━━━┿━━━┿━━━╅───┼───┤
│ 8 │   │   │ ◯ │   ┃   │   │   │   ┃   │   │
├───┼───╆━━━┿━━━┿━━━╋━━━┿━━━┿━━━┿━━━╃───┼───┤
│ 9 │   ┃   │   │   ┃   │   │   │   │   │   │
└───┴───┺━━━┷━━━┷━━━┹───┴───┴───┴───┴───┴───┘

╔══════════════════════════════════════════════════════════════════════════════════════════════════╗
║ <HAL> Je tire sur la case I1 qui est la plus probable (11 bateaux possibles)                     ║
║ <HAL> I1 : Manqué                                                                                ║
╚══════════════════════════════════════════════════════════════════════════════════════════════════╝
\end{verbatim}}
\newpage

{\scriptsize
\begin{verbatim}
    ┌───┬───┬───┬───┬───┬───┬───┬───┬───┬───┐
    │ A │ B │ C │ D │ E │ F │ G │ H │ I │ J │
┌───┼───┼───┼───┼───┼───┼───┼───┼───┼───┼───┤
│ 0 │   │   │   │   │ ◯ │   │   │   │   │   │
├───┼───┼───┼───┼───┼───┼───┼───┼───┼───┼───┤
│ 1 │   │   │   │   │   │ ◯ │   │   │ ◯ │   │
├───┼───┼───┼───┼───┼───┼───┼───┼───┼───╆━━━┪
│ 2 │   │   │   │ ◯ │   │   │ ◯ │   │   ┃   ┃
├───┼───┼───┼───┼───┼───┼───┼───┼───┼───╂───┨
│ 3 │   │ ◯ │ ◯ │ ◯ │   │   │   │ ◯ │   ┃   ┃
├───┼───╆━━━┿━━━┿━━━╅───┼───┼───┼───┼───╄━━━┩
│ 4 │ ◯ ┃ ✖ │ ✖ │ ✖ ┃ ◯ │   │   │   │   │   │
├───┼───╄━━━┿━━━┿━━━╃───┼───┼───┼───┼───┼───┤
│ 5 │   │ ◯ │ ◯ │ ◯ │ ◯ │ ◯ │ ◯ │ ◯ │ ◯ │   │
├───┼───┼───┼───┼───╆━━━┿━━━┿━━━┿━━━┿━━━╅───┤
│ 6 │   │   │   │ ◯ ┃ ✖ │ ✖ │ ✖ │ ✖ │ ✖ ┃ ◯ │
├───┼───┼───┼───┼───╄━━━┿━━━┿━━━┿━━━┿━━━╃───┤
│ 7 │   │   │   │   │ ◯ │ ◯ │ ◯ │ ◯ │ ◯ │   │
├───┼───┼───┼───┼───╆━━━┿━━━┿━━━┿━━━╅───┼───┤
│ 8 │   │   │ ◯ │   ┃   │   │   │   ┃   │   │
├───┼───╆━━━┿━━━┿━━━╋━━━┿━━━┿━━━┿━━━╃───┼───┤
│ 9 │   ┃   │   │ ✖ ┃   │   │   │   │   │   │
└───┴───┺━━━┷━━━┷━━━┹───┴───┴───┴───┴───┴───┘

╔══════════════════════════════════════════════════════════════════════════════════════════════════╗
║ <HAL> Je tire sur la case D9 qui est la plus probable (11 bateaux possibles)                     ║
║ <HAL> D9 : Touché                                                                                ║
║ <HAL> J'ajoute la case E9 à la file d'attente                                                    ║
║ <HAL> J'ajoute la case C9 à la file d'attente                                                    ║
║ <HAL> J'ajoute la case D8 à la file d'attente                                                    ║
║ <HAL> J'ordonne ma file d'attente en fonction des possibilités :                                 ║
║ <HAL> E9 : 6 bateaux possibles                                                                   ║
║ <HAL> C9 : 6 bateaux possibles                                                                   ║
║ <HAL> D8 : 2 bateaux possibles                                                                   ║
║ <HAL> File d'attente : E9 C9 D8                                                                  ║
╚══════════════════════════════════════════════════════════════════════════════════════════════════╝
\end{verbatim}}
\newpage

{\scriptsize
\begin{verbatim}
    ┌───┬───┬───┬───┬───┬───┬───┬───┬───┬───┐
    │ A │ B │ C │ D │ E │ F │ G │ H │ I │ J │
┌───┼───┼───┼───┼───┼───┼───┼───┼───┼───┼───┤
│ 0 │   │   │   │   │ ◯ │   │   │   │   │   │
├───┼───┼───┼───┼───┼───┼───┼───┼───┼───┼───┤
│ 1 │   │   │   │   │   │ ◯ │   │   │ ◯ │   │
├───┼───┼───┼───┼───┼───┼───┼───┼───┼───╆━━━┪
│ 2 │   │   │   │ ◯ │   │   │ ◯ │   │   ┃   ┃
├───┼───┼───┼───┼───┼───┼───┼───┼───┼───╂───┨
│ 3 │   │ ◯ │ ◯ │ ◯ │   │   │   │ ◯ │   ┃   ┃
├───┼───╆━━━┿━━━┿━━━╅───┼───┼───┼───┼───╄━━━┩
│ 4 │ ◯ ┃ ✖ │ ✖ │ ✖ ┃ ◯ │   │   │   │   │   │
├───┼───╄━━━┿━━━┿━━━╃───┼───┼───┼───┼───┼───┤
│ 5 │   │ ◯ │ ◯ │ ◯ │ ◯ │ ◯ │ ◯ │ ◯ │ ◯ │   │
├───┼───┼───┼───┼───╆━━━┿━━━┿━━━┿━━━┿━━━╅───┤
│ 6 │   │   │   │ ◯ ┃ ✖ │ ✖ │ ✖ │ ✖ │ ✖ ┃ ◯ │
├───┼───┼───┼───┼───╄━━━┿━━━┿━━━┿━━━┿━━━╃───┤
│ 7 │   │   │   │   │ ◯ │ ◯ │ ◯ │ ◯ │ ◯ │   │
├───┼───┼───┼───┼───╆━━━┿━━━┿━━━┿━━━╅───┼───┤
│ 8 │   │   │ ◯ │   ┃   │   │   │   ┃   │   │
├───┼───╆━━━┿━━━┿━━━╋━━━┿━━━┿━━━┿━━━╃───┼───┤
│ 9 │   ┃   │   │ ✖ ┃ ◯ │   │   │   │   │   │
└───┴───┺━━━┷━━━┷━━━┹───┴───┴───┴───┴───┴───┘

╔══════════════════════════════════════════════════════════════════════════════════════════════════╗
║ <HAL> Je tire sur la case E9 de la file d'attente                                                ║
║ <HAL> E9 : Manqué                                                                                ║
╚══════════════════════════════════════════════════════════════════════════════════════════════════╝
\end{verbatim}}
\newpage

{\scriptsize
\begin{verbatim}
    ┌───┬───┬───┬───┬───┬───┬───┬───┬───┬───┐
    │ A │ B │ C │ D │ E │ F │ G │ H │ I │ J │
┌───┼───┼───┼───┼───┼───┼───┼───┼───┼───┼───┤
│ 0 │   │   │   │   │ ◯ │   │   │   │   │   │
├───┼───┼───┼───┼───┼───┼───┼───┼───┼───┼───┤
│ 1 │   │   │   │   │   │ ◯ │   │   │ ◯ │   │
├───┼───┼───┼───┼───┼───┼───┼───┼───┼───╆━━━┪
│ 2 │   │   │   │ ◯ │   │   │ ◯ │   │   ┃   ┃
├───┼───┼───┼───┼───┼───┼───┼───┼───┼───╂───┨
│ 3 │   │ ◯ │ ◯ │ ◯ │   │   │   │ ◯ │   ┃   ┃
├───┼───╆━━━┿━━━┿━━━╅───┼───┼───┼───┼───╄━━━┩
│ 4 │ ◯ ┃ ✖ │ ✖ │ ✖ ┃ ◯ │   │   │   │   │   │
├───┼───╄━━━┿━━━┿━━━╃───┼───┼───┼───┼───┼───┤
│ 5 │   │ ◯ │ ◯ │ ◯ │ ◯ │ ◯ │ ◯ │ ◯ │ ◯ │   │
├───┼───┼───┼───┼───╆━━━┿━━━┿━━━┿━━━┿━━━╅───┤
│ 6 │   │   │   │ ◯ ┃ ✖ │ ✖ │ ✖ │ ✖ │ ✖ ┃ ◯ │
├───┼───┼───┼───┼───╄━━━┿━━━┿━━━┿━━━┿━━━╃───┤
│ 7 │   │   │   │   │ ◯ │ ◯ │ ◯ │ ◯ │ ◯ │   │
├───┼───┼───┼───┼───╆━━━┿━━━┿━━━┿━━━╅───┼───┤
│ 8 │   │   │ ◯ │   ┃   │   │   │   ┃   │   │
├───┼───╆━━━┿━━━┿━━━╋━━━┿━━━┿━━━┿━━━╃───┼───┤
│ 9 │   ┃   │ ✖ │ ✖ ┃ ◯ │   │   │   │   │   │
└───┴───┺━━━┷━━━┷━━━┹───┴───┴───┴───┴───┴───┘

╔══════════════════════════════════════════════════════════════════════════════════════════════════╗
║ <HAL> Je tire sur la case C9 de la file d'attente                                                ║
║ <HAL> C9 : Touché                                                                                ║
║ <HAL> Le bateau touché est horizontal                                                            ║
║ <HAL> J'enlève la case D8 de la file d'attente                                                   ║
║ <HAL> J'ajoute la case B9 à la file d'attente                                                    ║
║ <HAL> File d'attente : B9                                                                        ║
╚══════════════════════════════════════════════════════════════════════════════════════════════════╝
\end{verbatim}}
\newpage

{\scriptsize
\begin{verbatim}
    ┌───┬───┬───┬───┬───┬───┬───┬───┬───┬───┐
    │ A │ B │ C │ D │ E │ F │ G │ H │ I │ J │
┌───┼───┼───┼───┼───┼───┼───┼───┼───┼───┼───┤
│ 0 │   │   │   │   │ ◯ │   │   │   │   │   │
├───┼───┼───┼───┼───┼───┼───┼───┼───┼───┼───┤
│ 1 │   │   │   │   │   │ ◯ │   │   │ ◯ │   │
├───┼───┼───┼───┼───┼───┼───┼───┼───┼───╆━━━┪
│ 2 │   │   │   │ ◯ │   │   │ ◯ │   │   ┃   ┃
├───┼───┼───┼───┼───┼───┼───┼───┼───┼───╂───┨
│ 3 │   │ ◯ │ ◯ │ ◯ │   │   │   │ ◯ │   ┃   ┃
├───┼───╆━━━┿━━━┿━━━╅───┼───┼───┼───┼───╄━━━┩
│ 4 │ ◯ ┃ ✖ │ ✖ │ ✖ ┃ ◯ │   │   │   │   │   │
├───┼───╄━━━┿━━━┿━━━╃───┼───┼───┼───┼───┼───┤
│ 5 │   │ ◯ │ ◯ │ ◯ │ ◯ │ ◯ │ ◯ │ ◯ │ ◯ │   │
├───┼───┼───┼───┼───╆━━━┿━━━┿━━━┿━━━┿━━━╅───┤
│ 6 │   │   │   │ ◯ ┃ ✖ │ ✖ │ ✖ │ ✖ │ ✖ ┃ ◯ │
├───┼───┼───┼───┼───╄━━━┿━━━┿━━━┿━━━┿━━━╃───┤
│ 7 │   │   │   │   │ ◯ │ ◯ │ ◯ │ ◯ │ ◯ │   │
├───┼───┼───┼───┼───╆━━━┿━━━┿━━━┿━━━╅───┼───┤
│ 8 │   │   │ ◯ │   ┃   │   │   │   ┃   │   │
├───┼───╆━━━┿━━━┿━━━╋━━━┿━━━┿━━━┿━━━╃───┼───┤
│ 9 │   ┃ ✖ │ ✖ │ ✖ ┃ ◯ │   │   │   │   │   │
└───┴───┺━━━┷━━━┷━━━┹───┴───┴───┴───┴───┴───┘

╔══════════════════════════════════════════════════════════════════════════════════════════════════╗
║ <HAL> Je tire sur la case B9 de la file d'attente                                                ║
║ <HAL> B9 : Touché                                                                                ║
║ <HAL> J'ajoute la case A9 à la file d'attente                                                    ║
║ <HAL> File d'attente : A9                                                                        ║
╚══════════════════════════════════════════════════════════════════════════════════════════════════╝
\end{verbatim}}
\newpage

{\scriptsize
\begin{verbatim}
    ┌───┬───┬───┬───┬───┬───┬───┬───┬───┬───┐
    │ A │ B │ C │ D │ E │ F │ G │ H │ I │ J │
┌───┼───┼───┼───┼───┼───┼───┼───┼───┼───┼───┤
│ 0 │   │   │   │   │ ◯ │   │   │   │   │   │
├───┼───┼───┼───┼───┼───┼───┼───┼───┼───┼───┤
│ 1 │   │   │   │   │   │ ◯ │   │   │ ◯ │   │
├───┼───┼───┼───┼───┼───┼───┼───┼───┼───╆━━━┪
│ 2 │   │   │   │ ◯ │   │   │ ◯ │   │   ┃   ┃
├───┼───┼───┼───┼───┼───┼───┼───┼───┼───╂───┨
│ 3 │   │ ◯ │ ◯ │ ◯ │   │   │   │ ◯ │   ┃   ┃
├───┼───╆━━━┿━━━┿━━━╅───┼───┼───┼───┼───╄━━━┩
│ 4 │ ◯ ┃ ✖ │ ✖ │ ✖ ┃ ◯ │   │   │   │   │   │
├───┼───╄━━━┿━━━┿━━━╃───┼───┼───┼───┼───┼───┤
│ 5 │   │ ◯ │ ◯ │ ◯ │ ◯ │ ◯ │ ◯ │ ◯ │ ◯ │   │
├───┼───┼───┼───┼───╆━━━┿━━━┿━━━┿━━━┿━━━╅───┤
│ 6 │   │   │   │ ◯ ┃ ✖ │ ✖ │ ✖ │ ✖ │ ✖ ┃ ◯ │
├───┼───┼───┼───┼───╄━━━┿━━━┿━━━┿━━━┿━━━╃───┤
│ 7 │   │   │   │   │ ◯ │ ◯ │ ◯ │ ◯ │ ◯ │   │
├───┼───┼───┼───┼───╆━━━┿━━━┿━━━┿━━━╅───┼───┤
│ 8 │   │   │ ◯ │   ┃   │   │   │   ┃   │   │
├───┼───╆━━━┿━━━┿━━━╋━━━┿━━━┿━━━┿━━━╃───┼───┤
│ 9 │ ◯ ┃ ✖ │ ✖ │ ✖ ┃ ◯ │   │   │   │   │   │
└───┴───┺━━━┷━━━┷━━━┹───┴───┴───┴───┴───┴───┘

╔══════════════════════════════════════════════════════════════════════════════════════════════════╗
║ <HAL> Je tire sur la case A9 de la file d'attente                                                ║
║ <HAL> A9 : Manqué                                                                                ║
╚══════════════════════════════════════════════════════════════════════════════════════════════════╝
\end{verbatim}}
\newpage

{\scriptsize
\begin{verbatim}
    ┌───┬───┬───┬───┬───┬───┬───┬───┬───┬───┐
    │ A │ B │ C │ D │ E │ F │ G │ H │ I │ J │
┌───┼───┼───┼───┼───┼───┼───┼───┼───┼───┼───┤
│ 0 │   │   │   │   │ ◯ │   │   │   │   │   │
├───┼───┼───┼───┼───┼───┼───┼───┼───┼───┼───┤
│ 1 │   │ ◯ │   │   │   │ ◯ │   │   │ ◯ │   │
├───┼───┼───┼───┼───┼───┼───┼───┼───┼───╆━━━┪
│ 2 │   │   │   │ ◯ │   │   │ ◯ │   │   ┃   ┃
├───┼───┼───┼───┼───┼───┼───┼───┼───┼───╂───┨
│ 3 │   │ ◯ │ ◯ │ ◯ │   │   │   │ ◯ │   ┃   ┃
├───┼───╆━━━┿━━━┿━━━╅───┼───┼───┼───┼───╄━━━┩
│ 4 │ ◯ ┃ ✖ │ ✖ │ ✖ ┃ ◯ │   │   │   │   │   │
├───┼───╄━━━┿━━━┿━━━╃───┼───┼───┼───┼───┼───┤
│ 5 │   │ ◯ │ ◯ │ ◯ │ ◯ │ ◯ │ ◯ │ ◯ │ ◯ │   │
├───┼───┼───┼───┼───╆━━━┿━━━┿━━━┿━━━┿━━━╅───┤
│ 6 │   │   │   │ ◯ ┃ ✖ │ ✖ │ ✖ │ ✖ │ ✖ ┃ ◯ │
├───┼───┼───┼───┼───╄━━━┿━━━┿━━━┿━━━┿━━━╃───┤
│ 7 │   │   │   │   │ ◯ │ ◯ │ ◯ │ ◯ │ ◯ │   │
├───┼───┼───┼───┼───╆━━━┿━━━┿━━━┿━━━╅───┼───┤
│ 8 │   │ ◯ │ ◯ │ ◯ ┃   │   │   │   ┃   │   │
├───┼───╆━━━┿━━━┿━━━╋━━━┿━━━┿━━━┿━━━╃───┼───┤
│ 9 │ ◯ ┃ ✖ │ ✖ │ ✖ ┃ ◯ │   │   │   │   │   │
└───┴───┺━━━┷━━━┷━━━┹───┴───┴───┴───┴───┴───┘

╔══════════════════════════════════════════════════════════════════════════════════════════════════╗
║ <HAL> Bateau de taille 3 coulé ! Je l'enlève de la liste des bateaux à chercher                  ║
║ <HAL> Bateaux restant à couler : 4 2                                                             ║
║ <HAL> J'élimine la case adjacente D8                                                             ║
║ <HAL> J'élimine la case adjacente B8                                                             ║
║ <HAL> Je tire sur la case B1 qui est la plus probable (6 bateaux possibles)                      ║
║ <HAL> B1 : Manqué                                                                                ║
╚══════════════════════════════════════════════════════════════════════════════════════════════════╝
\end{verbatim}}
\newpage

{\scriptsize
\begin{verbatim}
    ┌───┬───┬───┬───┬───┬───┬───┬───┬───┬───┐
    │ A │ B │ C │ D │ E │ F │ G │ H │ I │ J │
┌───┼───┼───┼───┼───┼───┼───┼───┼───┼───┼───┤
│ 0 │   │   │   │   │ ◯ │   │   │   │   │   │
├───┼───┼───┼───┼───┼───┼───┼───┼───┼───┼───┤
│ 1 │   │ ◯ │   │   │   │ ◯ │   │   │ ◯ │   │
├───┼───┼───┼───┼───┼───┼───┼───┼───┼───╆━━━┪
│ 2 │   │   │   │ ◯ │   │   │ ◯ │   │   ┃   ┃
├───┼───┼───┼───┼───┼───┼───┼───┼───┼───╂───┨
│ 3 │   │ ◯ │ ◯ │ ◯ │   │   │   │ ◯ │   ┃ ✖ ┃
├───┼───╆━━━┿━━━┿━━━╅───┼───┼───┼───┼───╄━━━┩
│ 4 │ ◯ ┃ ✖ │ ✖ │ ✖ ┃ ◯ │   │   │   │   │   │
├───┼───╄━━━┿━━━┿━━━╃───┼───┼───┼───┼───┼───┤
│ 5 │   │ ◯ │ ◯ │ ◯ │ ◯ │ ◯ │ ◯ │ ◯ │ ◯ │   │
├───┼───┼───┼───┼───╆━━━┿━━━┿━━━┿━━━┿━━━╅───┤
│ 6 │   │   │   │ ◯ ┃ ✖ │ ✖ │ ✖ │ ✖ │ ✖ ┃ ◯ │
├───┼───┼───┼───┼───╄━━━┿━━━┿━━━┿━━━┿━━━╃───┤
│ 7 │   │   │   │   │ ◯ │ ◯ │ ◯ │ ◯ │ ◯ │   │
├───┼───┼───┼───┼───╆━━━┿━━━┿━━━┿━━━╅───┼───┤
│ 8 │   │ ◯ │ ◯ │ ◯ ┃   │   │   │   ┃   │   │
├───┼───╆━━━┿━━━┿━━━╋━━━┿━━━┿━━━┿━━━╃───┼───┤
│ 9 │ ◯ ┃ ✖ │ ✖ │ ✖ ┃ ◯ │   │   │   │   │   │
└───┴───┺━━━┷━━━┷━━━┹───┴───┴───┴───┴───┴───┘

╔══════════════════════════════════════════════════════════════════════════════════════════════════╗
║ <HAL> Je tire sur la case J3 qui est la plus probable (6 bateaux possibles)                      ║
║ <HAL> J3 : Touché                                                                                ║
║ <HAL> J'ajoute la case I3 à la file d'attente                                                    ║
║ <HAL> J'ajoute la case J2 à la file d'attente                                                    ║
║ <HAL> J'ajoute la case J4 à la file d'attente                                                    ║
║ <HAL> J'ordonne ma file d'attente en fonction des possibilités :                                 ║
║ <HAL> J2 : 4 bateaux possibles                                                                   ║
║ <HAL> J4 : 3 bateaux possibles                                                                   ║
║ <HAL> I3 : 1 bateaux possibles                                                                   ║
║ <HAL> File d'attente : J2 J4 I3                                                                  ║
╚══════════════════════════════════════════════════════════════════════════════════════════════════╝
\end{verbatim}}
\newpage

{\scriptsize
\begin{verbatim}
    ┌───┬───┬───┬───┬───┬───┬───┬───┬───┬───┐
    │ A │ B │ C │ D │ E │ F │ G │ H │ I │ J │
┌───┼───┼───┼───┼───┼───┼───┼───┼───┼───┼───┤
│ 0 │   │   │   │   │ ◯ │   │   │   │   │   │
├───┼───┼───┼───┼───┼───┼───┼───┼───┼───┼───┤
│ 1 │   │ ◯ │   │   │   │ ◯ │   │   │ ◯ │   │
├───┼───┼───┼───┼───┼───┼───┼───┼───┼───╆━━━┪
│ 2 │   │   │   │ ◯ │   │   │ ◯ │   │   ┃ ✖ ┃
├───┼───┼───┼───┼───┼───┼───┼───┼───┼───╂───┨
│ 3 │   │ ◯ │ ◯ │ ◯ │   │   │   │ ◯ │   ┃ ✖ ┃
├───┼───╆━━━┿━━━┿━━━╅───┼───┼───┼───┼───╄━━━┩
│ 4 │ ◯ ┃ ✖ │ ✖ │ ✖ ┃ ◯ │   │   │   │   │   │
├───┼───╄━━━┿━━━┿━━━╃───┼───┼───┼───┼───┼───┤
│ 5 │   │ ◯ │ ◯ │ ◯ │ ◯ │ ◯ │ ◯ │ ◯ │ ◯ │   │
├───┼───┼───┼───┼───╆━━━┿━━━┿━━━┿━━━┿━━━╅───┤
│ 6 │   │   │   │ ◯ ┃ ✖ │ ✖ │ ✖ │ ✖ │ ✖ ┃ ◯ │
├───┼───┼───┼───┼───╄━━━┿━━━┿━━━┿━━━┿━━━╃───┤
│ 7 │   │   │   │   │ ◯ │ ◯ │ ◯ │ ◯ │ ◯ │   │
├───┼───┼───┼───┼───╆━━━┿━━━┿━━━┿━━━╅───┼───┤
│ 8 │   │ ◯ │ ◯ │ ◯ ┃   │   │   │   ┃   │   │
├───┼───╆━━━┿━━━┿━━━╋━━━┿━━━┿━━━┿━━━╃───┼───┤
│ 9 │ ◯ ┃ ✖ │ ✖ │ ✖ ┃ ◯ │   │   │   │   │   │
└───┴───┺━━━┷━━━┷━━━┹───┴───┴───┴───┴───┴───┘

╔══════════════════════════════════════════════════════════════════════════════════════════════════╗
║ <HAL> Je tire sur la case J2 de la file d'attente                                                ║
║ <HAL> J2 : Touché                                                                                ║
║ <HAL> Le bateau touché est vertical                                                              ║
║ <HAL> J'enlève la case I3 de la file d'attente                                                   ║
║ <HAL> J'ajoute la case J1 à la file d'attente                                                    ║
║ <HAL> File d'attente : J4 J1                                                                     ║
╚══════════════════════════════════════════════════════════════════════════════════════════════════╝
\end{verbatim}}
\newpage

{\scriptsize
\begin{verbatim}
    ┌───┬───┬───┬───┬───┬───┬───┬───┬───┬───┐
    │ A │ B │ C │ D │ E │ F │ G │ H │ I │ J │
┌───┼───┼───┼───┼───┼───┼───┼───┼───┼───┼───┤
│ 0 │   │   │   │   │ ◯ │   │   │   │   │   │
├───┼───┼───┼───┼───┼───┼───┼───┼───┼───┼───┤
│ 1 │   │ ◯ │   │   │   │ ◯ │   │   │ ◯ │   │
├───┼───┼───┼───┼───┼───┼───┼───┼───┼───╆━━━┪
│ 2 │   │   │   │ ◯ │   │   │ ◯ │   │   ┃ ✖ ┃
├───┼───┼───┼───┼───┼───┼───┼───┼───┼───╂───┨
│ 3 │   │ ◯ │ ◯ │ ◯ │   │   │   │ ◯ │   ┃ ✖ ┃
├───┼───╆━━━┿━━━┿━━━╅───┼───┼───┼───┼───╄━━━┩
│ 4 │ ◯ ┃ ✖ │ ✖ │ ✖ ┃ ◯ │   │   │   │   │ ◯ │
├───┼───╄━━━┿━━━┿━━━╃───┼───┼───┼───┼───┼───┤
│ 5 │   │ ◯ │ ◯ │ ◯ │ ◯ │ ◯ │ ◯ │ ◯ │ ◯ │   │
├───┼───┼───┼───┼───╆━━━┿━━━┿━━━┿━━━┿━━━╅───┤
│ 6 │   │   │   │ ◯ ┃ ✖ │ ✖ │ ✖ │ ✖ │ ✖ ┃ ◯ │
├───┼───┼───┼───┼───╄━━━┿━━━┿━━━┿━━━┿━━━╃───┤
│ 7 │   │   │   │   │ ◯ │ ◯ │ ◯ │ ◯ │ ◯ │   │
├───┼───┼───┼───┼───╆━━━┿━━━┿━━━┿━━━╅───┼───┤
│ 8 │   │ ◯ │ ◯ │ ◯ ┃   │   │   │   ┃   │   │
├───┼───╆━━━┿━━━┿━━━╋━━━┿━━━┿━━━┿━━━╃───┼───┤
│ 9 │ ◯ ┃ ✖ │ ✖ │ ✖ ┃ ◯ │   │   │   │   │   │
└───┴───┺━━━┷━━━┷━━━┹───┴───┴───┴───┴───┴───┘

╔══════════════════════════════════════════════════════════════════════════════════════════════════╗
║ <HAL> Je tire sur la case J4 de la file d'attente                                                ║
║ <HAL> J4 : Manqué                                                                                ║
╚══════════════════════════════════════════════════════════════════════════════════════════════════╝
\end{verbatim}}
\newpage

{\scriptsize
\begin{verbatim}
    ┌───┬───┬───┬───┬───┬───┬───┬───┬───┬───┐
    │ A │ B │ C │ D │ E │ F │ G │ H │ I │ J │
┌───┼───┼───┼───┼───┼───┼───┼───┼───┼───┼───┤
│ 0 │   │   │   │   │ ◯ │   │   │   │   │   │
├───┼───┼───┼───┼───┼───┼───┼───┼───┼───┼───┤
│ 1 │   │ ◯ │   │   │   │ ◯ │   │   │ ◯ │ ◯ │
├───┼───┼───┼───┼───┼───┼───┼───┼───┼───╆━━━┪
│ 2 │   │   │   │ ◯ │   │   │ ◯ │   │   ┃ ✖ ┃
├───┼───┼───┼───┼───┼───┼───┼───┼───┼───╂───┨
│ 3 │   │ ◯ │ ◯ │ ◯ │   │   │   │ ◯ │   ┃ ✖ ┃
├───┼───╆━━━┿━━━┿━━━╅───┼───┼───┼───┼───╄━━━┩
│ 4 │ ◯ ┃ ✖ │ ✖ │ ✖ ┃ ◯ │   │   │   │   │ ◯ │
├───┼───╄━━━┿━━━┿━━━╃───┼───┼───┼───┼───┼───┤
│ 5 │   │ ◯ │ ◯ │ ◯ │ ◯ │ ◯ │ ◯ │ ◯ │ ◯ │   │
├───┼───┼───┼───┼───╆━━━┿━━━┿━━━┿━━━┿━━━╅───┤
│ 6 │   │   │   │ ◯ ┃ ✖ │ ✖ │ ✖ │ ✖ │ ✖ ┃ ◯ │
├───┼───┼───┼───┼───╄━━━┿━━━┿━━━┿━━━┿━━━╃───┤
│ 7 │   │   │   │   │ ◯ │ ◯ │ ◯ │ ◯ │ ◯ │   │
├───┼───┼───┼───┼───╆━━━┿━━━┿━━━┿━━━╅───┼───┤
│ 8 │   │ ◯ │ ◯ │ ◯ ┃   │   │   │   ┃   │   │
├───┼───╆━━━┿━━━┿━━━╋━━━┿━━━┿━━━┿━━━╃───┼───┤
│ 9 │ ◯ ┃ ✖ │ ✖ │ ✖ ┃ ◯ │   │   │   │   │   │
└───┴───┺━━━┷━━━┷━━━┹───┴───┴───┴───┴───┴───┘

╔══════════════════════════════════════════════════════════════════════════════════════════════════╗
║ <HAL> Je tire sur la case J1 de la file d'attente                                                ║
║ <HAL> J1 : Manqué                                                                                ║
╚══════════════════════════════════════════════════════════════════════════════════════════════════╝
\end{verbatim}}
\newpage

{\scriptsize
\begin{verbatim}
    ┌───┬───┬───┬───┬───┬───┬───┬───┬───┬───┐
    │ A │ B │ C │ D │ E │ F │ G │ H │ I │ J │
┌───┼───┼───┼───┼───┼───┼───┼───┼───┼───┼───┤
│ 0 │   │   │   │   │ ◯ │   │   │   │   │   │
├───┼───┼───┼───┼───┼───┼───┼───┼───┼───┼───┤
│ 1 │   │ ◯ │ ◯ │ ◯ │ ◯ │ ◯ │ ◯ │ ◯ │ ◯ │ ◯ │
├───┼───┼───┼───┼───┼───┼───┼───┼───┼───╆━━━┪
│ 2 │   │ ◯ │ ◯ │ ◯ │ ◯ │ ◯ │ ◯ │ ◯ │ ◯ ┃ ✖ ┃
├───┼───┼───┼───┼───┼───┼───┼───┼───┼───╂───┨
│ 3 │   │ ◯ │ ◯ │ ◯ │ ◯ │ ◯ │ ◯ │ ◯ │ ◯ ┃ ✖ ┃
├───┼───╆━━━┿━━━┿━━━╅───┼───┼───┼───┼───╄━━━┩
│ 4 │ ◯ ┃ ✖ │ ✖ │ ✖ ┃ ◯ │   │   │   │   │ ◯ │
├───┼───╄━━━┿━━━┿━━━╃───┼───┼───┼───┼───┼───┤
│ 5 │   │ ◯ │ ◯ │ ◯ │ ◯ │ ◯ │ ◯ │ ◯ │ ◯ │ ◯ │
├───┼───┼───┼───┼───╆━━━┿━━━┿━━━┿━━━┿━━━╅───┤
│ 6 │   │ ◯ │ ◯ │ ◯ ┃ ✖ │ ✖ │ ✖ │ ✖ │ ✖ ┃ ◯ │
├───┼───┼───┼───┼───╄━━━┿━━━┿━━━┿━━━┿━━━╃───┤
│ 7 │   │   │   │   │ ◯ │ ◯ │ ◯ │ ◯ │ ◯ │ ◯ │
├───┼───┼───┼───┼───╆━━━┿━━━┿━━━┿━━━╅───┼───┤
│ 8 │   │ ◯ │ ◯ │ ◯ ┃   │   │ ✖ │   ┃   │   │
├───┼───╆━━━┿━━━┿━━━╋━━━┿━━━┿━━━┿━━━╃───┼───┤
│ 9 │ ◯ ┃ ✖ │ ✖ │ ✖ ┃ ◯ │   │   │   │   │   │
└───┴───┺━━━┷━━━┷━━━┹───┴───┴───┴───┴───┴───┘

╔══════════════════════════════════════════════════════════════════════════════════════════════════╗
║ <HAL> Bateau de taille 2 coulé ! Je l'enlève de la liste des bateaux à chercher                  ║
║ <HAL> Bateaux restant à couler : 4                                                               ║
║ <HAL> J'élimine la case adjacente I3                                                             ║
║ <HAL> J'élimine la case adjacente I2                                                             ║
║ <HAL> J'élimine la cases B2 : zone trop petite pour le plus petit bateau de taille 4             ║
║ <HAL> J'élimine la cases B6 : zone trop petite pour le plus petit bateau de taille 4             ║
║ <HAL> J'élimine la cases C1 : zone trop petite pour le plus petit bateau de taille 4             ║
║ <HAL> J'élimine la cases C2 : zone trop petite pour le plus petit bateau de taille 4             ║
║ <HAL> J'élimine la cases C6 : zone trop petite pour le plus petit bateau de taille 4             ║
║ <HAL> J'élimine la cases D1 : zone trop petite pour le plus petit bateau de taille 4             ║
║ <HAL> J'élimine la cases E1 : zone trop petite pour le plus petit bateau de taille 4             ║
║ <HAL> J'élimine la cases E2 : zone trop petite pour le plus petit bateau de taille 4             ║
║ <HAL> J'élimine la cases E3 : zone trop petite pour le plus petit bateau de taille 4             ║
║ <HAL> J'élimine la cases F2 : zone trop petite pour le plus petit bateau de taille 4             ║
║ <HAL> J'élimine la cases F3 : zone trop petite pour le plus petit bateau de taille 4             ║
║ <HAL> J'élimine la cases G1 : zone trop petite pour le plus petit bateau de taille 4             ║
║ <HAL> J'élimine la cases G3 : zone trop petite pour le plus petit bateau de taille 4             ║
║ <HAL> J'élimine la cases H1 : zone trop petite pour le plus petit bateau de taille 4             ║
║ <HAL> J'élimine la cases H2 : zone trop petite pour le plus petit bateau de taille 4             ║
║ <HAL> J'élimine la cases J5 : zone trop petite pour le plus petit bateau de taille 4             ║
║ <HAL> J'élimine la cases J7 : zone trop petite pour le plus petit bateau de taille 4             ║
║ <HAL> Je tire sur la case G8 qui est la plus probable (3 bateaux possibles)                      ║
║ <HAL> G8 : Touché                                                                                ║
║ <HAL> J'ajoute la case H8 à la file d'attente                                                    ║
║ <HAL> J'ajoute la case F8 à la file d'attente                                                    ║
║ <HAL> Le plus petit bateau, de taille 4, ne rentre pas verticalement en case G8                  ║
║ <HAL> J'ordonne ma file d'attente en fonction des possibilités :                                 ║
║ <HAL> H8 : 3 bateaux possibles                                                                   ║
║ <HAL> F8 : 2 bateaux possibles                                                                   ║
║ <HAL> File d'attente : H8 F8                                                                     ║
╚══════════════════════════════════════════════════════════════════════════════════════════════════╝
\end{verbatim}}
\newpage

{\scriptsize
\begin{verbatim}
    ┌───┬───┬───┬───┬───┬───┬───┬───┬───┬───┐
    │ A │ B │ C │ D │ E │ F │ G │ H │ I │ J │
┌───┼───┼───┼───┼───┼───┼───┼───┼───┼───┼───┤
│ 0 │   │   │   │   │ ◯ │   │   │   │   │   │
├───┼───┼───┼───┼───┼───┼───┼───┼───┼───┼───┤
│ 1 │   │ ◯ │ ◯ │ ◯ │ ◯ │ ◯ │ ◯ │ ◯ │ ◯ │ ◯ │
├───┼───┼───┼───┼───┼───┼───┼───┼───┼───╆━━━┪
│ 2 │   │ ◯ │ ◯ │ ◯ │ ◯ │ ◯ │ ◯ │ ◯ │ ◯ ┃ ✖ ┃
├───┼───┼───┼───┼───┼───┼───┼───┼───┼───╂───┨
│ 3 │   │ ◯ │ ◯ │ ◯ │ ◯ │ ◯ │ ◯ │ ◯ │ ◯ ┃ ✖ ┃
├───┼───╆━━━┿━━━┿━━━╅───┼───┼───┼───┼───╄━━━┩
│ 4 │ ◯ ┃ ✖ │ ✖ │ ✖ ┃ ◯ │   │   │   │   │ ◯ │
├───┼───╄━━━┿━━━┿━━━╃───┼───┼───┼───┼───┼───┤
│ 5 │   │ ◯ │ ◯ │ ◯ │ ◯ │ ◯ │ ◯ │ ◯ │ ◯ │ ◯ │
├───┼───┼───┼───┼───╆━━━┿━━━┿━━━┿━━━┿━━━╅───┤
│ 6 │   │ ◯ │ ◯ │ ◯ ┃ ✖ │ ✖ │ ✖ │ ✖ │ ✖ ┃ ◯ │
├───┼───┼───┼───┼───╄━━━┿━━━┿━━━┿━━━┿━━━╃───┤
│ 7 │   │   │   │   │ ◯ │ ◯ │ ◯ │ ◯ │ ◯ │ ◯ │
├───┼───┼───┼───┼───╆━━━┿━━━┿━━━┿━━━╅───┼───┤
│ 8 │   │ ◯ │ ◯ │ ◯ ┃   │   │ ✖ │ ✖ ┃   │   │
├───┼───╆━━━┿━━━┿━━━╋━━━┿━━━┿━━━┿━━━╃───┼───┤
│ 9 │ ◯ ┃ ✖ │ ✖ │ ✖ ┃ ◯ │   │   │   │   │   │
└───┴───┺━━━┷━━━┷━━━┹───┴───┴───┴───┴───┴───┘

╔══════════════════════════════════════════════════════════════════════════════════════════════════╗
║ <HAL> Je tire sur la case H8 de la file d'attente                                                ║
║ <HAL> H8 : Touché                                                                                ║
║ <HAL> Le bateau touché est horizontal                                                            ║
║ <HAL> J'ajoute la case I8 à la file d'attente                                                    ║
║ <HAL> File d'attente : F8 I8                                                                     ║
╚══════════════════════════════════════════════════════════════════════════════════════════════════╝
\end{verbatim}}
\newpage

{\scriptsize
\begin{verbatim}
    ┌───┬───┬───┬───┬───┬───┬───┬───┬───┬───┐
    │ A │ B │ C │ D │ E │ F │ G │ H │ I │ J │
┌───┼───┼───┼───┼───┼───┼───┼───┼───┼───┼───┤
│ 0 │   │   │   │   │ ◯ │   │   │   │   │   │
├───┼───┼───┼───┼───┼───┼───┼───┼───┼───┼───┤
│ 1 │   │ ◯ │ ◯ │ ◯ │ ◯ │ ◯ │ ◯ │ ◯ │ ◯ │ ◯ │
├───┼───┼───┼───┼───┼───┼───┼───┼───┼───╆━━━┪
│ 2 │   │ ◯ │ ◯ │ ◯ │ ◯ │ ◯ │ ◯ │ ◯ │ ◯ ┃ ✖ ┃
├───┼───┼───┼───┼───┼───┼───┼───┼───┼───╂───┨
│ 3 │   │ ◯ │ ◯ │ ◯ │ ◯ │ ◯ │ ◯ │ ◯ │ ◯ ┃ ✖ ┃
├───┼───╆━━━┿━━━┿━━━╅───┼───┼───┼───┼───╄━━━┩
│ 4 │ ◯ ┃ ✖ │ ✖ │ ✖ ┃ ◯ │   │   │   │   │ ◯ │
├───┼───╄━━━┿━━━┿━━━╃───┼───┼───┼───┼───┼───┤
│ 5 │   │ ◯ │ ◯ │ ◯ │ ◯ │ ◯ │ ◯ │ ◯ │ ◯ │ ◯ │
├───┼───┼───┼───┼───╆━━━┿━━━┿━━━┿━━━┿━━━╅───┤
│ 6 │   │ ◯ │ ◯ │ ◯ ┃ ✖ │ ✖ │ ✖ │ ✖ │ ✖ ┃ ◯ │
├───┼───┼───┼───┼───╄━━━┿━━━┿━━━┿━━━┿━━━╃───┤
│ 7 │   │   │   │   │ ◯ │ ◯ │ ◯ │ ◯ │ ◯ │ ◯ │
├───┼───┼───┼───┼───╆━━━┿━━━┿━━━┿━━━╅───┼───┤
│ 8 │   │ ◯ │ ◯ │ ◯ ┃   │ ✖ │ ✖ │ ✖ ┃   │   │
├───┼───╆━━━┿━━━┿━━━╋━━━┿━━━┿━━━┿━━━╃───┼───┤
│ 9 │ ◯ ┃ ✖ │ ✖ │ ✖ ┃ ◯ │   │   │   │   │   │
└───┴───┺━━━┷━━━┷━━━┹───┴───┴───┴───┴───┴───┘

╔══════════════════════════════════════════════════════════════════════════════════════════════════╗
║ <HAL> Je tire sur la case F8 de la file d'attente                                                ║
║ <HAL> F8 : Touché                                                                                ║
║ <HAL> J'ajoute la case E8 à la file d'attente                                                    ║
║ <HAL> File d'attente : I8 E8                                                                     ║
╚══════════════════════════════════════════════════════════════════════════════════════════════════╝
\end{verbatim}}
\newpage

{\scriptsize
\begin{verbatim}
    ┌───┬───┬───┬───┬───┬───┬───┬───┬───┬───┐
    │ A │ B │ C │ D │ E │ F │ G │ H │ I │ J │
┌───┼───┼───┼───┼───┼───┼───┼───┼───┼───┼───┤
│ 0 │   │   │   │   │ ◯ │   │   │   │   │   │
├───┼───┼───┼───┼───┼───┼───┼───┼───┼───┼───┤
│ 1 │   │ ◯ │ ◯ │ ◯ │ ◯ │ ◯ │ ◯ │ ◯ │ ◯ │ ◯ │
├───┼───┼───┼───┼───┼───┼───┼───┼───┼───╆━━━┪
│ 2 │   │ ◯ │ ◯ │ ◯ │ ◯ │ ◯ │ ◯ │ ◯ │ ◯ ┃ ✖ ┃
├───┼───┼───┼───┼───┼───┼───┼───┼───┼───╂───┨
│ 3 │   │ ◯ │ ◯ │ ◯ │ ◯ │ ◯ │ ◯ │ ◯ │ ◯ ┃ ✖ ┃
├───┼───╆━━━┿━━━┿━━━╅───┼───┼───┼───┼───╄━━━┩
│ 4 │ ◯ ┃ ✖ │ ✖ │ ✖ ┃ ◯ │   │   │   │   │ ◯ │
├───┼───╄━━━┿━━━┿━━━╃───┼───┼───┼───┼───┼───┤
│ 5 │   │ ◯ │ ◯ │ ◯ │ ◯ │ ◯ │ ◯ │ ◯ │ ◯ │ ◯ │
├───┼───┼───┼───┼───╆━━━┿━━━┿━━━┿━━━┿━━━╅───┤
│ 6 │   │ ◯ │ ◯ │ ◯ ┃ ✖ │ ✖ │ ✖ │ ✖ │ ✖ ┃ ◯ │
├───┼───┼───┼───┼───╄━━━┿━━━┿━━━┿━━━┿━━━╃───┤
│ 7 │   │   │   │   │ ◯ │ ◯ │ ◯ │ ◯ │ ◯ │ ◯ │
├───┼───┼───┼───┼───╆━━━┿━━━┿━━━┿━━━╅───┼───┤
│ 8 │   │ ◯ │ ◯ │ ◯ ┃   │ ✖ │ ✖ │ ✖ ┃ ◯ │   │
├───┼───╆━━━┿━━━┿━━━╋━━━┿━━━┿━━━┿━━━╃───┼───┤
│ 9 │ ◯ ┃ ✖ │ ✖ │ ✖ ┃ ◯ │   │   │   │   │   │
└───┴───┺━━━┷━━━┷━━━┹───┴───┴───┴───┴───┴───┘

╔══════════════════════════════════════════════════════════════════════════════════════════════════╗
║ <HAL> Je tire sur la case I8 de la file d'attente                                                ║
║ <HAL> I8 : Manqué                                                                                ║
╚══════════════════════════════════════════════════════════════════════════════════════════════════╝
\end{verbatim}}
\newpage

{\scriptsize
\begin{verbatim}
    ┌───┬───┬───┬───┬───┬───┬───┬───┬───┬───┐
    │ A │ B │ C │ D │ E │ F │ G │ H │ I │ J │
┌───┼───┼───┼───┼───┼───┼───┼───┼───┼───┼───┤
│ 0 │   │   │   │   │ ◯ │   │   │   │   │   │
├───┼───┼───┼───┼───┼───┼───┼───┼───┼───┼───┤
│ 1 │   │ ◯ │ ◯ │ ◯ │ ◯ │ ◯ │ ◯ │ ◯ │ ◯ │ ◯ │
├───┼───┼───┼───┼───┼───┼───┼───┼───┼───╆━━━┪
│ 2 │   │ ◯ │ ◯ │ ◯ │ ◯ │ ◯ │ ◯ │ ◯ │ ◯ ┃ ✖ ┃
├───┼───┼───┼───┼───┼───┼───┼───┼───┼───╂───┨
│ 3 │   │ ◯ │ ◯ │ ◯ │ ◯ │ ◯ │ ◯ │ ◯ │ ◯ ┃ ✖ ┃
├───┼───╆━━━┿━━━┿━━━╅───┼───┼───┼───┼───╄━━━┩
│ 4 │ ◯ ┃ ✖ │ ✖ │ ✖ ┃ ◯ │   │   │   │   │ ◯ │
├───┼───╄━━━┿━━━┿━━━╃───┼───┼───┼───┼───┼───┤
│ 5 │   │ ◯ │ ◯ │ ◯ │ ◯ │ ◯ │ ◯ │ ◯ │ ◯ │ ◯ │
├───┼───┼───┼───┼───╆━━━┿━━━┿━━━┿━━━┿━━━╅───┤
│ 6 │   │ ◯ │ ◯ │ ◯ ┃ ✖ │ ✖ │ ✖ │ ✖ │ ✖ ┃ ◯ │
├───┼───┼───┼───┼───╄━━━┿━━━┿━━━┿━━━┿━━━╃───┤
│ 7 │   │   │   │   │ ◯ │ ◯ │ ◯ │ ◯ │ ◯ │ ◯ │
├───┼───┼───┼───┼───╆━━━┿━━━┿━━━┿━━━╅───┼───┤
│ 8 │   │ ◯ │ ◯ │ ◯ ┃ ✖ │ ✖ │ ✖ │ ✖ ┃ ◯ │   │
├───┼───╆━━━┿━━━┿━━━╋━━━┿━━━┿━━━┿━━━╃───┼───┤
│ 9 │ ◯ ┃ ✖ │ ✖ │ ✖ ┃ ◯ │   │   │   │   │   │
└───┴───┺━━━┷━━━┷━━━┹───┴───┴───┴───┴───┴───┘

╔══════════════════════════════════════════════════════════════════════════════════════════════════╗
║ <HAL> Je tire sur la case E8 de la file d'attente                                                ║
║ <HAL> E8 : Touché                                                                                ║
║ <HAL> Bateau de taille 4 coulé car c'est le plus grand restant                                   ║
║ <HAL> Je vide ma file d'attente                                                                  ║
║ <HAL> Partie terminée en 36 coups                                                                ║
╚══════════════════════════════════════════════════════════════════════════════════════════════════╝
\end{verbatim}
}


\chapter{Codes des caractères graphiques}\label{annexe_codescar}

Voici la liste des caractères graphiques utilisés dans l'affichage en console. Ces caractères font presque tous partie de la famille Unicode Box Drawing, consultable sur la page \texttt{http://www.unicode.org/charts/PDF/U2500.pdf}.

\begin{verbatim}
# Caractères simples pour la grille
# ---------------------------------
# Traits
CAR_H = u'\u2500'          # Trait Horizontal : ─
CAR_V = u'\u2502'          # Trait Vertical : │
# Coins
CAR_CHG = u'\u250C'        # Coin Haut Gauche : ┌
CAR_CHD = u'\u2510'        # Coin Haut Droite : ┐
CAR_CBG = u'\u2514'        # Coin Bas Gauche : └
CAR_CBD = u'\u2518'        # Coin Bas Droite : ┘
# T
CAR_TH = u'\u252C'         # T Haut : ┬
CAR_TB = u'\u2534'         # T Bas : ┴
CAR_TG = u'\u251C'         # T Gauche : ├
CAR_TD = u'\u2524'         # T Droite : ┤
# +
CAR_CX = u'\u253C'         # Croix Centrale : ┼

# Caractères en gras pour les bateaux
# -----------------------------------
# Traits
CAR_GH = u'\u2501'         # Trait Gras Horizontal : ━
CAR_GV = u'\u2503'         # Trait Gras Vertical : ┃
# T
CAR_GTB = u'\u2537'        # T Gras Bas : ┷
CAR_GTD = u'\u2528'        # T Gras Droite : ┨
CAR_GTDH = u'\u252A'       # T Droite Haut : ┪
CAR_GTDB = u'\u2529'       # T Droite Bas : ┩
CAR_GTBG = u'\u253A'       # T Bas Gauche : ┺
CAR_GTBD = u'\u2539'       # T Bas Droite : ┹

# Coins
CAR_GCBD = u'\u251B'       # Coin Gras Bas Gauche : ┛
# +
CAR_GCXHG = u'\u2546'      # Croix Gras Haut Gauche : ╆
CAR_GCXHD = u'\u2545'      # Croix Gras Haut Droite : ╅
CAR_GCXBG = u'\u2544'      # Croix Gras Bas Gauche : ╄
CAR_GCXBD = u'\u2543'      # Croix Gras Bas Droite : ╃
CAR_GCX = u'\u254B'        # Croix Gras Centrale : ╋
CAR_GCXH = u'\u253F'       # Croix Gras Horizontal : ┿
CAR_GCXV = u'\u2542'       # Croix Gras Vertical : ╂



# Touché / Manqué
# ---------------
CAR_TOUCH = u'\u2716'      # Touché : ✖
CAR_MANQ = u'\u25EF'       # Manqué : ◯
\end{verbatim}

%\end{appendices}

\end{document}