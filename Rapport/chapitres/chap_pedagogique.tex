\chapter{Point de vue pédagogique}

Bien évidemment, ce projet dépasse largement ce qui est exigible d'un élève (même très bon) de lycée. Certains points peuvent néanmoins être abordés en simplifiant certaines parties et en l'abordant soit comme une série de TP guidés (les élèves doivent coder le contenu des fonctions dont on leur donne le prototype), soit comme projet de fin d'année. On peut aborder les points suivants :
\begin{itemize}
\item La structure de la grille : un bon exemple de codage d'une structure complexe (définir les états des cases, utilisation d'un dictionnaire ou d'une liste double, tests des cases valides,...)
\item Le placement de bateaux : sûrement la partie la moins évidente, mais oblige à réfléchir sur la façon de définir un bateau 
\item L'affichage (simple) de la grille : utilisation de boucles imbriquées et de tests pour afficher les bons symboles, et gestion de la mises en page
\item Éventuellement une interface graphique en utilisant des boutons ou un canevas pour les cases
\item La possibilité pour un joueur de tirer sur une case et retour du résultat
\item Une résolution de la grille par l'ordinateur avec uniquement des tirs aléatoires sur les cases vides (les plus en avances pourront réfléchir à des méthodes plus évoluées)
\end{itemize}

