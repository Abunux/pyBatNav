\chapter{Interface web}

Le dossier \texttt{interface\_web} contient les fichiers pour une interface web. Celle-ci est réalisée en HTML5 et CCS3 et utilise des \texttt{canvas} dont la gestion est faite en Javascript. La communication avec les modules du projet se fait via le script cgi \texttt{bn\_cgi.py}. Un serveur minimal est également implémenté dans le fichier \texttt{bn\_webserveur.py}.

Un problème rencontré a été la persistance des données de jeu entre les lancements successifs du scripts (les différentes pages). Pour résoudre ce problème, au premier lancement d'une partie, le script crée un ID aléatoire et va sauvegarder les données de la partie dans le fichier \texttt{./sessions/session\_ID} grâce au module built-in \texttt{shelve}. Cet ID est stocké ensuite dans le code source de la page HTML en tant que variable Javascript. Il ne restera alors plus qu'à recharger ces donnés lors des coups suivants.

\section{Serveur web}
La partie serveur est gérée par le fichier \texttt{bn\_webserveur.py}.
Celle-ci crée un serveur web, grâce à l'objet \texttt{HTTPServer} du module  \texttt{http.server}, sur le port 8000 accessible uniquement en local (\texttt{localhost} ou \texttt{127.0.0.1}). La gestion des scripts \texttt{cgi} est faite par l'objet \texttt{CGIHTTPRequestHandler} du module \texttt{http.server}.

Le deuxième rôle de \texttt{bn\_webserveur.py} est d'effacer les anciens fichiers de session au cas où il en resterait.

\section{HTML5, CSS3 et Javascript}
Le fichier \texttt{index.html} présente rapidement le projet et permet de lancer les trois modes de jeu : jeu solo, résolution automatique seule et partie contre l'ordinateur. Son style CSS est dans le fichier \texttt{./css/style\_index.css}.

Lors du lancement d'une partie, le code de la page est généré par le script \texttt{bn\_cgi.py} et le style CSS de la page est dans les fichiers \texttt{./css/style\_solo.css} ou \texttt{./css/style\_duos.css}, suivant s'il faut afficher une ou deux grilles. Ces deux styles héritent de \texttt{./css/style\_base.css} qui contient leurs éléments communs.

Le dessin de la grille est effectué par un \texttt{canvas} en HTML5. Cet objet permet de dessiner grâce à des fonctions en Javascript. La récupération des coordonnées de la souris est également gérée en Javascript.

\section{Script cgi}
Un script \texttt{cgi} permet de générer du code source HTML pour l'envoyer au serveur en fonction des données qu'il reçoit. Ici, ces données sont :
\begin{itemize}
\item \texttt{id} : l'identificateur de la partie, égal à 0 au premier lancement.
\item \texttt{mode} : le mode de jeu (0 : partie solo, 1 : partie contre l'ordinateur, 2 : résolution automatique).
\item \texttt{jx} et \texttt{jy} : les coordonnées de la case cliquée par le joueur. 
\end{itemize}
Une URL sera donc de la forme :
\begin{center}
\texttt{http://localhost:8000/cgi-bin/bn\_cgi.py?id=638229159\&mode=1\&jx=2\&jy=6}
\end{center}

\medskip

Le script va alors récupérer ces paramètres et lancer une action de jeu appropriée : 
\begin{itemize}
\item Si l'\texttt{id} vaut 0, on vient de lancer le script pour la première fois. On crée donc un \texttt{id} aléatoire et on crée les acteurs du jeu (joueurs et grilles) que l'on sauvegarde dans un fichier de session.\\
Sinon, on récupère les données du jeu dans le fichier de session correspondant.\\
Cet \texttt{id} est sauvegardé dans le code HTML de la page dans une variable Javascript \texttt{sessionID}.
\item \texttt{mode} va déterminer l'action à entreprendre : faire jouer l'ordinateur ou le joueur, et également déterminer la feuille de style.
\end{itemize}

\medskip

Une fois ces paramètres récupérés, et les objets de jeu créés, le script va créer le source HTML et les fonctions Javascript d'affichage du canvas. Grâce à la variable \texttt{sessionID}, on pourra faire des liens dynamiques vers la page suivante de la partie. 

\section{Évolutions possibles}
Grâce à cette infrastructure, on peut tout à fait envisager des créer une base de données des utilisateurs connectés et organiser des parties à deux joueurs, garder des traces de leurs statistiques (nombre de partie jouées, gagnées, nombre de coups moyens,...).

Par ailleurs, cette partie étant juste expérimentale, je n'ai pas implémenté la possibilité pour un joueur de placer lui même ses bateaux, mais ce ne serait pas difficile à faire (juste un petit peu long). 