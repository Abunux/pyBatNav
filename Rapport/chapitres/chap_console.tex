\chapter{Affichage console}
Le module \texttt{bn\_console.py} implémente l'interface en mode console.

L'idée de cette interface est de rendre hommage au style de jeu des années 80 en essayant d'en garder au maximum l'esprit. 

Il est fortement conseillé de mettre la fenêtre de terminal en plein écran afin d'avoir un affichage complet des grilles. 

\section{Préliminaires}
\subsection{Constantes graphiques}
Pour afficher les grilles j'utilise des caractères graphiques en unicode (famille Box Drawing de codes U2500 à U257F). Ceux-ci donnent tous les outils afin de fabriquer des grilles, y compris avec des caractères gras. Pour des raisons de commodité, le code de chacun des caractères utilisés est stocké dans une constante (par exemple \texttt{CAR\_CX=u'\textbackslash u253C'} correspond à la croix centrale \texttt{┼}). La liste complète des codes des caractères utilisés est donnée en annexe \ref{annexe_codescar} page \pageref{annexe_codescar}.

\subsection{Effacer le terminal}
Le module \texttt{os} permet d'une part d'accéder à la version du système d'exploitation avec \texttt{os.name} et, d'autre part, de lancer des commandes système avec \texttt{os.system(commande)}. La combinaison de ces deux commandes permet facilement de pouvoir effacer l'écran en utilisant la commande \texttt{cls} sous Windows et \texttt{clear} sous Linux.

\subsection{Fusion des deux grilles}
Lors d'une partie contre un adversaire, il faut pouvoir afficher côte à côte la grille de suivi du joueur ainsi que sa propre grille avec, au fur et à mesure, les coups joués par l'adversaire. Afin de réaliser cette opération on utilise la fonction \texttt{fusion(chaine1, chaine2)}. Celle-ci prend en entrée deux chaînes de caractères et retourne la chaîne fusionnée de la façon suivante : chaque chaîne est convertie en liste en prenant comme séparateur le caractère de retour de ligne \texttt{'\textbackslash n'} grâce à la méthode \texttt{String.split('\textbackslash n')}. Ensuite, en prenant à tour de rôle les éléments de chacune des listes et en insérant un caractère de trait vertical entre les deux on crée la chaîne fusionnée. 

Le résultat peut être vu plus bas en page \pageref{deux_joueurs}.

\subsection{Autres fonctions d'affichage}
La fonction \texttt{centre(chaine, longueur)} centre la chaîne sur un espace de longueur donnée en insérant le nombre d'espaces nécessaires. Cette fonction sera utilisée pour l'affichage des noms des joueurs.

La fonction \texttt{boite(texte, prefixe, longueur)} permet d'encadrer le texte dans une boîte de longueur donnée, chaque ligne étant précédée d'un préfixe. Cette fonction sera utilisée pour afficher la liste des messages pour chaque joueur à chaque tour, les préfixe servant à identifier l'auteur du message.

Notons enfin qu'afin de pouvoir réutiliser le code de ce module dans d'autres contextes (comme une interface en \texttt{tkinter}), la fonction \texttt{print(*args)} a été encapsulée dans une fonction \texttt{info(*args)} de sorte qu'il suffit de surcharger cette dernière pour envoyer l'affichage ailleurs (par exemple dans une boîte de texte dans une fenêtre graphique)

\section{Affichage des grilles}
La classe \texttt{GrilleC(Grille)} hérite de la classe \texttt{Grille} en ajoutant uniquement les fonctions d'affichage.

\subsection{Affichage simple de la grille}
La méthode \texttt{GrilleC.make\_chaine(self)} crée la chaîne de caractères de la grille simple. En plus des coins, chaque case utilise 3 caractère horizontaux (ce qui permet de centrer un symbole) et 1 caractère vertical.

Pour cet affichage, on crée les lignes les unes après les autres (dans deux boucles impbriquées) en marquant les cases suivant les valeurs de \texttt{Grille.etat}. La seule subtilité provient des deux premières et de la dernière ligne (à cause des coins).


\subsection{Affichage de la grille avec ses propres bateaux en gras}
Cette partie est beaucoup plus délicate. L'idée est d'afficher une grille en entourant ses propres bateaux et en marquant les coups joués par l'adversaire (sa grille de suivi). C'est le rôle de la fonction \texttt{GrilleC.make\_chaine\_adverse(self, grille)}, où \texttt{grille} est soit sa propre grille de jeu, soit celle de l'adversaire si on veut tricher (pour les tests bien sûr...) ou en fin de partie, si on a perdu, pour avoir la solution. Par convention, comme \texttt{grille} est une grille de jeu, nous allons noter dans les explications les seuls états possibles par $1$ si la case est occupée par un bateau et $0$ sinon (ou si la case est hors grille).

Les contraintes que nous nous fixons sont les suivantes :
\begin{itemize}
\item Les bords des bateaux doivent être en gras
\item Les séparations à l'intérieur d'un bateau doivent être en clair
\item Lorsque deux bateaux se touchent par un coin, il faut bien sûr que ces coins soit en gras (soit une croix en gras)
\end{itemize}


Le bord de chaque case de la grille se fera sur la ligne du bas, la séparation verticale de gauche et le coin en bas à gauche. Le cas de la ligne horizontal juste sous les lettre sera fait à part, ainsi que la dernière ligne horizontale et la dernière ligne verticale de la grille (à cause du bord).

\begin{enumerate}
\item Première ligne, sous les lettres des colonnes : pour chaque case de la ligne $0$ on va tester son état, ainsi que l'état de la case de gauche (pour savoir si on est en début ou en fin de bateau, ou au milieu d'un bateau). On obtient les configurations suivantes (la case testée est celle de droite) :

\begin{verbatim}
   ┼───     ┿━━━     ╅───     ╆━━━
 0   0    1   1    1   0    0   1
\end{verbatim}


\item Lignes suivantes, jusqu'à l'avant dernière : ici c'est plus délicat car il faut tester, en plus de celle de gauche (pour la séparation verticale), la case en-dessous (pour la séparation horizontale) et celle en-dessous à gauche (pour le coin) pour obtenir les 11 configurations suivantes (la case testée est celle en haut à droite) :

\begin{verbatim}
 1 │ 1    0 ┃ 1    0 ┃ 1    0 ┃ 1
   ┿━━━     ╄━━━     ╋━━━     ╂───
 0   0    0   0    1   0    0   1
  
 
 1 ┃ 0    1 ┃ 0    1 ┃ 0    0 │ 0    0 │ 0    0 │ 0    0 │ 0
   ╃───     ╂───     ╋━━━     ┿━━━     ┼───     ╅───     ╆━━━
 0   0    1   0    0   1    1   1    0   0    1   0    0   1
\end{verbatim}

Les 5 autres configurations de 4 cases restantes sont impossibles (bateaux collés par un côté).

\item Enfin, pour la dernière colonne va juste tester la case en-dessous, et pour la dernière ligne, on va juste tester celle de gauche. Pour la case tout en bas à droite il faudra juste finir en mettant un coin.

\begin{verbatim}
 0 │      0 │      1 ┃      1 ┃  
   ┤        ┪        ┨        ┩
 0        1        1        0
 
 1   1    1 ┃ 0    0   1    0 │ 0
   ┷━━━     ┹───     ┺━━━     ┴───

 1 ┃      0 │
   ┛        ┘
\end{verbatim}
\end{enumerate}

Au final, cela permet de construire toute la grille, dans tous les cas de figure possibles. Le résultat est visible sur la grille de droite ci-dessous.

Notons que cette technique pourrait servir dans d'autres jeux, comme un Sudoku.


\newpage
\section{Modes de jeu}
Après un écran d'introduction et un menu sommaire, le programme propose 3 modes de jeu, dans lesquels on peut choisir le niveau de l'ordinateur (et si on veut tricher), ainsi que le test statistique des algorithmes :
\begin{enumerate}
\item Jeu en solo sur une grille aléatoire.
\item Résolution d'une grille aléatoire par l'ordinateur, avec mise en évidence des bateaux qu'il doit trouver (voir annexe \ref{annexe_algo_action} page \pageref{annexe_algo_action}).
\item Jeu contre l'ordinateur (suivant l'algorithme de déroulement de partie en annexe \ref{algo_liste}, page \pageref{algo_partie}).

L'affichage sera alors le suivant :

\label{deux_joueurs}
{\scriptsize
\begin{boxedverbatim}
                  ╔══════╗                     ┃                     ╔═════╗                   
                  ║ Toto ║                     ┃                     ║ HAL ║                   
                  ╚══════╝                     ┃                     ╚═════╝                   
    ┌───┬───┬───┬───┬───┬───┬───┬───┬───┬───┐  ┃      ┌───┬───┬───┬───┬───┬───┬───┬───┬───┬───┐
    │ A │ B │ C │ D │ E │ F │ G │ H │ I │ J │  ┃      │ A │ B │ C │ D │ E │ F │ G │ H │ I │ J │
┌───┼───┼───┼───┼───┼───┼───┼───┼───┼───┼───┤  ┃  ┌───┼───┼───╆━━━╅───┼───┼───┼───┼───┼───┼───┤
│ 0 │   │   │   │   │   │   │   │ ◯ │ ✖ │ ✖ │  ┃  │ 0 │   │   ┃   ┃   │   │   │   │   │   │   │
├───┼───┼───┼───┼───┼───┼───┼───┼───┼───┼───┤  ┃  ├───┼───┼───╂───╂───┼───┼───┼───┼───┼───┼───┤
│ 1 │   │   │   │   │   │   │   │   │ ◯ │ ◯ │  ┃  │ 1 │   │   ┃   ┃   │   │   │   │   │   │   │
├───┼───┼───┼───┼───┼───┼───┼───┼───┼───┼───┤  ┃  ├───┼───┼───╂───╂───┼───┼───┼───┼───╆━━━╅───┤
│ 2 │ ◯ │   │   │   │   │   │   │   │   │   │  ┃  │ 2 │   │   ┃   ┃   │   │   │ ◯ │   ┃   ┃   │
├───┼───┼───┼───┼───┼───┼───┼───┼───┼───┼───┤  ┃  ├───┼───┼───╂───╂───┼───┼───┼───┼───╂───╂───┤
│ 3 │   │   │   │   │   │   │   │   │   │   │  ┃  │ 3 │   │   ┃   ┃ ◯ │   │   │   │ ◯ ┃   ┃   │
├───┼───┼───┼───┼───┼───┼───┼───┼───┼───┼───┤  ┃  ├───┼───┼───╄━━━╃───┼───┼───┼───┼───╂───╂───┤
│ 4 │   │   │   │   │   │   │ ◯ │   │   │   │  ┃  │ 4 │   │   │   │   │ ◯ │   │   │   ┃ ✖ ┃   │
├───┼───┼───┼───┼───┼───┼───┼───┼───┼───┼───┤  ┃  ├───┼───┼───┼───┼───┼───┼───┼───┼───╄━━━╃───┤
│ 5 │   │ ◯ │   │   │   │   │   │ ◯ │   │   │  ┃  │ 5 │   │   │   │   │   │ ◯ │   │   │   │   │
├───┼───┼───┼───┼───┼───┼───┼───┼───┼───┼───┤  ┃  ├───┼───┼───┼───┼───┼───┼───┼───╆━━━┿━━━╅───┤
│ 6 │   │   │ ✖ │   │   │   │   │   │   │   │  ┃  │ 6 │   │   │ ◯ │ ◯ │ ◯ │ ◯ │ ◯ ┃   │   ┃   │
├───┼───┼───┼───┼───┼───┼───┼───┼───┼───┼───┤  ┃  ├───┼───┼───╆━━━┿━━━┿━━━╅───┼───╄━━━┿━━━╃───┤
│ 7 │   │   │   │   │   │   │   │   │   │   │  ┃  │ 7 │   │ ◯ ┃ ✖ │ ✖ │ ✖ ┃ ◯ │   │   │   │   │
├───┼───┼───┼───┼───┼───┼───┼───┼───┼───┼───┤  ┃  ├───┼───┼───╄━━━┿━━━┿━━━╃───┼───┼───┼───┼───┤
│ 8 │   │   │ ◯ │   │   │   │   │   │   │   │  ┃  │ 8 │   │   │ ◯ │ ◯ │ ◯ │   │   │   │   │   │
├───┼───┼───┼───┼───┼───┼───┼───┼───┼───┼───┤  ┃  ├───┼───┼───┼───╆━━━┿━━━┿━━━┿━━━┿━━━╅───┼───┤
│ 9 │   │   │   │   │   │ ◯ │   │   │   │ ◯ │  ┃  │ 9 │   │   │   ┃   │   │   │   │   ┃   │   │
└───┴───┴───┴───┴───┴───┴───┴───┴───┴───┴───┘  ┃  └───┴───┴───┴───┺━━━┷━━━┷━━━┷━━━┷━━━┹───┴───┘

<Toto> Coup (Entrée pour un coup aléatoire) : 

\end{boxedverbatim}
}
\normalsize

Notons que chaque fois qu'un coup est demandé au joueur, le programme teste la validité de sa réponse (chaîne qui n'est pas une case, case hors grille ou encore une case déjà jouée).

\item Tests statistiques des algorithmes de résolution avec choix des paramètres de la grille.
\end{enumerate}


