\chapter{Gestion des joueurs et de la partie}\label{chap_joueurs}

La gestion des joueurs et du déroulement de la partie se font dans le module \texttt{bn\_joueur.py} mais les classes \texttt{Joueur} et \texttt{Partie} sont très minimales et ne constituent majoritairement qu'un squelette pour la suite. Elles seront largement héritées, que ce soit par la classe \texttt{Ordi} qui implémente l'algorithme de résolution, que pour les différentes interfaces (console et graphique).

\section{La classe Joueur}
Lors de son initialisation, on peut donner un nom au joueur et on initialise sa grille de jeu \texttt{Joueur.grille\_joueur}, la grille de l'adversaire \texttt{Joueur.grille\_adverse} ainsi que sa grille de suivi des coups \texttt{Joueur.grille\_suivi}.

On en profite aussi pour initialiser quelques variables d'état comme la liste des coups déjà joués et le nombre de coups joués.

La méthode principale de cette classe est \texttt{Joueur.tire(self, case)} qui permet de tirer sur une case et d'avoir en retour le résultat du coup (y compris si le coup n'est pas valide).

\medskip

Notons l'attribut \texttt{Joueur.messages} qui est une liste contenant différents messages d'information (comme par exemple "A2 : Touché", ou encore les messages indiquant comment l'algorithme résout la grille). Lors de l'affichage des messages, il suffit de vider cette liste grâce à des \texttt{pop(0)} successifs dans la méthode \texttt{Joueur.affiche\_messages(self)} en affichant chaque élément pour avoir un suivi.

\section{La classe Partie}
Ici encore, un squelette et des méthodes très générales pour une classe qui sera héritée dans les interfaces.

Elle se contente de définir l'adversaire, de placer les bateaux du joueur et de récupérer les paramètres de l'adversaire (sa grille et le coup qu'il vient de jouer).

Notons l'instruction \texttt{isinstance(self.adversaire, Ordi)} qui permet de savoir si l'adversaire est l'ordinateur.

\medskip

À la base nous voulions faire un mode de jeu en réseau et c'est ici que se seraient trouvées les instructions de communication.

\medskip

L'algorithme décrivant le déroulement de la partie est donné en annexe \ref{algo_liste}, page \pageref{algo_partie}.
%\newpage

%Le déroulement d'une partie à deux joueurs se déroule selon l'algorithme suivant (le joueur porte le numéro $0$ et son adversaire le numéro $1$) :
%
%\begin{algo1}\label{algo_partie}
%Placement des bateaux du joueur\\
%Récupération des bateaux de l'adversaire\\
%Aléa(0,1)\sto joueur\_en\_cours (celui qui commence)\\
%0\sto nb\_coups\\
%Tant qu'aucun joueur n'a fini :\\
%\tab{1}nb\_coups+1\sto nb\_coups\\
%\tab{1}Si joueur\_en\_cours == 0 et l'adversaire n'a pas fini :\\
%\tab{2}Le joueur joue un coup\\
%\tab{2}Mise à jour de l'affichage des grilles\\
%\tab{2}Affichage des messages du joueur (résultat du coup)\\
%\tab{1}Sinon :\\
%\tab{2}Récupération du coup de l'adversaire\\
%\tab{2}Mise à jour de l'affichage des grilles\\
%\tab{2}Affichage des messages de l'adversaire\\
%\tab{1}(joueur\_en\_cours+1)\%2\sto joueur\_en\_cours (changement de joueur)\\
%Affichage des grilles avec la solution\\
%Affichage du gagnant\\
%\end{algo1}