\chapter{Quelques fonctions utiles}

\section{Coordonnées des cases}
Les cases de la grille sont codées par des tuples $(x,y)$, où $x$ est la colonne et $y$ la ligne. Aussi nous utilisons la fonction \texttt{alpha(case)} qui, à partir des coordonnées, retourne sa représentation naturelle (par exemple \texttt{'B4'}) et \texttt{coord(case\_alpha)} qui réalise l'opération réciproque.

Ces deux fonctions utilisent le code ASCII.

\section{Constantes de direction}
Les constantes suivantes, définies dans le module \texttt{bn\_utiles.py} indiquent les différentes directions, et sont utilisées dans tout le projet :
\begin{itemize}
\item \texttt{BN\_DROITE = (1, 0)}
\item \texttt{BN\_GAUCHE = (-1, 0)}
\item \texttt{BN\_BAS = (0, 1)}
\item \texttt{BN\_HAUT = (0, -1)}
\end{itemize}
Ainsi que :
\begin{itemize}
\item \texttt{BN\_ALLDIR = (1, 1)} (toutes les direction)
\item \texttt{BN\_HORIZONTAL = (1, 0)} (à gauche et à droite)
\item \texttt{BN\_VERTICAL = (0, 1)} (en haut et en bas)
\end{itemize}
Elles permettent de rendre le code plus clair et plus compact.

\section{Paramètres en ligne de commande}
Le lancement du programme principal \texttt{main.py} admet un paramètre en ligne de commande. Ce paramètre est géré par le module \texttt{argparse} qui est très pratique. La prototype est le suivant :

\begin{verbatim}
usage: main.py [-h] [--interface INTERFACE]

Jeu de bataille navale

optional arguments:
  -h, --help            show this help message and exit
  --interface INTERFACE, -i INTERFACE
                        Choix de l'interface : 'console' ou 'tkinter'
\end{verbatim}