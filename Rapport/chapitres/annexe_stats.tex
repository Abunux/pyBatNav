\chapter{Étude statistique des algorithmes de résolution}\label{annexe_stats}
Le module \texttt{bn\_stats.py} fournit, dans la classe \texttt{Stats}, les outils pour analyser statistiquement une distribution de valeurs (calculs des indicateurs statistiques classiques, représentation en histogramme et diagramme en boîte grâce aux modules \texttt{numpy} et \texttt{matplotlib}).

Cette classe fournit également les outils pour sauvegarder et charger la liste des résultats bruts dans un fichier texte pour des analyses plus poussées futures.

La méthode de test est la suivante : on crée une liste \texttt{distrib} de longueur \texttt{xmax*ymax+1} qui est initialisée avec des valeurs nulles.\\
On répète $n$ fois la résolution sur une grille aléatoire, chaque fois différente et, en notant $k$ le nombre de coups de la résolution, on augmente \texttt{distrib[k]} de 1.

Le calcul du temps se fait en lançant un chronomètre grâce à la fonction \texttt{time()} du module \texttt{time}, qui renvoie le nombre de seconde écoulées depuis epoch (le 1\up{er} janvier 1970). Donc en sauvegardant cette valeur au début de la simulation dans une variable \texttt{start} et en calculant \texttt{time()-start} à la fin de la simulation, on obtient le temps total écoulé.\\
Afin de chronométrer uniquement le temps de résolution (et non de la création de la grille), ce chronomètre est mis à jour uniquement lors de la résolution effective.

Tous les temps ont été mesuré sur un ordinateur disposant d'un processeur Intel i7-4800-MQ cadencé à 2,7 GHz et de 16 Go de mémoire vive, en mode monoprocesseur.

\vfill 
\begin{flushright}
Résultats à partir de la page suivante $\rightarrow$
\end{flushright}
\newpage
\section{Niveau 1}
Dans ce niveau tous les tirs sont aléatoires uniformément sur les cases vides, et il n'y a pas de phase de tirs ciblés.\\
On obtient les résultats suivants sur un échantillon de $n=10\,000$ parties :

\begin{center}\label{histo_algo}
\fbox{\includegraphics[scale=0.7]{./media/distrib_HAL_niveau=1_n=10000.png}}
\end{center}

Comme on pouvait s'y attendre, les résultats sont catastrophiques. Par contre la résolution est quasi immédiate (3 ms par partie en moyenne)
\newpage
\section{Niveau 2}
Au niveau 2, la phase de tirs en aveugle est aléatoire uniformément sur les cases vide, mais on ajoute la phase de tirs ciblés lorsqu'on touche un bateau.

Les résultats sur $n=10\,000$ parties sont les suivants :

 \begin{center}\label{histo_algo}
\fbox{\includegraphics[scale=0.7]{./media/distrib_HAL_niveau=2_n=10000.png}}
\end{center}

On note une nette amélioration du fait de la gestion de la phase de tirs ciblés. La forme de la courbe de distribution est d'ailleurs totalement différente de la précédente.\\
Par contre la résolutions prend quasiment 10 fois plus de temps.
\newpage
\section{Niveau 5}
Les résultats obtenues sur un échantillon de $n=1\,000\,000$ parties sont les suivants :
\begin{center}
\begin{tabular}[t]{|c|l|}
\hline
Nombre de coups & Effectifs\\
\hline
21 & 5\\
\hline
22 & 12\\
\hline
23 & 38\\
\hline
24 & 139\\
\hline
25 & 345\\
\hline
26 & 756\\
\hline
27 & 1635\\
\hline
28 & 3141\\
\hline
29 & 5188\\
\hline
30 & 8244\\
\hline
31 & 12579\\
\hline
32 & 17849\\
\hline
33 & 24091\\
\hline
34 & 30884\\
\hline
35 & 38162\\
\hline
36 & 45397\\
\hline
37 & 51988\\
\hline
38 & 57489\\
\hline
39 & 61778\\
\hline
40 & 64082\\
\hline
41 & 64215\\
\hline
42 & 62822\\
\hline
43 & 59966\\
\hline
\end{tabular}\hspace{0.5cm}
\begin{tabular}[t]{|c|l|}
\hline
Nombre de coups & Effectifs\\
\hline
44 & 56471\\
\hline
45 & 51981\\
\hline
46 & 46666\\
\hline
47 & 41422\\
\hline
48 & 35961\\
\hline
49 & 31474\\
\hline
50 & 26963\\
\hline
51 & 22700\\
\hline
52 & 19270\\
\hline
53 & 15816\\
\hline
54 & 12374\\
\hline
55 & 9586\\
\hline
56 & 6861\\
\hline
57 & 4819\\
\hline
58 & 3020\\
\hline
59 & 1844\\
\hline
60 & 1062\\
\hline
61 & 521\\
\hline
62 & 241\\
\hline
63 & 101\\
\hline
64 & 28\\
\hline
65 & 10\\
\hline
66 & 4\\
\hline
\end{tabular}
\end{center}

\begin{center}\label{histo_algo}
\fbox{\includegraphics[scale=0.7]{./media/distrib_1000000.png}}
\end{center}

Notons les excellentes performances avec une moyenne de 42,06 coups pour un temps de résolution moyen de seulement 31,2 ms\footnote{Temps mesuré sur un processeur Intel Core i7 4800-MQ à 2,7 Ghz} par partie.
