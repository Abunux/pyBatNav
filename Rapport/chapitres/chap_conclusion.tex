\chapter{Conclusion}
%\section{Conclusion de Frédéric Muller}
Ce travail a été très stimulant et m'a pris de nombreuses heures, mais j'y ai pris beaucoup de plaisir. Entre la mise en place des structures de données pour avoir un projet minimal, l'algorithme de résolution et son optimisation, la mise en place des outils statistiques, et la construction de l'interface console, je me suis beaucoup amusé.

Certes il reste des points à développer, comme la mise en place d'une architecture de jeu en réseau, mais je pense que ce projet est déjà bien abouti. 

La rédaction du rapport en \LaTeX\ a également été très plaisante, avec quelques figures réalisées avec tikz, l'affichage des caractères unicodes, ou encore la rapatriement automatique des docstrings (et des nombreux autres problèmes techniques qui m'ont permis de progresser). Cela m'a rappelé mes années d'étude, et notamment mon DESS IM dans lequel je faisais beaucoup de projets informatiques de ce type.

Enfin j'ai trouvé un grand intérêt dans l'obligation de rendre un travail propre et fini, ce qui n'est pas le cas dans un projet personnel, dans lequel on est moins exigeant. Cela faisait longtemps que je n'avais pas fait ça et, rien que pour ça, je suis heureux d'avoir suivi cette formation.
