\chapter{Conclusion}
%\section{Conclusion de Frédéric Muller}
Ce travail a été très stimulant et m'a pris de nombreuses heures (environ 200), mais j'y ai pris beaucoup de plaisir. Il a d'abord fallu mettre en place des structures de données pour avoir un projet minimal. Ensuite est venu le temps de coder l'algorithme de résolution et son optimisation, les outils statistiques, la construction de l'interface console et enfin l'interface web. Lors de ces différentes phases, de nombreux problèmes sont survenus et leurs résolutions m'ont permis de progresser.

Certes il reste des points à développer, comme la mise en place d'une architecture de jeu en réseau, ou une interface en \texttt{tkinter} qui devait être réalisée par mon binôme, mais je pense que ce projet est déjà bien abouti. 

La rédaction du rapport en \LaTeX\ a également été très plaisante, avec quelques figures réalisées avec tikz, l'affichage des caractères unicodes, ou encore la rapatriement automatique des docstrings. 
%
%Cela m'a rappelé mes années d'étude, et notamment mon DESS IM dans lequel je faisais beaucoup de projets informatiques de ce type.
%
%Enfin j'ai trouvé un grand intérêt dans l'obligation de rendre un travail propre et fini, ce qui n'est pas le cas dans un projet personnel, dans lequel on est moins exigeant. Cela faisait longtemps que je n'avais pas fait ça et, rien que pour ça, je suis heureux d'avoir suivi cette formation.
