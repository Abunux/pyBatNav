\chapter{Conclusions}
\section{Conclusion personnelle}
Ce travail a été très stimulant et m'a pris un temps conséquent (plus de 200 heures et 3500 lignes de code), mais j'y ai pris beaucoup de plaisir. Il a d'abord fallu mettre en place des structures de données pour avoir un projet minimal. Ensuite est venu le temps de coder l'algorithme de résolution et son optimisation, les outils statistiques, la construction de l'interface console, puis de l'interface web et enfin l'interface en \texttt{tkinter} réalisée dans l'urgence. Lors de ces différentes phases, de nombreux problèmes sont survenus et leurs résolutions m'ont permis de progresser. J'ai notamment découvert certains modules Python comme le très pratique \texttt{shelve} ou encore la gestion des scripts \texttt{cgi} dans un serveur web en Python. J'ai également dû apprendre à gérer un \texttt{canvas} en HTML5 et me perfectionner un peu en Javascript ainsi qu'en CSS3. Pour le reste j'ai manipulé des choses que je connaissais déjà bien comme \texttt{matpoltlib}, \texttt{numpy} ou encore \texttt{tkinter}, et la partie qui m'a pris le plus de temps a été la mise en place de l'algorithme de résolution et ses multiples tests et optimisations.

\medskip

La rédaction du rapport en \LaTeX\ a également été très plaisante, avec quelques figures réalisées avec \texttt{tikz}, l'affichage des caractères unicodes, ou encore la rapatriement automatique des docstrings. 

\section{Évolutions possibles}
Il reste de nombreux points à développer pour une évolution future du projet :
\begin{itemize}
\item Mise en place d'une architecture réseau.
\item Gestion et sauvegarde des paramètres (nom du joueur, paramètres de la grille,...).
\item Sauvegarde des scores du joueur et statistiques de jeu.
\item Un hébergement de l'interface web et une gestion des utilisateurs (inscription, connexion, choix de l'adversaire, statistiques des joueurs, tournois, points,...).
\item On pourrait également facilement faire une interface pour tablette grâce à la bibliothèque \texttt{kivy} (\texttt{https://kivy.org/}), ou encore une interface avec \texttt{pygame} pour avoir quelque chose de plus ludique (sprites pour les bateaux, effets d'explosion quand on touche, bruitages...).
\item On pourrait aussi envisager une bataille spatiale, sur une grille à trois dimensions  .
\end{itemize}


%Certes il reste des points à développer, comme la mise en place d'une architecture de jeu en réseau, ou encore une gestion des paramètres ou une sauvegarde des scores dans l'interface \texttt{tkinter}, mais je pense que ce projet est déjà bien abouti. 

%\medskip


%
%Cela m'a rappelé mes années d'étude, et notamment mon DESS IM dans lequel je faisais beaucoup de projets informatiques de ce type.
%
%Enfin j'ai trouvé un grand intérêt dans l'obligation de rendre un travail propre et fini, ce qui n'est pas le cas dans un projet personnel, dans lequel on est moins exigeant. Cela faisait longtemps que je n'avais pas fait ça et, rien que pour ça, je suis heureux d'avoir suivi cette formation.
