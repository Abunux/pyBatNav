\chapter{Noyau du programme}
Dans ce chapitre nous allons détailler les structure de données utilisées ainsi que l'algorithme de résolution.
\section{Le module bn\_grille.py}
Ce module définit la grille de jeu. Il permet de gérer les bateaux sur la grille, ainsi que l'état de chaque case.

\subsection{La classe Bateau}
Cette classe, très minimaliste, définit un bateau par sa case de départ, sa taille et sa direction. Elle permet de récupérer :
\begin{itemize}
\item sa case de fin,
\item la liste de ses cases occupées,
\item la liste de ses cases adjacentes.
\end{itemize}

\subsection{La classe Grille}
Cette classe est l'une des principales du projet. Elle gère l'ensemble de la grille et permet de mémoriser :
\begin{itemize}
\item les bateaux du joueur,
\item les cases jouées et leur état (touchée ou manquée)
\end{itemize}

L'attribut \texttt{Grille.etat} fournit l'état de la grille. C'est un dictionnaire indexé par les couples \texttt{(0,0)}, \texttt{(0,1)},... , \texttt{(9,9)}, dans lesquels la première coordonnée correspond à la colonne de la case et la deuxième à sa ligne.\\
L'état d'un case peut être :
\begin{itemize}
\item $0$ : case non jouée
\item $1$ : case touchée
\item $-1$ : case manquée ou impossible
\end{itemize}
L'intérêt d'utiliser un dictionnaire plutôt qu'une double liste tient au fait queles appels sont plus simples et plus naturels et, surtout, que l'utilisation d'une table de hachage permet la recherche d'un élément en $O(1)$.  


\section{Le module bn\_joueur.py}

\section{Algorithme de résolution}

\section{Autres modules du projet}




