\chapter{Algorithmes intéressants}

\section{Gestion de la grille}

\subsection{Détermination des espaces vides}\label{get_max_space}
\texttt{Grille.get\_max\_space(self, case, direction, sens)} :
\begin{algo1}
direction[0]\sto dh\\
direction[1]\sto dv\\
0\sto m\\
case[0]\sto x\\
case[1]\sto y\\
Tant que la case (x+dh, y+dv) est vide et est dans la grille :\\
\tab{1} m+1\sto m\\
\tab{1} x+dh\sto x\\
\tab{1} y+dv\sto y\\
Retourner m\\
\end{algo1}

\subsection{Création d'un flotte aléatoire}\label{init_alea}
\texttt{Grille.init\_alea(self)} :
\begin{algo1}
0\sto nb\_bateaux\\
Tant que nb\_bateaux < nombre de bateaux à placer :\\
\tab{1} 0\sto nb\_bateaux\\
\tab{1} On crée une copie temporaire de la grille dans gtmp\\
\tab{1} Pour chaque taille de bateau à placer :\\
\tab{2} On récupère les positions possibles pour ce bateau dans gtmp\\
\tab{2} Si aucune possibilité :\\
\tab{3} On casse la boucle et on recommence tout\\
\tab{4} (pour éviter une situation de blocage)\\
\tab{2} Sinon :\\
\tab{3} On choisit une position et une direction au hasard\\
\tab{4} (parmi celles possibles)\\
\tab{3} On ajoute le bateau à gtmp\\
\tab{3} nb\_bateaux+1\sto nb\_bateaux\\
Enfin on copie l'état de gtmp dans notre grille \\
\end{algo1}

\subsection{Détermination des probabilités par échantillonnage}\label{case_max_echantillons}
\texttt{Grille.case\_max\_echantillons(self, nb\_echantillons)} :
\begin{algo1}
probas est un dictionnaire indexé sur les cases\\
Pour chaque case, 0\sto probas[case]\\
On répète nb\_echantillons fois :\\
\tab{1}grille\_tmp reçoit une copie temporaire de la grille du suivi\\
\tab{1}On crée une flotte aléatoire sur grille\_tmp\\
\tab{1}Pour chaque case vide dans la grille de suivi originale :\\
\tab{2}Si la case contient un bateau dans grille\_tmp :\\
\tab{3}probas[case]+1\sto probas[case]\\
Pour chaque case, probas[case]/nb\_echantillons\sto probas[case]\\
case\_max est la case qui a la plus grande probabilité pmax\\
On retourne (case\_max, pmax)\\
\end{algo1}

\subsection{Détermination du nombre de bateaux possibles sur chaque case}\label{case_max}
\texttt{Grille.case\_max(self)} :
\begin{algo1}
probas est un dictionnaire indexé sur les cases\\
Pour chaque case, 0\sto probas[case]\\
On récupère la liste des bateaux possibles sur chaque case\\
\tab{1} (dans le dictionnaire possibles)\\
Pour chaque taille de bateau restant :\\
\tab{1}Pour chaque (case, direction) permettant de démarrer le bateau :\\
\tab{2}La probabilité de chaque case occupée par le bateau est augmentée de 1\\
case\_max est la case qui a le plus de possibilités pmax\\
On retourne (case\_max, pmax)\\
\end{algo1}

\newpage
\section{Déroulement de la partie}

Une partie à deux joueurs se déroule selon l'algorithme suivant (le joueur porte le numéro $0$ et son adversaire le numéro $1$) :

\begin{algo1}\label{algo_partie}
Placement des bateaux du joueur\\
Récupération des bateaux de l'adversaire\\
Aléa(0,1)\sto joueur\_en\_cours (celui qui commence)\\
0\sto nb\_coups\\
Tant qu'aucun joueur n'a fini :\\
\tab{1}nb\_coups+1\sto nb\_coups\\
\tab{1}Si joueur\_en\_cours == 0 et l'adversaire n'a pas fini :\\
\tab{2}Le joueur joue un coup\\
\tab{2}Mise à jour de l'affichage des grilles\\
\tab{2}Affichage des messages du joueur (résultat du coup)\\
\tab{1}Sinon :\\
\tab{2}Récupération du coup de l'adversaire\\
\tab{2}Mise à jour de l'affichage des grilles\\
\tab{2}Affichage des messages de l'adversaire\\
\tab{1}(joueur\_en\_cours+1)\%2\sto joueur\_en\_cours (changement de joueur)\\
Affichage des grilles avec la solution\\
Affichage du gagnant\\
\end{algo1}

\newpage
\section{Algorithme de résolution}\label{algo_resolution}
Voici l'algorithme complet de résolution de la grille par l'ordinateur :

\begin{algo1}
La file d'attente est une liste vide\\
liste\_touches est une liste vide\\
Tant que le grille n'est pas résolue :\\
\tab{1}Si la file d'attente est vide :\\
\tab{2}Si liste\_touches n'est pas vide :\\
\tab{3}On enlève le bateau de taille len(liste\_touches)\\
\tab{3}On élimine les cases adjacentes à celles de liste\_touches\\
\tab{3}On vide liste\_touches\\
\tab{2}On élimine les zones trop petites\\
\tab{2}case\_courante reçoit une case en aveugle (suivant le niveau)\\
\tab{1}Sinon :\\
\tab{2}case\_courante reçoit le premier élément de la file d'attente\\
\tab{2}On enlève cette case de la file d'attente\\
\tab{1}On tire sur case\_courante\\
\tab{1}Si on a touché :\\
\tab{2}Si liste\_touches est vide :\\
\tab{3}On ajoute case\_courante dans liste\_touches\\
\tab{3}case\_courante\sto case\_touchee\\
\tab{3}On ajoute ses cases adjacentes dans la file d'attente\\
\tab{4}(en testant également les directions impossibles éventuelles)\\
\tab{2}Sinon :\\
\tab{3}Si len(liste\_touches) == 1 :\\
\tab{4}On détecte la direction du bateau\\
\tab{3}On met à jour la file d'attente\\
\tab{4}(avec la case adjacente à case\_courante dans la bonne direction)\\
\tab{2}Si le bateau touché est le plus grand restant :\\
\tab{3}On vide la file d'attente\\
\tab{1}Sinon :\\
\tab{2}Si len(liste\_touches) == 1 :\\
\tab{3}On met à jour la file d'attente\\
\tab{4}(on élimine éventuellement la case en face de case\_touchee)\\
On affiche le nombre de coups\\
\end{algo1}