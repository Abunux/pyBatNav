\chapter{Étude des algorithmes clés}\label{algo_liste}
Dans cette partie, une description commentée des algorithmes clés du projet. Les algorithmes vraiment simples (comme par exemple tirer sur une case) ne sont pas mis.

Le fait de rédiger cette partie a permis de prendre du recul sur ces algorithmes et de faire quelques améliorations (formaliser les choses est toujours bénéfique).
\section{Gestion de la grille}


\subsection{Détermination des espaces vides}\label{get_max_space}
\texttt{Grille.get\_max\_space(self, case, direction, sens)} :
\begin{algo1}
Si direction == TOUTES\_DIRECTIONS :\\
\tab{1} On renvoie le maximum de get\_max\_space(case, HORIZONTAL, sens=1)\\
\tab{2} et get\_max\_space(case, VERTICAL, sens=1)\\
direction[0]\sto dh\\
direction[1]\sto dv\\
1\sto m\\
case[0]\sto x\\
case[1]\sto y\\
Tant que la case (x+dh, y+dv) est vide et est dans la grille :\\
\tab{1}m+1\sto m\\
\tab{1}x+dh\sto x\\
\tab{1}y+dv\sto y\\
Si sens == 1 (on regarde dans l'autre sens):\\
\tab{1}case[0]\sto x\\
\tab{1}case[1]\sto y\\
\tab{1}Tant que la case (x-dh, y-dv) est vide et est dans la grille :\\
\tab{2}m+1\sto m\\
\tab{2}x-dh\sto x\\
\tab{2}y-dv\sto y\\
Retourner m\\
\end{algo1}

Cet algorithme est très simple. Il se contente de compter dans chaque direction le nombre de cases vides et, éventuellement dans les deux sens.

\newpage

\subsection{Détermination des bateaux possibles démarrant sur chaque case et des placements possibles de chaque bateaux}\label{get_possibles}
\texttt{Grille.get\_possibles(self)} :
\begin{algo1}
On met à jour la liste des cases vides\\
\# Récupération des bateaux possibles sur chaque case\\
possibles\_case est un dictionnaire indexé par les cases\\
Les éléments de possibles\_case sont des listes vides\\
On récupère la liste unique des bateaux restants dans tmp\_taille\_bateaux\\
Pour chaque case vide :\\
\tab{1}Pour chaque direction dans [DROITE, BAS] :\\
\tab{2}tmax est l'espace maximum dans direction\\
\tab{2}On ajoute à possible\_case[case] les tuples (taille, direction)\\
\tab{3}pour les tailles dans tmp\_taille\_bateaux, si taille <= tmax\\
\ \\
\# Récupération des position possibles pour chaque bateau\\
possibles est un dictionnaire indexé par les tailles de bateaux\\
Les éléments de possibles sont des listes vides\\
Pour chaque case index dans possible\_case :\\
\tab{1}Pour chaque couple (taille, direction) dans possible\_case[case] :\\
\tab{2}On ajoute (taille, direction) à possibles[taille]\\ 
\end{algo1}

Cet algorithme est un des points clés du projet. Dans un premier temps on récupère la liste de tous les bateaux possibles démarrant sur chaque case, ainsi que leurs directions possibles. Dans un deuxième temps on récupère, pour chaque bateau, ses placements possibles. 

Les dictionnaires créés ont cette allure :
\begin{itemize}
\item \texttt{possibles\_case=\{(0,0):[(5,(1,0)), (5,(0,1)),...], (0,1):...\}}
\item \texttt{possibles=\{5:[((0,0), (1,0)), ((0,0), (0,1)), ((1,0), (1,0)),...], 4:...\}}
\end{itemize}

\medskip

Afin d'éviter de répéter les mêmes calculs, la liste \texttt{taille\_bateaux} des bateaux restants sur la grille est transformée en une liste \texttt{tmp\_taille\_bateaux} qui ne contient que ses éléments uniques. Pour cela on la convertit en ensemble puis on revient à une liste avec l'instruction 
\begin{center}
\texttt{tmp\_taille\_bateaux = list(set(taille\_bateaux))}
\end{center}

\medskip

À des fin de tests, cette fonction accepte un argument \texttt{tri} qui, s'il est Vrai trie les résultats. Ce tri est uniquement nécessaire pour les tests et est mis à Faux par défaut (cela prend un petit peu plus de temps de trier à chaque fois).

\newpage

\subsection{Création de bateaux aléatoires}
\subsubsection{Ajout d'un bateau aléatoire}\label{add_bateau_alea}
\texttt{Grille.add\_bateau\_alea(self, taille)} :
\begin{algo1}
Récupérer les placements de départ possibles des bateaux\\
\tab{1}(dans le dictionnaire possibles)\\
Si possibles[taille] est vide :\\
\tab{1}Retourner Faux\\
Sinon :\\
\tab{1}Choisir un tuple (case, direction) au hasard dans possibles[taille]\\
\tab{1}Ajouter la bateau (taille, case, direction)\\
\tab{1}Retourner Vrai\\ 
\end{algo1}

Si \texttt{possibles[taille]} est vide, cela signifie qu'on ne peut pas placer ce bateau et donc en renvoie Faux pour éviter une situation de blocage par la suite, sinon on renvoie Vrai pour indiquer que tout s'est bien passé.

\subsubsection{Création d'une flotte aléatoire}\label{init_alea}
\texttt{Grille.init\_alea(self)} :
\begin{algo1}
0\sto nb\_bateaux\\
Tant que nb\_bateaux < nombre de bateaux à placer :\\
\tab{1}0\sto nb\_bateaux\\
\tab{1}On crée une copie temporaire de la grille dans gtmp\\
\tab{1}Pour chaque taille de bateau à placer :\\
\tab{2}On essaie de placer un bateau aléatoire de cette taille dans gtmp\\
\tab{2}Si pas possible :\\
\tab{3}On casse la boucle et on recommence tout\\
\tab{4}(pour éviter une situation de blocage)\\
\tab{2}Sinon :\\
\tab{3}nb\_bateaux+1\sto nb\_bateaux\\
\tab{3}On enlève taille de la liste des bateaux restants à placer\\
Enfin on copie l'état de gtmp dans notre grille \\
\end{algo1}

Pour placer chaque bateau, on utilise la fonction précédente avec sa valeur de sortie (vrai ou faux) qui permet de savoir si le placement a été possible.

%\begin{algo1}
%0\sto nb\_bateaux\\
%Tant que nb\_bateaux < nombre de bateaux à placer :\\
%\tab{1} 0\sto nb\_bateaux\\
%\tab{1} On crée une copie temporaire de la grille dans gtmp\\
%\tab{1} Pour chaque taille de bateau à placer :\\
%\tab{2} On récupère les positions possibles pour ce bateau dans gtmp\\
%\tab{2} Si aucune possibilité :\\
%\tab{3} On casse la boucle et on recommence tout\\
%\tab{4} (pour éviter une situation de blocage)\\
%\tab{2} Sinon :\\
%\tab{3} On choisit une position et une direction au hasard\\
%\tab{4} (parmi celles possibles)\\
%\tab{3} On ajoute le bateau à gtmp\\
%\tab{3} nb\_bateaux+1\sto nb\_bateaux\\
%Enfin on copie l'état de gtmp dans notre grille \\
%\end{algo1}

\newpage
\section{Algorithme de résolution}
\subsection{Optimisation de la phase de tirs en aveugle}
\subsubsection{Détermination des probabilités par échantillonnage}\label{case_max_echantillons}

\texttt{Grille.case\_max\_echantillons(self, nb\_echantillons)} :
\begin{algo1}
probas est un dictionnaire indexé sur les cases\\
Pour chaque case, 0\sto probas[case]\\
On répète nb\_echantillons fois :\\
\tab{1}grille\_tmp reçoit une copie temporaire de la grille du suivi\\
\tab{1}On crée une flotte aléatoire sur grille\_tmp\\
\tab{1}Pour chaque case vide dans la grille de suivi originale :\\
\tab{2}Si la case contient un bateau dans grille\_tmp :\\
\tab{3}probas[case]+1\sto probas[case]\\
Pour chaque case, probas[case]/nb\_echantillons\sto probas[case]\\
case\_max est la case qui a la plus grande probabilité pmax\\
On retourne (case\_max, pmax)\\
\end{algo1}

C'est l'algorithme de niveau 4 qui crée une estimation de la distribution de probabilité avec des échantillons de flottes aléatoires. Aucune difficulté particulière n'est à signaler.


\subsubsection{Détermination du nombre de bateaux possibles sur chaque case}\label{case_max}
\texttt{Grille.case\_max(self)} :
\begin{algo1}
probas est un dictionnaire indexé sur les cases\\
Pour chaque case, 0\sto probas[case]\\
On récupère la liste placements possibles pour chaque bateau\\
\tab{1} (dans le dictionnaire possibles)\\
Pour chaque taille de bateau restant :\\
\tab{1}Pour chaque (case, direction) dans possibles[taille] :\\
\tab{2}\#La probabilité de chaque case occupée par le bateau est augmentée de 1\\
\tab{2}Pour k allant de 0 à taille :\\
\tab{3}probas[case+k*direction] est augmentée de 1\\
case\_max est la case qui a le plus de possibilités pmax\\
On retourne (case\_max, pmax)\\
\end{algo1}

C'est l'algorithme de niveau 5 qui détermine, pour chaque case, le nombre de bateaux possibles.

Dans la mesure où la liste \texttt{possibles[taille]} ne donne que les points de départ de chaque bateau, il est nécessaire de parcourir toutes ses cases occupée dans la dernière boucle. \texttt{case+k*direction} signifie \texttt{(case[0]+k*direction[0],case[1]+k*direction[1])}.

\newpage
\subsubsection{Énumération de tous les cas}
\texttt{Grille.case\_max\_all(self)} :
\begin{algo1}
gtmp est une copie temporaire de la grille\\
probas\_all est un dictionnaire indexé par les cases\\
Pour chaque case, 0\sto probas\_all[case]\\
Créer toutes les répartitions de bateaux sur gtmp\\
Récupérer la case qui en contient le plus\\ 
\end{algo1}

\texttt{Grille.make\_all(self, gtmp)} :
\begin{algo1}
\# En entrée : une grille gtmp\\
S'il n'y a plus de bateaux dans la liste de gtmp :\\
\tab{1}Mettre à jour les probabilités avec les cases occupées sur gtmp\\
\tab{1}Sortir de la fonction\\
Récupérer les positions de bateaux possibles sur gtmp\\
taille est la taille du premier bateau restant\\
Pour (case, direction) dans possibles[taille] :\\
\tab{1}gtmp2 est une copie temporaire de gtmp\\
\tab{1}Ajouter le bateau (taille, case, direction) à gtmp2\\
\tab{1}Enlever ce bateau de la liste de ceux de gtmp2\\
\tab{1}Créer toutes les répartitions de bateaux sur gtmp2\\
\end{algo1}

Cet algorithme est récursif. On utilise une grille temporaire \texttt{gtmp} sur laquelle on va faire tous les arrangements possibles. Pour chaque placement possible d'un bateau, on crée une autre grille temporaire \texttt{gtmp2} sur laquelle on va recommencer à faire tous les arrangements possibles avec les bateaux restants.

\medskip

Le problème de cet algorithme est qu'il est exponentiel. Avec une grille vide de $10\times 10$, on obtient les données suivantes :
\begin{itemize}
\item \texttt{taille\_bateaux=[5,4]} : $\approx 10^4$ répartitions, 0,8 secondes
\item \texttt{taille\_bateaux=[5,4,3]} : $\approx 10^6$ répartitions, 72 secondes
\item \texttt{taille\_bateaux=[5,4,3,2]} : $101\,286\,480$ répartitions, 6566 secondes
\end{itemize}

À chaque fois qu'on ajoute un bateau, le nombre de répartitions possibles, ainsi que le temps de calcul pour les déterminer, est multiplié par 100.

Pour une liste de bateaux \texttt{taille\_bateaux=[5,4,3,3,2]} on s'attend donc à $10^{10}$ répartitions et un temps de calcul d'environ 8 jours (pour un seul coup !).

Certes, sur une grille vide, on pourrait jouer avec les symétries (et diviser le calcul par $8$) mais celles-ci sont tout de suite brisées dès qu'une case a été jouée.

Même si cette méthode est inutilisable en tant que telle, elle peut être intéressante sur de petites grilles, ou avec peu de bateaux, et donc peut être utilisée en fin de partie. C'est le rôle du niveau 6 dans lequel on utilise cette méthode dès que le nombre de cases vides descend en-dessous d'un certain seuil.

Pour information voici, sur la page suivante, le nombre de répartitions sur chaque case pour \texttt{taille\_bateaux=[5,4,3,2]} :


\begin{center}
\begin{adjustbox}{angle=90}
\begin{tabular}{|c|c|c|c|c|c|c|c|c|c|c|}
\hline
& \texttt{A} & \texttt{B} & \texttt{C} & \texttt{D} & \texttt{E} & \texttt{F} & \texttt{G} & \texttt{H} & \texttt{I} & \texttt{J} \\
\hline
\texttt{0} & 8138842 & 11060807 & 13598504 & 15286936 & 16111642 & 16111642 & 15286936 & 13598504 & 11060807 & 8138842\\
\hline
\texttt{1} & 11060807 & 12059986 & 13448547 & 14238869 & 14511549 & 14511549 & 14238869 & 13448547 & 12059986 & 11060807\\
\hline
\texttt{2} & 13598504 & 13448547 & 14505458 & 15088255 & 15244715 & 15244715 & 15088255 & 14505458 & 13448547 & 13598504\\
\hline
\texttt{3} & 15286936 & 14238869 & 15088255 & 15542880 & 15653410 & 15653410 & 15542880 & 15088255 & 14238869 & 15286936\\
\hline
\texttt{4} & 16111642 & 14511549 & 15244715 & 15653410 & 15769046 & 15769046 & 15653410 & 15244715 & 14511549 & 16111642\\
\hline
\texttt{5} & 16111642 & 14511549 & 15244715 & 15653410 & 15769046 & 15769046 & 15653410 & 15244715 & 14511549 & 16111642\\
\hline
\texttt{6} & 15286936 & 14238869 & 15088255 & 15542880 & 15653410 & 15653410 & 15542880 & 15088255 & 14238869 & 15286936\\
\hline
\texttt{7} & 13598504 & 13448547 & 14505458 & 15088255 & 15244715 & 15244715 & 15088255 & 14505458 & 13448547 & 13598504\\
\hline
\texttt{8} & 11060807 & 12059986 & 13448547 & 14238869 & 14511549 & 14511549 & 14238869 & 13448547 & 12059986 & 11060807\\
\hline
\texttt{9} & 8138842 & 11060807 & 13598504 & 15286936 & 16111642 & 16111642 & 15286936 & 13598504 & 11060807 & 8138842\\
\hline
\multicolumn{11}{c}{}\\
\multicolumn{11}{c}{\textit{Nombre de bateaux sur chaque case pour une grille vide avec \texttt{taille\_bateaux=[5,4,3,2]}}}\\
\multicolumn{11}{c}{\textit{à partir d'une liste exhaustive de toutes les répartitions possibles}}
\end{tabular}
\end{adjustbox}
\end{center}

\newpage

\subsection{Gestion de la file d'attente}
\subsubsection{Création de la file d'attente}\label{add_adjacentes_premiere}
\texttt{Ordi.add\_adjacentes\_premiere(self)} :
\begin{algo1}
adjacent est la liste des cases vides adjacentes à case\_touchee\\
taille\_min est la taille minimale des bateaux restants\\
Pour direction dans [HORIZONTAL, VERTICAL] :\\
\tab{1}direction[0]\sto k\\
\tab{1}Si l'espace maximum dans direction sur case touché >= taille\_min :\\
\tab{2}Pour case dans adjacent :\\
\tab{3}Si case[k]==case\_touchee[k] :\\
\tab{4}Ajouter case à la file d'attente\\
\tab{3}Sinon afficher que cette direction ne convient pas\\
\end{algo1}

Pas de grande difficulté sur celui-ci. On se contente de regarder si le plus petit bateau rentre dans chaque direction et on ajoute à la file d'attente les cases correspondantes.

Les tests des deux directions sont regroupés dans une boucle afin de rendre le code plus compact.
\newpage
\subsubsection{Optimisation de la file d'attente}\label{case_max_touchee}
\texttt{Grille.case\_max\_touchee(self, case\_touchee)} :

Pour des raisons de mise en page, notons \texttt{ct} la case touchée.
\begin{algo1}
On marque temporairement case\_touchee comme vide\\ 
On récupère la liste placements possibles pour chaque bateau\\
\tab{1} (dans le dictionnaire possibles)\\
probas est un dictionnaire indexé sur les cases\\
Pour chaque case, 0\sto probas[case]\\
Pour chaque taille de bateau restant :\\
\tab{1}Pour chaque direction dans [HORIZONTAL, VERTICAL] :\\
\tab{2}\#Bateau qui se termine sur case\_touchee\\
\tab{2}Si ct[direction[1]]-(taille-1)*direction[direction[1]]>=0\\
\tab{5}et ((ct[0]-(taille-1)*direction[0],\\ 
\tab{7}ct[1]-(taille-1)*direction[1]), direction)\\
\tab{7}est dans possibles :\\
\tab{3}probas[(ct[0]-direction[0],ct[1]-direction[1])] += 1\\
\tab{2}\#Bateau à cheval sur sur case\_touchee\\
\tab{2}Pour k allant de 1 à taille-2 :\\
\tab{3}Si ct[direction[1]]-k*direction[direction[1]]>=0\\
\tab{6}et ((ct[0]-k*direction[0], ct[1]-k*direction[1]),\\
\tab{6} direction) est dans possibles :\\
\tab{4}probas[(ct[0]-direction[0],ct[1]-direction[1])] += 1\\
\tab{4}probas[(ct[0]+direction[0],ct[1]+direction[1])] += 1\\
\tab{2}\#Bateau qui démarre sur case\_touchee\\
\tab{2}Si ((ct[0], ct[1]), direction) est dans possibles :\\
\tab{3}probas[(ct[0]+direction[0],ct[1]+direction[1])] += 1\\
On remet l'état de case\_touchee à touché\\
On trie probas dans l'ordre décroissant du nombre de possibilités\\
On renvoie cette liste\\
\end{algo1}

Le fait de boucler sur les directions \texttt{HORIZONTAL} et \texttt{VERTICAL}, c'est-à-dire dans\texttt{[(1,0), (0,1)]}, permet de condenser les calculs sur ces deux directions dans une seule boucle.\\
\texttt{ct[direction[1]]} et \texttt{direction[direction[1]]} permettent d'obtenir respectivement \texttt{ct[0]} et \texttt{direction[0]} dans le cas horizontal, et \texttt{ct[1]} et \texttt{direction[1]} dans le cas vertical.

Notons que dans cet algorithme on en s'intéresse qu'aux cases adjacentes à \texttt{case\_touchee}.

\medskip

Le tri final de la liste se fait de la façon suivante : on commence par convertir le dictionnaire \texttt{probas} en une liste de tuple \texttt{(case, probas[case])} dans \texttt{probas\_liste}, puis un l'ordonne avec l'instruction
\begin{center}
\texttt{sorted(probas\_liste, key=lambda proba: proba[1], reverse = True)}
\end{center}

\newpage
\subsubsection{Mise à jour après le deuxième coup touché}\label{update_queue_touche}
\texttt{Ordi.update\_queue\_touche(self)} :
\begin{algo1}
Si case\_courante[1] == case\_touchee[1] :\\
\tab{1}direction=HORIZONTAL\\
Sinon :\\
\tab{1}direction=VERTICAL\\
Si on vient de découvrir la direction du bateau (len(liste\_touchee==1)):\\
\tab{1}On affiche cette direction\\
\tab{1}On enlève les cases \\
\tab{3} (case\_touchee[0]-direction[1], case\_touchee[1]-direction[0])\\ 
\tab{3}et (case\_touchee[0]+direction[1], case\_touchee[1]+direction[0])\\
\tab{3}de la file d'attente\\
nv\_case est la case\\
\tab{2}(case\_courante[0] + direction[0]*signe(case\_courante[0]-case\_touchee[0]),\\
\tab{2} case\_courante[1] + direction[1]*signe(case\_courante[1]-case\_touchee[1]))\\
Si nv\_case est dans la grille et est vide :\\
\tab{1}On ajoute nv\_case à la file d'attente\\
On ajoute case\_courante à liste\_touches\\
\end{algo1}

La liste \texttt{liste\_touches} permet de savoir combien de cases ont été touchées sur ce bateau. Si c'est la deuxième, on met à jour les file d'attente avec les case dans la bonne direction.

\texttt{signe(x)} renvoie 1 si $x>0$ et $-1$ si $x<0$.\\
\texttt{signe(case\_courante[i]-case\_touche[i])}, pour $i\in\{0,1\}$, permet de déterminer le sens dans lequel on vient de toucher la nouvelle case, ce qui permet d'ajouter la case en bout de configuration à la file d'attente (si elle est valide et vide).

\newpage
\subsubsection{Mise à jour après le deuxième coup manqué}\label{update_queue_manque}
\texttt{Ordi.update\_queue\_manque(self)} :

\begin{algo1}
taille\_min est la plus petite taille de bateau restant\\
delta = \\
\tab{1}(case\_courante[0]-case\_touchee[0],\\
\tab{1} case\_courante[1]-case\_touchee[1])\\
direction = \\
\tab{1}(abs(case\_courante[0]-case\_touchee[0]),\\
\tab{1} abs(case\_courante[1]-case\_touchee[1]))\\
case\_face = \\
\tab{1}(case\_touchee[0]-delta[0],\\
\tab{1} case\_touchee[1]-delta[1])\\
Si l'espace vide sur case\_face dans direction < taille\_min-1 :\\
\tab{1}Enlever case\_face de la file d'attente\\
\end{algo1}

\begin{itemize}
\item \texttt{delta} est l'écart entre la première case touchée du bateau et la case sur laquelle on vient de jouer
\item \texttt{direction} est la direction dans laquelle on vient de jouer
\item \texttt{case\_face} est la case en face de celle qu'on vient de jouer
\end{itemize}

\medskip

Dans la mesure où il y a déjà eu une case touchée sur ce bateau, on teste s'il y a assez de place sur \texttt{case\_face} pour faire rentrer un bateau de taille \texttt{taille\_min}-1 dans la direction \texttt{direction}.

\newpage
\subsection{Algorithme complet de résolution}\label{algo_resolution}
Voici l'algorithme complet de résolution de la grille par l'ordinateur :

\begin{algo1}
La file d'attente est une liste vide\\
liste\_touches est une liste vide\\
Tant que le grille n'est pas résolue :\\
\tab{1}Si la file d'attente est vide (tir en aveugle) :\\
\tab{2}Si liste\_touches n'est pas vide :\\
\tab{3}On enlève le bateau de taille len(liste\_touches)\\
\tab{3}On élimine les cases adjacentes à celles de liste\_touches\\
\tab{3}On vide liste\_touches\\
\tab{2}On élimine les zones trop petites\\
\tab{2}case\_courante reçoit une case en aveugle (suivant le niveau)\\
\tab{1}Sinon (tir ciblé) :\\
\tab{2}case\_courante reçoit le premier élément de la file d'attente\\
\tab{2}On enlève cette case de la file d'attente\\
\tab{1}On tire sur case\_courante\\
\tab{1}Si on a touché :\\
\tab{2}Si liste\_touches est vide (1ère case du bateau) :\\
\tab{3}On ajoute case\_courante dans liste\_touches\\
\tab{3}case\_courante\sto case\_touchee\\
\tab{3}On ajoute ses cases adjacentes dans la file d'attente\\
\tab{4}(en testant également les directions impossibles éventuelles)\\
\tab{2}Sinon :\\
\tab{3}Si len(liste\_touches) == 1 (2éme case du bateau):\\
\tab{4}On détecte la direction du bateau\\
\tab{3}On met à jour la file d'attente\\
\tab{4}(avec la case adjacente à case\_courante dans la bonne direction\\
\tab{4} si elle est vide)\\
\tab{2}Si le bateau touché est le plus grand restant :\\
\tab{3}On vide la file d'attente\\
\tab{1}Sinon :\\
\tab{2}Si len(liste\_touches) == 1 :\\
\tab{3}On met à jour la file d'attente\\
\tab{4}(on élimine éventuellement la case en face de case\_touchee)\\
On affiche le nombre de coups\\
\end{algo1}

\newpage
\section{Déroulement de la partie}\label{algo_partie}

Une partie à deux joueurs se déroule selon l'algorithme suivant (le joueur porte le numéro $0$ et son adversaire le numéro $1$) :

\begin{algo1}
Placement des bateaux du joueur\\
Récupération des bateaux de l'adversaire\\
Aléa(0,1)\sto joueur\_en\_cours (celui qui commence)\\
%0\sto nb\_coups\\
Tant qu'aucun joueur n'a fini :\\
%\tab{1}nb\_coups+1\sto nb\_coups\\
\tab{1}Si joueur\_en\_cours == 0 et l'adversaire n'a pas fini :\\
\tab{2}Le joueur joue un coup\\
\tab{2}Mise à jour de l'affichage des grilles\\
\tab{2}Affichage des messages du joueur (résultat du coup)\\
\tab{1}Sinon :\\
\tab{2}Récupération du coup de l'adversaire\\
\tab{2}Mise à jour de l'affichage des grilles\\
\tab{2}Affichage des messages de l'adversaire\\
\tab{1}(joueur\_en\_cours+1)\%2\sto joueur\_en\_cours (changement de joueur)\\
Affichage des grilles avec la solution\\
Affichage du gagnant\\
\end{algo1}

