\chapter{Interface graphique}

Une interface réalisée grâce à la bibliothèque \texttt{tkinter} est codée dans le fichier \texttt{bn\_tkinter}. Dans cette interface, on a la possibilité de joueur seul sur une grille aléatoire, de voir l'algorithme de résolution (avec choix du niveau) et de jouer contre l'algorithme (avec choix du niveau de ce dernier et placement de nos bateaux).

Cette interface devait être réalisée par mon binôme, mais ce dernier ayant abandonné la formation quelques semaines avant las date de rendu de projet, j'ai du la coder très rapidement et, malgré l'urgence, je suis assez satisfait du résultat.

\section{Principes de l'interface}

Le point principal se situe dans la classe \texttt{GrilleTK(Grille)} qui implémente les fonctionnalités graphiques (avec un \texttt{canvas}) des grilles.

La fenêtre principale est basée sur une \texttt{Frame},\texttt{main\_frame}, dans laquelle on va placer les grilles, ainsi que des fenêtres de texte. Entre chaque partie on l'efface en supprimant tous ses widgets enfants, obtenus dans la liste \texttt{main\_frame.pack\_slaves()}, grâce à leurs méthodes \texttt{destroy()}. 

J'ai également créé deux fenêtres annexes pour placer nos bateaux (\texttt{PlaceWin}) et choisir pour le niveau de l'algorithme (\texttt{LevelWin}).

Enfin les classes \texttt{JoueurTK(Joueur)} et \texttt{OrdiTK(Ordi, JoueurTK)} implémentent les fonctions graphiques, et notamment la gestion des événements souris, aux joueurs. Notons que pour pouvoir accéder à la boîte de texte \texttt{info} de l'interface principale qui affiche les informations de partie, on met l'attribut \texttt{name="info"} à ce widget et on y accède depuis la classe \texttt{JoueurTK(Joueur)} via \texttt{self.parent.children["info"]}, où \texttt{self.parent} est la frame principale (\texttt{main\_frame}).

\section{Affichage et gestion des grilles}