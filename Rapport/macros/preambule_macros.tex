\documentclass[a4paper,12pt,fleqn]{book}

% ---------------------------------------------------------------------------
%				Packages
% ---------------------------------------------------------------------------

% texdoc <nom_package> pour avoir des infos
\usepackage{etex}

\usepackage[utf8]{inputenc}						% Encodage français
\usepackage[frenchb]{babel}						% Mise en forme française
\usepackage[T1]{fontenc}						% Encodage caractères français
\usepackage{pmboxdraw}
\usepackage{newunicodechar}
\newunicodechar{✖}{×}
%\newunicodechar{◯}{o}
\newunicodechar{┪}{\pmboxdrawuni{252A}}
\newunicodechar{┨}{\pmboxdrawuni{2528}}
\newunicodechar{┩}{\pmboxdrawuni{2529}}

\usepackage{fourier}							% Différents symboles et polices
\usepackage[scaled=0.875]{helvet}				% Font générale
\usepackage{courier}							% Font télétype
\renewcommand{\ttdefault}{lmtt}					% Font télétype
\usepackage{frcursive}							% Ecriture manuscrite type écolier
\usepackage{calligra}							% Ecriture manuscrite classieuse
\usepackage{verbatim}							% Commentaires
%\newcommand{\verbatimfont}[1]{\renewcommand{\verbatim@font}{\ttfamily#1}}
%\newcommand{\verbatimfont}[1]{\renewcommand{\verbatim@font}{\ttdefault#1}}

%\usepackage{fancyvrb}

\usepackage[toc,page]{appendix}

\usepackage{listings}
\usepackage{amsfonts,amsmath,amssymb}			% Symboles maths
\usepackage{bm}									% Symboles maths en gras \bm{}
\usepackage{amstext}							% Texte en mode math de taille adaptée
\usepackage{amsopn}								% \DeclareMathOperator
\usepackage{mathrsfs}							% Symboles maths
\usepackage{mathtools}							% Symboles maths
\usepackage{theorem}							% Mise en forme des théorèmes

\usepackage{textcomp}							% Symboles
\usepackage{pifont}								% Symboles "ding"
\usepackage{wasysym}							% Symboles (smiley et logos)
\usepackage{epsdice}							% Symboles (faces d'un dé)
\usepackage[normalem]{ulem}						% Fioritures de texte (barré, etc...)
\usepackage{cancel}								% Barrer du texte (simplifier termes)
\usepackage{fancybox}							% Boîtes

\usepackage{tabularx}							% Tableaux évolués
\usepackage{diagbox}							% Cases en diagonale
%~ \usepackage{tabls}							% Espaces dans les tableaux (conflit avec bclogo)
\usepackage{multirow}							% Fusionner les lignes d'un tableau
\usepackage{enumerate}							% Enumérations personnalisées
\usepackage{multicol}							% Environnement multicolonnes
\usepackage{fancyhdr}							% En-têtes et pieds de page
%~ \usepackage[np]{numprint}					% Mise en forme des nombres

\usepackage[usenames, dvipsnames]{xcolor}		% Couleurs
\usepackage{graphicx}							% Insérer des images
\usepackage{pgf, tikz, tkz-tab, tkz-fct}		% Graphiques avec Tikz
\usetikzlibrary{arrows}
\usetikzlibrary{snakes}
\usepackage{alterqcm}							% QCM
\usepackage{circuitikz}							% Circuit éléctriques
\usepackage[tikz]{bclogo}						% Boites à logo

\usepackage{titlesec}							% Mise en forme des titres de sections
\usepackage{lastpage}							% Dernière page : \pageref{LastPage}

\usepackage{ifthen}								% Programmation conditions
\usepackage{multido}							% Boucles
\usepackage{calc}								% Calculs

\usepackage{footnote}
\makesavenoteenv{tabular}

% ---------------------------------------------------------------------------
%				Macros simples (caractères)
% ---------------------------------------------------------------------------

\newcommand{\euro}{\eurologo{}}
\newcommand{\R}{\ensuremath{\mathbb{R}}}
\newcommand{\N}{\ensuremath{\mathbb{N}}}
\newcommand{\D}{\ensuremath{\mathbb{D}}}
\newcommand{\Z}{\ensuremath{\mathbb{Z}}}
\newcommand{\Q}{\ensuremath{\mathbb{Q}}}
\newcommand{\C}{\ensuremath{\mathbb{C}}}
\newcommand{\e}{\text{e}}
\renewcommand{\i}{\text{i}}
\newcommand{\s}{\ensuremath{\mathcal{S}}}
\newcommand{\sol}[1]{\mathcal{S}=\left\lbrace #1 \right\rbrace}
\newcommand{\ou}{\mbox{ ou }}
\newcommand{\et}{\mbox{ et }}
\newcommand{\si}{\mbox{ si }}
\newcommand{\Df}{\ensuremath{\mathcal{D}_f}}
\newcommand{\Cf}{\ensuremath{\mathcal{C}_f}}
\newcommand{\Dg}{\ensuremath{\mathcal{D}_g}}
\newcommand{\Cg}{\ensuremath{\mathcal{C}_g}}

\renewcommand{\P}{\ensuremath{\text{P}}}
\newcommand{\card}{\text{card}}
\newcommand{\E}{\text{E}}
\newcommand{\V}{\text{V}}

\newcommand{\FI}{\textbf{F.I.}}

\newcommand{\eq}{\ \Leftrightarrow\ } % ou \iff
\newcommand{\implique}{\Rightarrow}
\newcommand{\pheq}{\phantom{\eq}}
\newcommand{\egdef}{\stackrel{\textit{déf}}{=}}

\renewcommand{\ge}{\geqslant}
\renewcommand{\le}{\leqslant}
\newcommand{\supeg}{\geqslant}
\newcommand{\infeg}{\leqslant}

\newcommand{\lacco}{\left\lbrace}
\newcommand{\racco}{\right\rbrace}
\newcommand{\labs}{\left|}
\newcommand{\rabs}{\right|}

\newcommand{\inclus}{\subset}
\newcommand{\ninclus}{\not\subset}
\newcommand{\union}{\cup}
\newcommand{\inter}{\cap}

\newcommand{\non}[1]{\text{non(}#1\text{)}}

\newcommand{\dx}{~\text{d}x}
\newcommand{\dt}{~\text{d}t}

\renewcommand{\Re}{\text{Re}}
\renewcommand{\Im}{\text{Im}}
\newcommand{\conj}[1]{\overline{#1}}
\newcommand{\abs}[1]{|#1|}

\newcommand{\pinf}{+\infty}
\newcommand{\minf}{-\infty}
\newcommand{\pminf}{\pm\infty}

\newcommand{\para}{\ /\!\!/\ }

\newcommand{\comb}[2]{\text{C}_{#1}^{#2}}

\newcommand{\vect}[1]{\mathchoice
	{\overrightarrow{\displaystyle\mathstrut#1\,\,}}
	{\overrightarrow{\textstyle\mathstrut#1\,\,}}
	{\overrightarrow{\scriptstyle\mathstrut#1\,\,}}
	{\overrightarrow{\scriptscriptstyle\mathstrut#1\,\,}}}

\def\Oij{$\left(\text{O},~\vect{i},~\vect{j}\right)$}
\def\Oijk{$\left(\text{O},~\vect{i},~ \vect{j},~ \vect{k}\right)$}
\def\Ouv{$\left(\text{O},~\vect{u},~\vect{v}\right)$}

\newcommand{\vu}{\vect{u}}
\newcommand{\vv}{\vect{v}}
\newcommand{\vw}{\vect{w}}
\newcommand{\vn}{\vect{n}}

\newcommand{\veccol}[3]{\left(\begin{array}{c}
{#1}\\{#2}\\{#3}
\end{array}\right)}

\definecolor{gris}{gray}{0.85}
\newcommand{\surl}[1]{\colorbox{gris}{\textbf{#1}}}

\renewcommand{\emph}{\textbf}

\newcommand{\saut}{\ \\}
\newcommand{\lignesep}{\vspace*{5pt}\hrule\vspace*{5pt}}

\newcommand{\fct}[5]{
	\begin{array}[t]{r ccl}
	{#1}\ : \ &{#2}&\longrightarrow&{#3}\\
	&{#4}&\longmapsto&{#5}
	\end{array}}

\newcommand{\encadre}[1]{\fbox{\begin{minipage}{\textwidth}
#1
\end{minipage}}}

%\newcommand{\boite}[1]{\fbox{\Huge\phantom{A}\hspace*{#1}}}
\newcommand{\boite}[1]{\fbox{\rule[-0.2cm]{0pt}{0.6cm}\hspace*{#1}}}


\newcommand{\suit}{\begin{tikzpicture}
\draw[color=white](-0.4em,0em)--(-0.25em,0em);
\draw ((-0.25em,-0.15em)--(0.07em,0.18em);
\draw (0.07em,0.18em) arc (135:0:0.25em);
\draw (0.5em,0em) arc (-180:-45:0.25em);
\draw (0.93em,-0.18em)--(1.36em,0.25em);
\draw (1.01em,0.25em)--(1.36em,0.25em)--(1.36em,-0.1em);
\draw[color=white](1.40em,0em)--(1.55em,0em);
\end{tikzpicture}}


%\newcommand{\suit}{\hookrightarrow}

% Commande programmation Casio

\newcommand{\touche}[1]{\fbox{\texttt{#1}}}

\newcommand{\toucheF}[2]{
$\underset{\text{\scriptsize F#2}}{\touche{#1}}$
}

\newcommand{\sto}{\ensuremath{\rightarrow}}

%\newcommand{\toucheFF}[2]{\begin{tabular}[t]{c}
%\touche{#1}\\
%{\scriptsize F#2}
%\end{tabular}}

\newcommand{\suiv}{$\vartriangleright$} % Menu suivant

\newcommand{\rl}{\begin{tikzpicture}[scale=0.7] % Retour à la ligne de la Casio
\draw[color=white] (0,0)--(0,0);
\draw [<-] (0.2em,0.2em)--(1em,0.2em)--(1em,0.8em);
\end{tikzpicture}}

\newcommand{\disp}{\begin{tikzpicture}[scale=0.7] % Triangle de la Casio
\draw[color=white] (0,0)--(0,0);
\fill (0.2em,0em)--(0.8em,0em)--(0.8em,0.8em)--(0.2em,0em);
\draw[color=white] (1em,1em)--(1em,1em);
\end{tikzpicture}}

\newcommand{\vers}{$\rightarrow$}

\newcommand{\enonce}{\textbf{Énoncé :}}
\newcommand{\solution}{\textbf{Solution :}}
\newcommand{\tq}{~,~}

\newcommand{\xmin}{x_{\text{min}}}
\newcommand{\xmax}{x_{\text{max}}}

% -------------------- Symboles : -------------------------------------------

\newcommand{\happy}{\smiley}
\newcommand{\sad}{\frownie}

\newcommand{\attention}{\danger}
\newcommand{\piege}{\bomb}
\newcommand{\interdit}{\noway}

\newcommand{\facede}[1]{\epsdice{#1}}

\newcommand{\hand}{\text{\ding{43}}}
\newcommand{\victory}{\text{\ding{44}}}

\newcommand{\trefle}{\text{\ding{168}}}
\newcommand{\carreau}{\text{\ding{169}}}
\newcommand{\coeur}{\text{\ding{170}}}
\newcommand{\pique}{\text{\ding{171}}}

\newcommand{\checkbox}{\text{\ding{114}}}
\newcommand{\checkedbox}{\text{\mbox{\ding{114}\hspace{-.7em}\raisebox{.2ex}[1ex]{\ding{51}}}}}

\newcommand{\scisors}{\ding{34}}
\newcommand{\couperici}{\scisors\dotfill\textit{\small{couper ici}}\dotfill\scisors}

% ---------------------------------------------------------------------------
% 				Environnements 
% ---------------------------------------------------------------------------

% ------------------------ Théorèmes ----------------------------------------

\theorembodyfont{\normalfont} \theoremstyle{break}



%\newcommand{\Ex}{\noindent\textbf{Exemple : }}
\newcommand{\Rappel}{\noindent\textbf{Rappel : }}

% ------------------------ Python --------------------------------------------

\newcounter{cptspace}
\newcommand{\tab}[1]{
	\setcounter{cptspace}{#1}
	\whiledo{\value{cptspace}>0}{
		\hspace*{0em}\hspace*{0em}
		\addtocounter{cptspace}{-1}}}

\newcommand{\prompt}{{>}{>}{>}\ }

\newenvironment{python}
	{\par\ttfamily\small\vspace{0.2cm}
	\setbox0=\hbox\bgroup
	\begin{minipage}{\textwidth}
	\vspace{0.2cm}
%	\begin{tabbing}
	}
	{\vspace{0.2cm}
%	\end{tabbing}
	\end{minipage}
	\egroup
    \fbox{\box0}
	\par\rmfamily\normalsize\vspace{0.2cm}\noindent
	}

\newenvironment{algo1}
	{\begin{center}\begin{tabular}{|>{\texttt\bgroup}l<{\egroup}|}
	\hline}
	{\hline
	\end{tabular}\end{center}
	}

%\newenvironment{python}
%	{\par\ttfamily\small\vspace{0.2cm}
%	\begin{bclogo}[couleurBord=black, arrondi = 0.1, logo={}, barre = none]{Code python :}
%%	\begin{minipage}{\textwidth}
%%	\vspace{0.2cm}
%	}
%	{%\vspace{0.2cm}
%%	\end{minipage}
%	\end{bclogo}
%	\par\rmfamily\normalsize\vspace{0.2cm}\noindent
%	}

\newcommand{\pyth}[1]{\texttt{#1}}

% -------------------------- Boites bclogo -----------------------------------
\newenvironment{warning}
	{\begin{bclogo}[couleurBord=black, arrondi = 0.1, logo = \bcattention]}%{\Large\attention}]}
	{\end{bclogo}}
	
\newenvironment{forbidden}
	{\begin{bclogo}[couleurBord=black, arrondi = 0.1, logo = \bcinterdit]}%{\Large\interdit}]}
	{\end{bclogo}}

% ---------------------------------------------------------------------------
% 				Raccoucis clavier
% ---------------------------------------------------------------------------

\newcommand{\fctsurI}[3]{
	Soit $#1$ la fonction définie sur $#3$ par $$#1(x)=#2$$}
	
\newcommand{\hp}{Hors programme de TSTI2D}

% ---------------------------------------------------------------------------
%				Macros évoluées
% ---------------------------------------------------------------------------

% ----------------- \tournerpage --------------------------------------------
% Indique de tourner la page en bas de page
%
\def\tournerpage{\vfill%
	\begin{flushright}	
		\textbf{Tourner la page }$\mathbf{\rightarrow}$
	\end{flushright}
	\newpage}

% ------------------ Exercices : \exo et \pb --------------------------------
% Crée un exo (ou un pb) valant #1 points (si #1=0 alors correction)
% Les exos sont numérotés automatiquement
% 
\newcounter{numexo}
\setcounter{numexo}{0}

\newcommand\exo[1]{\addtocounter{numexo}{1}\par\vspace{1cm}\textbf{\textsc{Exercice \thenumexo}}%
	\ifthenelse{\equal{#1}{0}}{}{ \hfill \textbf{#1 points}}\medskip\par}
	
\newcommand\pb[1]{\par\vspace{1cm}\textbf{\textsc{Problème}}%
	\ifthenelse{\equal{#1}{0}}{}{\hfill \textbf{#1 points}}\medskip\par}

% ------------------ Cartouche : \makecartouche -------------------------------
% {Titre}{{Classe}{Durée}{Calculatrice autorisée}{Ligne pour le nom}

\newcommand{\makecartoucheDsNOM}[4]{
	\ifthenelse{\boolean{#4}}{\newcommand{\calculatrice}{est }}{\newcommand{\calculatrice}{\textbf{n'est pas }}}
	\begin{center}\begin{tabular}{|p{.82\linewidth}|p{.15\linewidth}|}
	\hline
	#3 & \multicolumn{1}{r|}{#2}\\
	%~ \multicolumn{2}{|l|}{#3}\\
	% ----- Nom Prénom ----
	\hline
%	\textbf{\textsc{Nom - Prénom :}} & \textbf{\textsc{Classe :}}\\
	\multicolumn{2}{|l|}{\textbf{\textsc{Nom - Prénom :}}}\\
	\multicolumn{2}{|l|}{}\\
	\multicolumn{2}{|l|}{}\\
	% ---------------------
	\hline
	\multicolumn{2}{|c|}{}\\
	\multicolumn{2}{|c|}{\textbf{\LARGE{#1}}}\\
	\multicolumn{2}{|c|}{}\\
	\hline
	\multicolumn{2}{|c|}{\textit{La calculatrice \calculatrice autorisée. Aucun autre document n'est autorisé.}} \\
	\hline
	\end{tabular}\end{center}
}

\newcommand{\makecartoucheDsPASNOM}[4]{
	\ifthenelse{\boolean{#4}}{\newcommand{\calculatrice}{est }}{\newcommand{\calculatrice}{\textbf{n'est pas }}}
	\begin{center}\begin{tabular}{|p{.82\linewidth}|p{.15\linewidth}|}
	\hline
	#3 & \multicolumn{1}{r|}{#2}\\
	\hline
	\multicolumn{2}{|c|}{}\\
	\multicolumn{2}{|c|}{\textbf{\LARGE{#1}}}\\
	\multicolumn{2}{|c|}{}\\
	\hline
	\multicolumn{2}{|c|}{\textit{La calculatrice \calculatrice autorisée. Aucun autre document n'est autorisé.}} \\
	\hline
	\end{tabular}\end{center}
}

\newcommand{\makecartoucheDS}[5]{\ifthenelse{\boolean{#5}}{\makecartoucheDsNOM{#1}{#2}{#3}{#4}}{\makecartoucheDsPASNOM{#1}{#2}{#3}{#4}}}

\newcommand{\makecartoucheCours}[2]{
	\begin{center}\begin{tabular}{|p{.82\linewidth}|p{.15\linewidth}|}
	\hline
	 & \multicolumn{1}{r|}{#2}\\
	\hline
	\multicolumn{2}{|c|}{}\\
	\multicolumn{2}{|c|}{\textbf{\LARGE{#1}}}\\
	\multicolumn{2}{|c|}{}\\
	\hline
	\end{tabular}\end{center}
}


% --------------------------------- Pages de cahier ------------------------------

% --------------------- \cahierseyes ---------------------------------------------
% Crée une page de cahier réglures seyes (8mm et interlignes) de #1 lignes
%

% Périmé
\newcounter{dimcahierX}
\setcounter{dimcahierX}{16}
\newcounter{dimcahiersY}
\newcommand\cahierseyes[1]{
	\vspace*{-1cm}
	\setcounter{dimcahiersY}{#1*\real{-0.8}}
	\begin{center}
	\begin{tikzpicture}
		\tkzInit[xmin=0,xmax=\thedimcahierX,ymin=\thedimcahiersY,ymax=0]
		\tkzGrid[xstep=0.8, ystep=0.8,sub, subxstep=0.8, subystep=0.2]
	\end{tikzpicture}
	\end{center}
}

% Bon
\newcounter{DimcahierX}
\setcounter{DimcahierX}{16}
\newcounter{DimcahierY}
\newcommand\Cahierseyes[1]{
%	\vspace*{-0.8cm}
	\setcounter{DimcahierY}{#1}
	\begin{center}
	\begin{tikzpicture}
	\draw[xstep=0.8cm, ystep=0.2cm, color=lightgray] (0,0) grid (\theDimcahierX, 0.8*\theDimcahierY);
	\draw[xstep=0.8cm, ystep=0.8cm, color=gray] (0,0) grid (\theDimcahierX, 0.8*\theDimcahierY);
	\end{tikzpicture}
	\end{center}
}


% --------------------- \cahierpetca ---------------------------------------------
% Crée une page de cahier petits carreaux (5mm) de #1 lignes
%

%Périmé
\newcounter{dimcahierpcY}
\newcommand\cahierpetca[1]{
	\setcounter{dimcahierpcY}{#1*\real{0.5}}
	\vspace*{-1cm}
	\begin{center}
	\begin{tikzpicture}
		\tkzInit[xmin=0,xmax=\thedimcahierX,ymin=0,ymax=\thedimcahierpcY]
		\tkzGrid[xstep=0.5, ystep=0.5]
	\end{tikzpicture}
	\end{center}
}

%Bon
\newcounter{DimcahierpcY}
\newcommand\Cahierpetca[1]{
	\setcounter{DimcahierpcY}{#1}
	%\vspace*{-1cm}
	\begin{center}
	\begin{tikzpicture}
	\draw[xstep=0.5cm, ystep=0.5cm, color=gray] (0,0) grid (\theDimcahierX, 0.5*\theDimcahierpcY);
	\end{tikzpicture}
	\end{center}
}



% --------------------- \cahierligne ---------------------------------------------
% Crée une page de cahier avec juste des lignes (10mm) de #1 lignes
%
\newcounter{dimcahierlY}
\newcounter{miX}
\setcounter{miX}{\thedimcahierX*\real{0.5}}
\newcommand\cahierligne[1]{
	\setcounter{dimcahierlY}{#1-1}
	\begin{center}
	\begin{tikzpicture}
		\tkzInit[xmin=0,xmax=\thedimcahierX,ymin=0,ymax=#1]
		\foreach \k in {0,1,...,\thedimcahierlY}
		{\draw[color=gray] (0,\k)--(16,\k);}
		\draw[color=lightgray] (\themiX,0)--(\themiX,#1);
	\end{tikzpicture}
	\end{center}
}

% --------------------- \cahier ---------------------------------------------
% Par défaut, réglures seyes
%
\newcommand\cahier[1]{\Cahierseyes{#1}}

% --------------------- \papiermilli ----------------------------------------
% Crée une feuille de papier millimétré de dimensions x=#1 par y=#2
%
\newcommand\papiermilli[2]{
	\begin{center}
	\begin{tikzpicture}
		\tkzInit[xmin=0,xmax=#1,ymin=0,ymax=#2]
		\tkzGrid[color=lightgray,xstep=1, ystep=1,sub, subxstep=0.1, subystep=0.1]
		\tkzGrid[color=gray,xstep=1, ystep=1,sub, subxstep=0.5, subystep=0.5]
		\tkzGrid[color=darkgray,xstep=5, ystep=5]
	\end{tikzpicture}
	\end{center}
}
% --------------------- \systlin ----------------------------------------
% Système linéaire 2*2
% Paramètres : a, ±b, c, a', ±b', c'
%
\newcommand{\systlin}[6]{
$\left\lbrace
\begin{array}{r @{x~} l @{y~=~} l l}
#1&#2&#3&(L_1)\\
#4&#5&#6&(L_2)
\end{array}
\right.$
}


% ---------------------------------------------------------------------------
%				Formatage
% ---------------------------------------------------------------------------

% --------------------- Chapitres -------------------------------------------
\addto\captionsfrench{\renewcommand{\chaptername}{~Chapitre}}

\titleformat{\chapter}[frame]
	{\vspace*{-2cm}\titleline[r]{}\normalfont}
	{\filright\texttt\LARGE{~\chaptername~\thechapter~}}
	{5pt}
	{\Huge\bfseries\scshape\filcenter}
	{}
	
%~ \titleformat{\chapter}[display]
%~ {\normalfont\Large\filcenter}
%~ {\titlerule[1pt]%
 %~ \vspace{1pt}%
 %~ \titlerule
 %~ \vspace{1pc}%
 %~ \LARGE{\chaptertitlename} \thechapter}
%~ {1pc}
%~ {\titlerule
 %~ \vspace{1pc}%
 %~ \Huge\bfseries}
\renewcommand{\appendixname}{Annexe}
% --------------------  Divisions logiques ----------------------------------

\setcounter{secnumdepth}{4} \setcounter{tocdepth}{3}
\renewcommand{\thepart}{Partie \arabic{part}}
\renewcommand{\thechapter}{\arabic{chapter}}
\renewcommand{\thesection}{\arabic{section}}
\renewcommand{\thesubsection}{\arabic{section}.\arabic{subsection}}
\renewcommand{\thesubsubsection}{\arabic{section}.\arabic{subsection}.\arabic{subsubsection}}

% -------------------- Numérotations des questions et puces -----------------

%\renewcommand{\theenumi}{\textbf{\arabic{enumi}}}
%\renewcommand{\labelenumi}{\textbf{\theenumi.}}
%\renewcommand{\theenumii}{\textbf{\alph{enumii}}}
%\renewcommand{\labelenumii}{\textbf{\theenumii.}}
%\renewcommand{\labelenumiii}{\textbullet}

\AtBeginDocument{\renewcommand{\labelitemi}{\textbullet}}
\AtBeginDocument{\renewcommand{\labelitemii}{\textbullet}}


% % -------------------- Mise en page --------------------------------------- 

\usepackage[francais]{layout}
%\usepackage[a4paper]{geometry}


% % -------------------- Réglages divers------------------------------------- 

\everymath{\displaystyle\everymath{}}		% Toutes les équations en mode \displaystyle
\DecimalMathComma							% Virgule comme séparateur décimal
\frenchspacing								% Espaces français
%\setlength{\parindent}{0pt}					% Pas d'indentation de paragraphes

\colorlet{gris1}{black!20}
\colorlet{gris2}{black!30}

% ----------------------------------------------------------------------------

\lstset{literate=
  {á}{{\'a}}1 {é}{{\'e}}1 {í}{{\'i}}1 {ó}{{\'o}}1 {ú}{{\'u}}1
  {Á}{{\'A}}1 {É}{{\'E}}1 {Í}{{\'I}}1 {Ó}{{\'O}}1 {Ú}{{\'U}}1
  {à}{{\`a}}1 {è}{{\`e}}1 {ì}{{\`i}}1 {ò}{{\`o}}1 {ù}{{\`u}}1
  {À}{{\`A}}1 {È}{{\'E}}1 {Ì}{{\`I}}1 {Ò}{{\`O}}1 {Ù}{{\`U}}1
  {ä}{{\"a}}1 {ë}{{\"e}}1 {ï}{{\"i}}1 {ö}{{\"o}}1 {ü}{{\"u}}1
  {Ä}{{\"A}}1 {Ë}{{\"E}}1 {Ï}{{\"I}}1 {Ö}{{\"O}}1 {Ü}{{\"U}}1
  {â}{{\^a}}1 {ê}{{\^e}}1 {î}{{\^i}}1 {ô}{{\^o}}1 {û}{{\^u}}1
  {Â}{{\^A}}1 {Ê}{{\^E}}1 {Î}{{\^I}}1 {Ô}{{\^O}}1 {Û}{{\^U}}1
  {œ}{{\oe}}1 {Œ}{{\OE}}1 {æ}{{\ae}}1 {Æ}{{\AE}}1 {ß}{{\ss}}1
  {ű}{{\H{u}}}1 {Ű}{{\H{U}}}1 {ő}{{\H{o}}}1 {Ő}{{\H{O}}}1
  {ç}{{\c c}}1 {Ç}{{\c C}}1 {ø}{{\o}}1 {å}{{\r a}}1 {Å}{{\r A}}1
  {€}{{\EUR}}1 {£}{{\pounds}}1
}
\definecolor{mygreen}{rgb}{0,0.6,0}
\definecolor{mygray}{rgb}{0.5,0.5,0.5}
\definecolor{mymauve}{rgb}{0.58,0,0.82}
\lstset{ %
  backgroundcolor=\color{white},   % choose the background color; you must add \usepackage{color} or \usepackage{xcolor}
  basicstyle=\footnotesize,        % the size of the fonts that are used for the code
  breakatwhitespace=false,         % sets if automatic breaks should only happen at whitespace
  breaklines=true,                 % sets automatic line breaking
  captionpos=b,                    % sets the caption-position to bottom
  commentstyle=\color{mygreen},    % comment style
  deletekeywords={...},            % if you want to delete keywords from the given language
  escapeinside={\%*}{*)},          % if you want to add LaTeX within your code
  extendedchars=true,              % lets you use non-ASCII characters; for 8-bits encodings only, does not work with UTF-8
  frame=single,	                   % adds a frame around the code
  keepspaces=true,                 % keeps spaces in text, useful for keeping indentation of code (possibly needs columns=flexible)
  keywordstyle=\color{blue},       % keyword style
  language=Octave,                 % the language of the code
  otherkeywords={*,...},           % if you want to add more keywords to the set
  numbers=left,                    % where to put the line-numbers; possible values are (none, left, right)
  numbersep=5pt,                   % how far the line-numbers are from the code
  numberstyle=\tiny\color{mygray}, % the style that is used for the line-numbers
  rulecolor=\color{black},         % if not set, the frame-color may be changed on line-breaks within not-black text (e.g. comments (green here))
  showspaces=false,                % show spaces everywhere adding particular underscores; it overrides 'showstringspaces'
  showstringspaces=false,          % underline spaces within strings only
  showtabs=false,                  % show tabs within strings adding particular underscores
  stepnumber=2,                    % the step between two line-numbers. If it's 1, each line will be numbered
  stringstyle=\color{mymauve},     % string literal style
  tabsize=2,	                   % sets default tabsize to 2 spaces
%  title=\lstname                   % show the filename of files included with \lstinputlisting; also try caption instead of title
}

% =============================================================================
% 								FIN PREAMBULE
% =============================================================================
